
\chapter{Preface}

Welcome to the art of computer programming and to the 
new Perl~6 programming language. This will probably be 
the first or one of the very first published books using Perl~6, 
a powerful, expressive, malleable and highly extensible 
programming language. But this book is not so much 
about Perl~6, and more about learning 
how to write programs for computers. 

This book is intended for beginners and does not require 
any prior programming knowledge, but it is my hope 
that even those of you with programming experience will 
benefit from reading it.

\section*{The aim of this book}

This aim of this book is not primarily to teach Perl~6, 
but instead to teach the art 
of programming, using the Perl~6 language. After having 
completed this book, you should hopefully be able 
to write programs to solve relatively difficult problems in 
Perl~6, but our main aim is to teach computer science, software 
programming and problem solving more than to teach the Perl~6 
language itself. 

This means that we will not cover every aspect of Perl~6, but 
only a (relatively large, but yet incomplete) subset of it. 
By no means is this book intended to be a reference on the 
language.

It is not possible to learn programming or to learn a new 
programming language by just reading a book; practicing 
is essential. This book contains a lot of exercises. You 
are strongly encouraged to really try to do them. And, 
whether successful or not in solving the exercises, you 
should take a look at the solutions in the Appendix, 
as, very often, several solutions are suggested with further 
discussion on the subject and the issues involved. Sometimes, the solution 
section of the Appendix also introduces examples of topics 
that will be covered in the next chapter--and sometimes even 
things that will not be covered elsewhere in the book. So, 
please, really try to solve the exercises and also review 
the solutions, read them and try them. 

There are more than one thousand code examples in this book; 
study them, make sure to understand them and to run them. When 
possible, try to change them and see what happens. You're 
likely to learn a lot during this process.


\section*{The history of this book}

In the course of the last three to four years, I  
have translated or adapted to French a number of tutorials 
and articles on Perl 6, and I've also written a few entirely 
new ones in French~\footnote{See for example 
\url{http://perl.developpez.com/cours/\#TutorielsPerl6}.}. 
Together, these documents represented by the end of 2015 
somewhere between 250 and 300 pages of material on Perl~6. 
By that time, I had 
probably made public more material on Perl~6 in French than 
all other authors taken together.

Late 2015, I was feeling that a Perl~6 document for beginners 
was something missing that I was willing to undertake. 
I looked around and found that it did not seem to 
exist in English either. I came to the idea that, after all, 
it might be more useful to write such a document initially 
in English, to give it a broader audience. I was therefore 
contemplating to write a beginner introduction to Perl~6 
programming, my idea at the time was something like a 50 to 
70-page tutorial and I had started to gather material and ideas 
in this direction.

Then, something happened that changed my plans.

In December 2015, friends of mine were contemplating translating 
into French Allen B. Downey's \emph{Think Python, second edition}\footnote{See \url{http://greenteapress.com/wp/think-python-2e/}.}. 
I had read an earlier edition of that book and fully supported 
the idea of translating this excellent book\footnote{I know, it's 
about Python, not Perl. But not only I don't want to engage 
in ``language wars'' and think that we all have to learn from 
other languages, but, to me, Perl's motto, ``there is more than 
one way to do it'', also means that doing it in Python (or some 
other language) is truly an acceptable possibility.}. As it 
turned out, I ended up being a co-translator and the technical 
editor of the French translation of that book\footnote{See 
\url{http://allen-downey.developpez.com/livres/python/pensez-python/}.}.

While working on the French translation of Allen's Python book, 
the idea came to me that, rather than writing a tutorial on 
Perl~6, it might be more useful to make a ``Perl~6 translation'' 
of that book. Since I was in contact with Allen in the context 
of the French translation, I suggested this to Allen, who 
warmly welcomed the idea. This is how I started to write this 
book late January 2016, just after having completed the 
work on the Python book French translation.

This book is thus largely derived on Allen's \emph{Think Python}, 
but adapted to Perl~6. As it happened, it is also much more 
than just a ``Perl~6 translation'' of Allen's book, with 
quite a lot of new material, it has become a brand new book, 
largely indebted to Allen's book, but yet a new book for which 
I take all responsibility. Any error is obviously my fault, 
not Allen's.

My hope is that this will be useful to the Perl~6 community, and 
more broadly to the open source community and to general 
computer programming community. In an interview with 
\emph{LinuxVoice} (July 2015), Larry Wall, the creator of Perl~6, 
said: ``We do think that Perl 6 will be learnable as a first language.''
Hopefully this book will contribute to make this happen. 

\section*{Acknowledgments}

I just don't know how I could thank Larry Wall to the level of 
gratitude that he deserves for having created Perl in the first 
place, and Perl~6 more recently. Be blessed for eternity, Larry, 
for all of that. 

And thank you all of you who took part to this 
adventure (in no particular order), Tom, Damian, 
chromatic, Nathan, brian, Jan, Jarkko, John, Johan, Randall, 
Mark Jason, Ovid, Nick, Tim, Andy, Chip, Matt, Michael, Tatsuhiko, 
Dave, Rafael, Chris, Stevan, Saraty, Malcolm, Graham, Leon, 
Ricardo, Gurusamy, Scott and too many others to name.  

All my thanks also to those who believed in 
this Perl~6 project and made it happen, including those who 
quit at one point or another but contributed for some 
time, and I know that this wasn't always easy.

Many thanks to Allen Downey, who very kindly supported my idea of 
adapting his book to Perl~6 and helped me in many respects, but 
also refrained from interfering into what I was putting into 
this new book.

I very warmly thank the people at O'Reilly who accepted the 
idea of this book and who suggested many corrections or 
improvements. I want to thank especially 
Dawn Schanafelt, O'Reilly's editor of this book, whose advice 
have truly contributed to make this a better book.

Thanks a lot in advance to readers who will offer comments 
or submit suggestions or corrections, as well as encouragements.

If you see anything that would need to be corrected or that 
could be improved, please kindly send your comments to the 
following address: \url{think.perl6@gmail.com}.
% ...


\section*{Contributor List}

% ...
I would like to thank especially Moritz Lenz and Elizabeth 
Mattijsen who reviewed in detail drafts of this book 
and suggested quite a number of improvements and corrections. Liz 
spent really a lot of time on a detailed review of the full 
content of this book and I am especially grateful to her for 
her numerous and very useful comments. Thanks also to Timo Paulssen and 
ryanschoppe who also reviewed early drafts and provided some  
useful suggestions.


\clearemptydoublepage

% TABLE OF CONTENTS
\begin{latexonly}

\tableofcontents

\clearemptydoublepage

\end{latexonly}

