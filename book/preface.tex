
\chapter{Preface}

\emph{Important note:} this preface is not intended to be the 
final preface of this book, but more to give useful information 
to reviewers and early readers of its draft.

\section*{Warning to the reader}

If you're reading this section, it means that you've been 
given an \emph{early draft} of this book, not 
intended to be made public at this point. You may transmit 
this book to others privately if you think this may serve 
some useful purpose, but please do not make 
it public or even semi-public without first consulting 
me. My hope is to obtain comments, suggested corrections or 
any other opinions about the content of this document in 
order to improve it before making it available to the 
public.

This document contains at this point the complete intended 
book. Only early chapters have been reviewed by others, the 
later chapters still need quite a bit of proof reading.
\index{programming}

\section*{The aim of this book}

This aim of this book is not primarily to teach Perl~6. 
The foremost aim of this book is to try to teach the art 
of programming, using the Perl~6 language. After having 
completed this book, the reader should hopefully be able 
to write programs to solve relatively difficult problems in 
Perl~6, but our main aim is to teach computer science, software 
programming and problem solving more than to teach the Perl~6 
language itself. 

This means that we will not cover every aspect of Perl~6, but 
only a (relatively large, but yet incomplete) subset of it. 
By no means is this book intended to be a reference on the 
language.

It is not possible to learn programming or to learn a new 
programming language by just reading a book; practicing 
is essential. This book contains a lot of exercises. The 
reader is strongly encouraged to really try to do them. And, 
whether successful or not in solving the exercises, the reader 
should really take a look at the solutions in the Appendix, 
as, very often, several solutions are suggested with further 
discussion on the issues involved. Sometimes, the solution 
section of the Appendix is also introducing with examples topics 
that will be covered in the next chapter--and sometimes even 
things that will not be covered elsewhere in the book. So, 
please, really try to solve the exercises and also go to 
the solutions, read them and try them.

\section*{Inaccuracies and errors}

Since this book is largely intended for beginners, some of 
the explanations might not be entirely accurate, especially 
in the eyes of language lawyers: you can't always tell all 
the truth to children; sometimes, you need to simplify things 
to explain them to the layman. Hopefully most of these early 
simplifications are refined in later chapters, but perhaps 
not all of them.

Another reason for possible inaccuracies is that I am much 
more used to Perl~5, my daily programming language at work, 
than to Perl~6. There may be a few cases where I explain 
things in Perl~5 terms rather than in Perl~6 terms. My hope 
is that early readers of this document will point out these 
errors or inaccuracies to help me correct them.

Also, as explained in the next section just below, this book
is derived from Allen Downey's book on Python. Despite 
my efforts, there may be a place here or there where I left 
a comment related to Python that would not really hold for Perl~6. 
Again, I hope that early readers will help me spot such 
cases and correct them.

Many thanks in advance to readers who will offer comments 
or submit suggestions or corrections, as well as encouragements.

Please send your comments to the following address: 
think.perl6@gmail.com.

\section*{The history of this book}

In the course of the last two to three years, I (Laurent R.) 
have translated or adapted to French a number of tutorials 
and articles on Perl 6, and I've also written a few entirely 
new ones in French (see for example 
\url{http://perl.developpez.com/cours/#TutorielsPerl6}). 
Together, these documents represent somewhere between 250 and 300 pages of material on Perl~6. By the end of 2015, I had 
probably made public more material on Perl~6 in French than 
all other authors taken together.

Late 2015, I was feeling that a Perl~6 document for beginners 
was something missing that I was willing to undertake. At that 
time, I looked around and found that it did not seem to 
exist in English either. I came to the idea that, after all, 
it might be more useful to write such a document initially 
in English, to give it a broader audience. I was therefore 
contemplating to write a beginner introduction to Perl~6 
programming, my idea at the time was something like a 50 to 
70-page tutorial and I had started to gather material and ideas 
in this direction.

Then, something happened that changed my plans.

To me, TIMTOWTDI is really not an empty slogan. And, although 
I really love Perl~5 and even more Perl~6, this motto might 
also mean to do it in another language if needed, as the 
case may be.

In December 2015, friends of mine were contemplating translating 
into French Allen B. Downey's \emph{Think Python--how to think 
like a computer scientist, second edition} 
(\url{http://greenteapress.com/wp/think-python-2e/}). I had 
read an earlier edition and fully supported 
the idea of translating this excellent book. As it turned out, 
I ended up being a co-translator and the technical editor of 
the French translation of that book (\url{http://allen-downey.developpez.com/livres/python/pensez-python/}).

While working on the French translation of Allen's Python book, 
the idea came to me that, rather than writing a tutorial on 
Perl~6, it might be more useful to make a ``Perl~6 translation'' 
of that book. Since I was in contact with Allen in the context 
of the French translation, I suggested this to Allen, who 
warmly welcomed the idea. This is how I started to write this 
book late January 2016, just after having completed the 
work on the Python book French translation.

This book is thus largely derived on Allen's \emph{Think Python--how 
to think like a computer scientist}, but adapted to Perl~6. My 
hope is that this will be useful to the Perl~6 community, and 
more broadly to the general computer programming community. In 
an interview with \emph{LinuxVoice} (July 2015), Larry Wall said: 
``We do think that Perl 6 will be learnable as a first language.''
Hopefully this book will contribute to make this happen. 

Any error on Perl~6 (especially in the coding examples and 
solutions to the exercises) is obviously my error, not 
Allen's. And most other errors are quite probably also mine.

\section*{Acknowledgments}

I just don't know how I could thank Larry Wall to the level of 
gratitude that he deserves for having created Perl in the first 
place, and Perl~6 more recently. Be blessed for eternity, Larry, 
for all of that. 

And thank you all of you who took part to this 
adventure (in no particular order), Tom, Damian, 
chromatic, Nathan, brian, Jan, Jarkko, John, Johan, Randall, 
Mark Jason, Ovid, Nick, Tim, Andy, Chip, Matt, Michael, Tatsuhiko, 
Dave, Rafael, Chris, Stevan, Saraty, Malcolm, Graham, Leon, 
Ricardo, Gurusamy, Scott and so many others that I can't name.  

All my thanks also to those who believed in 
this Perl~6 project and made it happen, including those who 
quit at one point or another but contributed for some 
time, and I know that this wasn't always easy.

Many thanks to Allen Downey, who kindly supported my idea of 
adapting his book to Perl~6.

% ...


\section*{Contributor List}

% ...
I would like to thank especially Moritz Lenz and Elizabeth 
Mattijsen who reviewed in detail drafts of this book 
(only early chapters at this point in time) and suggested 
quite a number of improvements and corrections. Timo Paulssen and 
ryanschoppe alse reviewed early drafts and provided very 
useful comments.


\clearemptydoublepage

% TABLE OF CONTENTS
\begin{latexonly}

\tableofcontents

\clearemptydoublepage

\end{latexonly}

