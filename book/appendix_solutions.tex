\chapter{Solutions to the exercises}

This (long) chapter provides solutions to the exercises 
suggested in this book main matter. However, it contains 
much more than that.

First, in many cases, it provides several different solutions, 
illustrating different approaches to a problem, discussing 
their respective merits or drawbacks and often 
showing solutions that may be more efficient than others.

Second, it often provides a lot of additional information 
or complementary examples.

Just the sheer volume of code examples of this chapter is 
likely to teach you a lot about programming in general and 
about the Perl~6 language in particular.

Finally, this chapter sometimes introduces (with examples) new 
concepts that are covered only in later chapters of the book. 
Having seen such examples may help you to get a smoother 
grasp to these new ideas when you get to these chapters. 
In a few cases, this chapter covers or introduces notions 
that will not be covered anywhere else in the book.

When you solve an exercise, even if you're confident that you 
did it successfully, please make sure to consult the solutions 
in this chapter and to try them, you're likely to learn quite 
a bit from them.

\section{Exercises of Chapter~\ref{funcchap}: Functions and subroutines}

\subsection{Exercise~\ref{right_justify}: subroutine right-justify (p.~\pageref{right_justify})}
\label{sol_right_justify}
\index{right-justify}

The aim is to write a subroutine that prints a string
with enough leading spaces so that the last letter 
of the string is in column 70 of the display.

This is the first real exercise of this book, so let's first 
do it step by step:

\begin{verbatim}
use v6;
sub right-justify ($input-string) {
    my $str_length = chars $input-string;
    my $missing_length = 70 - $str_length;
    my $leading-spaces = " " x $missing_length;
    say $leading-spaces, $input-string;
}
right-justify("Larry Wall");
right-justify("The quick brown fox jumps over the lazy dog");
\end{verbatim}

This subroutine:
\begin{itemize}
\item assigns the input string length to the \verb'$str_length' variable;
\item computes into to the \verb'$missing_length' variable the 
number of spaces that will need to be added at the beginning 
of the displayed line to have it end in column 70;
\item creates the \verb'$leading-spaces' string with the 
needed number of spaces;
\item prints out the \verb'$leading-spaces' and 
\verb'$input-string' one after the other to obtain the desired 
result.
\end{itemize}

This displays the following:

\begin{verbatim}
                                                            Larry Wall
                           The quick brown fox jumps over the lazy dog
\end{verbatim}

We can, however, make this code shorter by composing some of the 
statements and expressions:
\index{composition}

\begin{verbatim}
sub right-justify ($input-string) {
    my $leading-spaces = " " x (70 - $input-string.chars);
    say $leading-spaces, $input-string;
}
\end{verbatim}

It could even be boiled down to a shorter single-line subroutine:

\begin{verbatim}
sub right-justify ($input-string) {
    say " " x 70 - $input-string.chars, $input-string;
}
\end{verbatim}

This works fine, but it may be argued that this last version is 
less clear. In fact, the \verb'$leading-spaces' temporary 
variable used in the previous version had a name that 
self-documented what the subroutine is doing. You can make 
very concise code as in the last example above, but sometimes 
it may become a little bit too terse, so there is a trade-off 
between concision and clarity.

Note that there are two built-in functions, {\tt printf} and 
{\tt sprintf} that can perform a similar task, but we will 
cover them later. There is also a \verb'.fmt' method for 
producing formatted output.
\index{printf function}
\index{sprintf function} 
\index{fmt method}

\subsection{Exercise~\ref{do_it_twice}: subroutine do-twice (p.~\pageref{do_it_twice})}
\label{sol_do_it_twice}
\index{do-twice}

To add an addressee to the greeting, we need to:
\begin{itemize}
\item pass a second argument in the call to 
{\tt do-twice} (the string ``World'');
\item add a new parameter in the {\tt do-twice} subroutine 
signature (\verb'$addressee');
\item add this new parameter as an argument in the calls 
to \verb'$code';
\item add a signature with one parameter (\verb'$addr') in 
the definition of the {\tt greet} subroutine; and 
\item use this new parameter in the print statement.
\end{itemize} 

This leads to the following code:

\begin{verbatim}
sub do-twice($code, $addressee) {
    $code($addressee); 
    $code($addressee);
}
sub greet (Str $addr) {
    say "Hello $addr!";
}
do-twice &greet, "World";
\end{verbatim}

This displays:

\begin{verbatim}
Hello World!
Hello World!
\end{verbatim}

For the next question, we replace the {\tt greet} subroutine 
by the {\tt print-twice} subroutine:

\begin{verbatim}
sub do-twice($code, $message) {
    $code($message); 
    $code($message);
}
sub print-twice($value) {
    say $value;
    say $value
}
do-twice &print-twice, "What's up doc";
\end{verbatim}

This prints the ``What's up doc'' four times.

Finally, we add the new {\tt do-four} subroutine and let 
it call twice the {\tt do-twice} subroutine, to print the 
message eight times:

\begin{verbatim}
sub do-twice($code, $message) {
    $code($message); 
    $code($message);
}
sub print-twice($value) {
    say $value;
    say $value
}
sub do-four ($code, $message) {
    do-twice $code, $message;
    do-twice $code, $message;
}
do-four &print-twice, "What's up doc";
\end{verbatim}

\subsection{Exercise~\ref{draw_a_grid}: subroutine print-grid 
(p.~\pageref{draw_a_grid})}
\label{sol_draw_a_grid}
\index{print-grid}

To print a grid such as the one requested in the exercise, 
we need to print each line one by one, and we basically 
have two types of lines: the three ``dotted lines'' and 
the eight lines without dashes, which we'll call ``empty 
lines'' for lack of a better name, because they are are 
partly empty (no dashes).

To avoid code repetition, one way to do it is to create 
a string for each of the two line types and to print 
these strings in accordance with the needs.

This is one possible solution:

\begin{verbatim}
use v6;

my $four-dashes = "- " x 4;
my $dotted_line = ("+ " ~ $four-dashes ) x 2 ~ "+" ;
my $spaces = " " x 9;
my $empty-line = ("|" ~ $spaces ) x 2 ~ "|" ;

sub say-four-times($value) {
    say $value;
    say $value;
    say $value;
    say $value;
}
sub print-grid {
    say $dotted_line;
    say-four-times $empty-line;
    say $dotted_line;
    say-four-times $empty-line;
    say $dotted_line;
}
print-grid;
\end{verbatim}

There are obviously better ways to do something four times 
than just repeating \verb'say $value;' four times as in the 
{\tt say-four-times} above, but this will be covered in 
the Chapter\ref{conditionals} (see Section~\ref{for_loops}).

To draw a similar grid with four rows and four columns, we need 
first to modify the strings used for printing the lines:

\begin{verbatim}
my $dotted_line = ("+ " ~ $four-dashes ) x 4 ~ "+" ;
# ...
my $empty-line = ("|" ~ $spaces ) x 4 ~ "|" ;
\end{verbatim}

In addition to that, we could modify {\tt print-grid} to just print 
each line the required number of times. But that would involve 
quite a bit of code repetition, and the aim of this exercise 
is to use subroutines to permit code reuse.

There are now two things that we repeatedly need to do four times. 
It makes sense to write a {\tt do-four-times} subroutine that 
will be used for creating both the {\tt print-four-times} (in 
charge of printing the four empty lines) and for calling 
four times entire rows. This subroutine will be passed the 
code reference for doing the specific actions required.

\begin{verbatim}
my $four-dashes = "- " x 4;
my $dotted_line = ("+ " ~ $four-dashes ) x 4 ~ "+" ;
my $spaces = " " x 9;
my $empty-line = ("|" ~ $spaces ) x 4 ~ "|" ;

sub do-four-times ($code) {
    $code();
    $code();
    $code();
    $code();
}
sub say-four-times($value) {
    do-four-times(sub {say $value});
}
sub print-bottom-less-grid {
    say $dotted_line;
    say-four-times $empty-line;
}
sub print-grid {
    do-four-times(&print-bottom-less-grid);
    say $dotted_line;
}
print-grid;
\end{verbatim}

In addition, rather than declaring global variables for the 
line strings, it is better practice to declare and define them 
within the subroutines where they are used. We also no longer 
need the {\tt print-four-times} subroutine, we can just 
pass the relevant arguments to the {\tt print-four-times} 
subroutine to the same effect. This could lead to the 
following program:

\begin{verbatim}
sub do-four-times ($code, $val) {
    $code($val);
    $code($val);
    $code($val);
    $code($val);
}
sub print-bottom-less-grid($dot-line) {
    say $dot-line;
    my $spaces = " " x 9;
    my $empty-line = ("|" ~ $spaces ) x 4 ~ "|" ;
    do-four-times(&say, $empty-line);
}
sub print-grid {
    my $four-dashes = "- " x 4;
    my $dotted_line = ("+ " ~ $four-dashes ) x 4 ~ "+" ;
    do-four-times(&print-bottom-less-grid, $dotted_line);
    say $dotted_line;
}
print-grid;
\end{verbatim}
\index{print-grid}


\section{Exercises of Chapter~\ref{conditionals}: Conditionals and recursion}

\label{sol_do_n_times}

\subsection{Subroutine do-n-times, exercise suggested in Section~\ref{do_n_times} (p.~\pageref{do_n_times})}

We need a subroutine that takes a function 
and a number, {\tt \$num}, as arguments, and that calls
the given function {\tt \$num} times.

The {\tt do-n-times} subroutine is recursive and is 
calling itself each time with a decremented argument. It 
stops ``recursing'' when this argument is 0. \verb'$subref' is an anonymous subroutine called within the body of 
{\tt do-n-times}:

\begin{verbatim}
sub do-n-times ($coderef, Int $num) {
     return if $num <= 0;
     $coderef();
     do-n-times $coderef, $num - 1;
}

my $subref = sub { say "Carpe diem";}

do-n-times $subref, 4;
\end{verbatim}

Which prints:

\begin{verbatim}
Carpe diem
Carpe diem
Carpe diem
Carpe diem
\end{verbatim}

\subsection{Exercise~\ref{int_div_modulo}: days, hours, 
minutes and seconds (p.~\pageref{int_div_modulo})}
\label{sol_int_div_modulo}

The following is one possible way of converting a number 
of seconds into a number of days, hours, minutes and seconds:
\index{modulo}
\index{operator!mod}
\index{integer division}
\index{operator!div}


\begin{verbatim}
days-HMS(240_000);

sub days-HMS (Int $seconds) {
    my $minutes = $seconds div 60;
    my $sec_left = $seconds mod 60;
    my ($hours, $min_left) = $minutes div 60, $minutes mod 60;
    my ($days, $hours_left) = $hours div 24, $hours mod 24;
    say "$days $hours_left $min_left $sec_left"; 
              #  prints: 2 18 40 0
}
\end{verbatim}

The first two lines do the integer division and modulo 
operation separately. For the next two cases, we do both 
operations in one single line, using a list syntax.

The \verb'$minutes', \verb'$hours' and \verb'$days' variables 
are all computed in essentially the same way. The code could 
be made more modular by  using a subroutine to compute 
\verb'$minutes', \verb'$hours' and \verb'$days'. Although 
fruitful subroutines will really be studied in the course 
of next chapter, we have seen a couple of examples of 
them and can provide a gist about how they could be used:

\begin{verbatim}
sub div_mod (Int $input, Int $num-base) {
    return $input div $num-base, $input mod $num-base;
}
sub days-HMS (Int $seconds) {
    my ($minutes, $sec_left) = div_mod $seconds, 60;
    my ($hours, $min_left)   = div_mod $minutes, 60;
    my ($days, $hours_left)  = div_mod $hours, 24;
    say "$days $hours_left $min_left $sec_left"; 
}
\end{verbatim}

To ask a user to enter a number of seconds, you might 
do this:
\index{prompt}
\index{function!prompt}

\begin{verbatim}
my $sec = prompt "Please enter the number of seconds: ";
days-HMS $sec.Int;
\end{verbatim}

In real life, it would usually be good to verify that the 
user-provided input is a positive integer and ask again if it is 
not. As a further exercise, you might try to insert the 
above code into a recursive subroutine that prints an 
error message and calls itself again if the user input is 
not valid. The solution to the next exercise 
(Section~\ref{sol_fermat_ex}) gives an example of a recursive 
subroutine designed to prompt 
the user to supply input again, this might help you figure 
out how to do it if you encounter difficulties. 
\index{input!validation}

Try replacing the following code line:
\begin{verbatim}
say "$days $hours_left $min_left $sec_left"; 
\end{verbatim}
%
with this one:
\index{printf function}

\begin{verbatim}
printf "%d days %d hours %d minutes %d seconds \n", days-HMS 240_000;
\end{verbatim}
to see a better formatted output.

\subsection{Exercise~\ref{fermat_ex}: Fermat's theorem (p.~\pageref{fermat_ex})}
\label{sol_fermat_ex}
\index{Fermat, Pierre}
\index{Fermat's Last Theorem}

The {\tt check-fermat} subroutine checks whether:

\[ a^n + b^n = c^n \]
%

is true for the supplied values of ``a'', ``b'', ``c'' and 
``n''.

\begin{verbatim}
sub check-fermat (Int $a, Int $b, Int $c, Int $n) {
    if $a**$n + $b**$n == $c**$n {
        if $n > 2 {
            say "Holy smokes, Fermat was wrong!" if $n > 2;
        } elsif $n == 2 or $n ==1 {
            say "Correct";
        }
        return True; 
    }
    return False
}

say "Correct for 3, 4, 5, 2" if check-fermat 3, 4, 5, 2;
get-input();

sub get-input {
    say "Your mission, Jim, should you decide to accept it, is to ";
    say "provide values of A, B, C and n satisfying Fermat's equation:";
    say "  A ** n + B ** n = C * *n";
    my $a = prompt "Please provide a value for A: ";
    my $b = prompt "Please provide a value for B: ";
    my $c = prompt "Please provide a value for C: ";
    my $n = prompt "Please provide a value for the exponent: ";
    if check-fermat($a.Int, $b.Int, $c.Int, $n.Int) {
        say "The equation holds true for your values";
    } else {
        say "Nope. The equation is not right."
    }
    my $try-again = prompt "Want to try again (Y/N)?";
    get-input if $try-again eq 'Y';
}
\end{verbatim}
%

Fermat's last theorem has been proven and, needless to say, 
the mission is truly impossible if $n > 2$; perhaps this 
time Jim Phelps should decline to accept the mission.

\subsection{Exercise~\ref{triangle}: is it a triangle? (p.~\pageref{triangle})}
\label{sol_triangle}

This is a possible routine to find out whether you can 
make a triangle with three given stick lengths

\begin{verbatim}
sub is-triangle ( Numeric $x, Numeric $y, Numeric $z) {
    my $valid = True;
    $valid = False if $x > $y + $z;
    $valid = False if $y > $x + $z;
    $valid = False if $z > $x + $y;
    if $valid {
        say "Yes"; 
    } else {
        say "No";
    }
}
is-triangle 1, 3, 4;  # -> Yes
is-triangle 1, 3, 6;  # -> No
\end{verbatim}

Another way to do this would be to start by finding the greatest 
length and test only that one, but that does not make the 
algorithm significantly simpler.

Prompting the user to input three length has been shown in 
the previous two exercises, there is nothing really new here.
However, this is one new way of doing it:

\begin{verbatim}
my ($a, $b, $c) = split " ", 
    prompt "Please enter three lengths (separated by spaces): ";
is-triangle $a.Int , $b.Int , $c.Int;
\end{verbatim}


\subsection{Exercise~\ref{fibonacci}: The Fibonacci numbers? (p.~\pageref{fibonacci})}
\label{sol_fibonacci}
\index{Fibonacci}
\index{Fibonacci!numbers}

The Fibonacci numbers are a sequence of numbers in which the 
first two numbers are equal to 1 and any subsequent number is the 
sum of the two preceding ones, for example:
\begin{verbatim}
1, 1, 2, 3, 5, 8, 13, 21, 34, ...
\end{verbatim}

Printing the first 20 Fibonacci numbers:

\begin{verbatim}
sub n_fibonacci (Int $n) {
    my $fib1 = 1;
    my $fib2 = 1;
    say $_ for $fib1, $fib2;
    for 3..$n {
        my $new_fib = $fib1 + $fib2;
        say $new_fib;
        ($fib1, $fib2) = $fib2, $new_fib;
    }
}
n_fibonacci 20;
\end{verbatim}

Printing the $n^{th}$ Fibonacci number:

\begin{verbatim}
my $n = prompt "Enter the searched Fibonacci number: ";
$n = $n.Int;
say fibo($n);

sub fibo (Int $n) {
    my ($fib1, $fib2) = 1, 1;
    for 3..$n {
        my $new_fib = $fib1 + $fib2;
        ($fib1, $fib2) = $fib2, $new_fib;
    }
    return $fib2;
}    
\end{verbatim}
\index{Fibonacci!numbers}

\subsection{Exercise~\ref{sub_recurse}: The {\tt recurse} subroutine (p.~\pageref{sub_recurse})}
\label{sol_sub_recurse}
\index{recursion}

Examining the code of the {\tt recurse} subroutine, the 
first thing you should notice is that each time it 
is called recursively, the first argument (\verb'$n') is 
decremented by one compared to the previous call. If the 
initial value of \verb'$n' is a \emph{positive integer}, the 
succession of calls will eventually lead to the base case 
where \verb'$n == 0', and the cascade of recursive calls 
will eventually stop. 

If \verb'$n' is not an integer or if it is negative, we will get 
into infinite recursion.

One way to visualize how the program runs is to display 
the subroutine parameters at each call:

\begin{verbatim}
sub recurse($n,$s) {
    say "Args : n = $n, s = $s";
    if ($n == 0) {
        say $s;
    } else {
        recurse $n - 1, $n + $s;
    }
}
recurse 3, 0;
\end{verbatim}
which would print:
\begin{verbatim}
Args : n = 3, s = 0
Args : n = 2, s = 3
Args : n = 1, s = 5
Args : n = 0, s = 6
6
\end{verbatim}

\index{guardian}
To guard against arguments leading to infinite recursion, 
we can add integer type constraints to the subroutine 
signature and some code to stop recursion if the first 
argument is negative, for example:

\begin{verbatim}
sub recurse(Int $n, Int $s) {
    say "Args : n = $n, s = $s";
    if $n == 0 {
        say $s;
    } elsif $n < 0 {
        die 'STOP! negative $n, we need to give up';
    } else {
        recurse $n - 1, $n + $s;
    }
}
\end{verbatim}
%

Now, if we call {\tt recurse} with a negative value for 
\verb'$n', we get an error message:

\begin{verbatim}
Args : n = -1, s = 0
STOP! negative $n, we need to give up
  in sub recurse at recurse2.pl6 line 6
  in block <unit> at recurse2.pl6 line 12
\end{verbatim}
%

And if we call it with a non integer value for \verb'$n':

\begin{verbatim}
===SORRY!=== Error while compiling recurse2.pl6
Calling recurse(Rat, Int) will never work with declared
signature (Int $n, Int $s)
at recurse2.pl6:12
------> <BOL><HERE>recurse 1.5, 0;
\end{verbatim}
%

\index{multi!keyword}
\index{multi!subroutine}
Another possibility might be to use a feature of Perl~6 which 
we haven't covered yet, \emph{multi} subroutines, described in 
Section~\ref{multisubs} (p.~\pageref{multisubs}). The idea is 
to declare two versions of the {\tt recurse} subroutine, 
which have the same name but different signatures. 
The compiler will figure out which version of {\tt recurse} 
to call depending on which signature applies to the 
arguments passed to the subroutine:

\begin{verbatim}
multi recurse(Int $n where $n >= 0, $s) {
    say "Args : n = $n, s = $s";
    if ($n == 0) {
        say $s;
    } else {
        recurse $n - 1, $n + $s;
    }
}

multi recurse($n , $s) {
    say "Args : n = $n, s = $s";
    # do something else for such a case, for example:
    # recurse (abs $n.Int), $s; # calling the 1st version of recurse
    # or simply:
    say 'STOP! invalid $n, we need to give up';
}
\end{verbatim}

If the first parameter is a positive integer, it is the 
first version of {\tt recurse} that will be called. 
Otherwise, it will be the second version that will run:

\begin{verbatim}
$ perl6 recurse.pl6
Args : n = 6.1, s = 0
STOP! negative $n, we need to give up

$ perl6 recurse.pl6
Args : n = -1, s = 0
STOP! invalid $n, we need to give up
\end{verbatim}
%

Try to run it with the following code for the second 
definition of {\tt recurse}:

\begin{verbatim}
multi recurse($n , $s) {
    say "Args : n = $n, s = $s";
    recurse (abs $n.Int), $s;
}
\end{verbatim}
%
to see what is happening in that case.

\section{Exercises of Chapter~\ref{fruitchap}: Fruitful functions}

\subsection{Compare, exercise at the end of Section~\ref{compare} (p.~\pageref{compare})}
\label{sol_compare}
\index{compare function}

Here's a subroutine that takes two numbers and compares them, 
and returns 1 if the first one is larger than the second, 0 if 
they are equal, and -1 otherwise (i.e. if the second is larger 
than the first):

\begin{verbatim}
sub compare (Numeric $x, Numeric $y) {
    return 1 if $x > $y;
    return 0 if $x == $y;
    return -1;
}

say compare 4, 7;   # -1
say compare 7, 4;   # 1
say compare 5, 5;   # 0
\end{verbatim}

Note: this exemplifies a three-way {\tt compare} function 
commonly used for sorting in a number of programming 
languages, including older versions of Perl (such as 
Perl~5). In Perl~6, the operators implementing this 
functionality (the three-way comparators \verb'cmp', 
\verb'leg' and \verb'<=>') return special types of 
values: {\tt Order::More}, {\tt Order::Less} and 
{\tt Order::Same}. (See Section~\ref{sorting} 
\ifplastex \else
on sorting in the chapter about arrays and lists 
\fi
for more details.)
\index{compare function}
\index{sorting}

\subsection{Hypotenuse, exercise at the end of Section~\ref{hypotenuse} (p.~\pageref{hypotenuse})}
\label{sol_hypotenuse}
\index{hypotenuse}

The aim is to use an incremental development plan for 
calculating the hypotenuse of a right triangle (using 
the Pythagorean theorem). 

We could start with an outline of the subroutine:

\begin{verbatim}
sub hypotenuse(Numeric $x, Numeric $y) {
    return 0;
}
say hypotenuse 3, 4;
\end{verbatim}
%

This will obviously print 0.

Next, we calculate the hypotenuse and print it within the subroutine:

\begin{verbatim}
sub hypotenuse(Numeric $x, Numeric $y) {
    my $hypotenuse = sqrt ($x ** 2 + $y ** 2);
    say "hypotenuse = $hypotenuse";
    return 0.0;
}
say hypotenuse 3, 4;
\end{verbatim}
%

This prints:

\begin{verbatim}
hypotenuse = 5
0
\end{verbatim}
%

\index{hypotenuse}
The subroutine is calculating correctly the hypotenuse (5), 
but is still returning the 0 dummy value. We can now return 
safely the result (and remove the scaffolding):

\begin{verbatim}
sub hypotenuse(Numeric $x, Numeric $y) {
    my $hypotenuse = sqrt ($x ** 2 + $y ** 2);
    return $hypotenuse;
}
say hypotenuse 3, 4;
\end{verbatim}
%

This prints correctly the value of the hypotenuse.

Finally, we can, if we wish, remove the temporary variable 
to further simplify the subroutine:
\begin{verbatim}
sub hypotenuse(Numeric $x, Numeric $y) {
    return sqrt ($x ** 2 + $y ** 2);
}
say hypotenuse 3, 4;
\end{verbatim}
%
\index{hypotenuse}

\subsection{Chained relational operators(in Section~\ref{isbetween})}
\label{sol_isbetween}
\index{is-between}

We need a subroutine to figure out whether $x \le y \le z$ 
is true or false. We simply need to test it with a chained relational operator and return that:

\begin{verbatim}
sub is-between(Numeric $x, Numeric $y, Numeric $z) {
    return $x <= $y <= $z;
}
say is-between 3, 5, 6; # True
say is-between 3, 8, 7; # False
say is-between 6, 5, 6; # False
say is-between 6, 6, 7; # True
\end{verbatim}
%

Note that the tests provided here are just a limited number of 
examples, given for illustration purposes. A more complete test 
suite might be needed (testing for example negative and non 
integer numbers). We will see later better ways 
of building more robust test suites (see for example 
Section~\ref{test_module} and exercise 
solution in Section~\ref{sol_fact_operator}).


\subsection{The Ackermann function (Exercise~\ref{ackermann})}
\label{sol_ackermann}
\index{Ackermann function}

Write a subroutine to compute the Ackermann function. The Ackermann function, $A(m, n)$, is defined as follows:

\begin{eqnarray*}
A(m, n) = \begin{cases} 
              n+1 & \mbox{if } m = 0 \\ 
        A(m-1, 1) & \mbox{if } m > 0 \mbox{ and } n = 0 \\ 
A(m-1, A(m, n-1)) & \mbox{if } m > 0 \mbox{ and } n > 0.
\end{cases} 
\end{eqnarray*}
%
Here's one way to compute the Ackermann function in Perl:

\begin{verbatim}
sub ack (Int $m, Int $n) {
    return $n + 1 if $m == 0;
    return ack($m - 1, 1) if $n == 0;
    return ack($m - 1, ack($m, $n-1));
}
say ack 3, 4;    # -> 125
\end{verbatim}

We have used parentheses to better show the structure, but 
it works well without them. Even in the last code line with
two calls to the subroutine, the subroutine signature with 
two integer numbers is sufficient for the Perl compiler 
to understand which arguments are associated with which call:

\begin{verbatim}
sub ack (Int $m, Int $n) {
    # say "m n = $m, $n";
    return $n + 1 if $m == 0;
    return ack $m - 1, 1 if $n == 0;
    return ack $m - 1, ack $m, $n-1;
}
\end{verbatim}
%
\index{Ackermann, Wilhelm}

The Ackermann function is defined for non-negative integers. 
As a further exercise, modify the {\tt ack} subroutine to 
prevent negative arguments. We have seen two different ways 
of doing that in Section~\ref{guardian}.


\subsection{Palindromes (Exercise~\ref{palindrome})}
\index{palindrome}
\label{sol_palindrome}

Write a recursive subroutine that checks if a word is a 
palindrome.

\begin{verbatim}
sub first_l(Str $word where $word.chars >= 2){
    return substr $word, 0, 1;
}

sub last_l(Str $word){
    return substr $word, *-1, 1;
}

sub middle_l(Str $word){
    return substr $word, 1, *-1;
}

sub is_palindrome(Str $word) {
    return True if $word.chars <= 1;
    return False if first_l($word) ne last_l($word);
    return is_palindrome(middle_l($word))
}
for ("bob", "otto", "laurent", "redivider", "detartrated") -> $x {
    say "Is $x a palindrome? Answer: ", is_palindrome($x);
}
\end{verbatim}

Result:
\begin{verbatim}
Is bob a palindrome? Answer: True
Is otto a palindrome? Answer: True
Is laurent a palindrome? Answer: False
Is redivider a palindrome? Answer: True
Is detartrated a palindrome? Answer: True
\end{verbatim}

The third parameter (length) of the {\tt substr} built-in 
function is optional. In that case, {\tt substr} will return 
all characters from a given position. So the \verb'first_l'  
subroutine could be simplified as follows:

\begin{verbatim}
sub first_l(Str $word where $word.chars >= 2){
    return substr $word, 0;
}
\end{verbatim} 

And the \verb'last_l' subroutine could benefit from the same 
simplification.

Note: The built-in {\tt flip} function or {\tt .flip} method 
of Perl returns a reversed version of a string and would 
provide a much easier solution:

\begin{verbatim}
sub is_palindrome(Str $word) {
    return $word eq $word.flip;
}
\end{verbatim}
\index{flip function}
\index{palindrome}

\subsection{Powers (Exercise~\ref{power})}
\index{power}
\label{sol_power}

Write a recursive subroutine checking whether a number is 
a power of another number.

\begin{verbatim}
sub is-power-of (Int $a, Int $b) {
    return False unless $a %% $b;
    return True if $a == $b;
    return is-power-of Int($a/$b), $b;
}

say is-power-of 16, 4;
say is-power-of 25, 5;
say is-power-of 125, 5;
say is-power-of 600, 20;
say is-power-of 8000, 20;
\end{verbatim}
%

Example run:
\begin{verbatim}
True
True
True
False
True
\end{verbatim}

Adding an execution trace to visualize the recursive calls:

\begin{verbatim}
sub is-power-of (Int $a, Int $b) {     
    return False unless $a %% $b;      
    return True if $a == $b;           
    say "$a\t$b";                      
    return is-power-of Int($a/$b), $b; 
}                                      
\end{verbatim}

Running {\tt is-power-of} with arguments 1024 and 2, 
with a printed trace of \verb'$a' and \verb'$b':

\begin{verbatim}
1024    2
512     2
256     2
128     2
64      2
32      2
16      2
8       2
4       2
True
\end{verbatim}
%
\index{power}

\subsection{Finding the GCD of two numbers, Exercise~\ref{gcd} (p.~\pageref{gcd})}
\index{gcd}
\index{greatest common divisor (GCD)}
\index{GCD (greatest common divisor)}
\label{sol_gcd}

Write a subroutine that returns the greatest common divisor of two numbers.

\begin{verbatim}
sub gcd(Int $a, Int $b) {
    return $a if $b == 0;
    return $b if $a == 0;
    return gcd($b, $a mod $b);
}

say gcd 125, 25;
say gcd 2048, 256;
say gcd 256, 4096;
say gcd 2048, 1;
say gcd 0, 256;
say gcd 33, 45;
\end{verbatim}

\index{Euclid's algorithm}
\index{algorithm!Euclid's}
Note that there is a simpler method to find the GCD of 
two numbers without using the modulo function. It is known
as Euclid's algorithm and is considered as the oldest known 
algorithm (see \url{https://en.wikipedia.org/wiki/Euclidean_algorithm}). The Euclidean algorithm is based on 
the observation that the GCD of two numbers does not change 
if the larger number is replaced by its difference with 
the smaller number.

This might be implemented in Perl with the following recursive 
subroutine:

\begin{verbatim}
sub gcd(Int $a, Int $b) { 
    return gcd($b, $a - $b) if $a > $b;
    return gcd($a, $b - $a) if $b > $a;
    return $a;
}
\end{verbatim}
%

This code works perfectly well in \emph{almost} all cases, 
at least for all strictly positive input values, 
but try to follow the flow of execution if one of the 
two arguments passed to the subroutine, say {\tt \$b}, 
is zero. In this case, {\tt gcd} enters in an infinite 
recursion. This is often called an \emph{edge case} or a 
\emph{corner case}, i.e. a special input value for which 
an apparently well-working program ceases to function properly.
\index{edge case}
\index{corner case}

We have a similar problem for negative input values.

One solution might be to add a signature constraint (or 
use a type subset):
\index{subset!type}
\index{type subset}
\begin{verbatim}
sub gcd(Int $a where $a > 0, Int $b where $b > 0) {
    ...
}
\end{verbatim}
%

but this is not really satisfactory because the GCD of any 
non-zero integer and 0 is well defined mathematically and 
is equal the first number.

Leaving aside for the moment the case of negative numbers, 
we could rewrite our subroutine as follows:

\begin{verbatim}
sub gcd(Int $a, Int $b) { 
    return $a if $b == 0;
    return $b if $a == 0;
    return gcd($b, $a - $b) if $a > $b;
    return gcd($a, $b - $a) if $b > $a;
    return $a;
}
\end{verbatim}
%

Concerning negative numbers, there is a theorem stating that 
the GCD of $a$ and $b$ is the same as the GCD of $a$ and $-b$:

\begin{center}
{\tt gcd(a,b) = gcd(−a,b) = gcd(a,−b) = gcd(−a,−b)}
\end{center}
%

We can modify further the {\tt gcd} subroutine:

\begin{verbatim}
sub gcd(Int $a is copy, Int $b is copy) { 
    $a = -$a if $a < 0;
    $b = -$b if $b < 0;
    return $a if $b == 0;
    return $b if $a == 0;
    return gcd($b, $a - $b) if $a > $b;
    return gcd($a, $b - $a) if $b > $a;
    return $a;
}
\end{verbatim}
%

This is now working fine, but remember that a recursive 
subroutine may be called many times and, for each call,
the first four code lines in the program above are 
executed, although they are really useful only at the 
first call: once these conditions have been checked during 
the first call to the subroutine, we know that the 
arguments must be and remain valid in the chain of recursive 
calls, so that these checks are useless after the first call.
This is somewhat wasteful and may lead to unnecessary performance problems.

Ideally, it might be better to separate these four lines 
that check the preconditions from the cascade of recursive 
calls. For example, might write two subroutines:

\begin{verbatim}
sub gcd1(Int $c, Int $d) {
    return gcd1($d, $c - $d) if $c > $d;;
    return gcd1($c, $d - $c) if $d > $c;
    return $c;
}

sub gcd(Int $a is copy, Int $b is copy) { 
    $a = -$a if $a < 0;
    $b = -$b if $b < 0;
    return $a if $b == 0;
    return $b if $a == 0;
    return gcd1 $a, $b;
}
\end{verbatim}
%

Now, {\tt gcd} is making all the necessary checks on the 
initial arguments and calls the recursive {\tt gcd1} 
subroutine with arguments that have been sanitized and 
will not lead to infinite recursion. Note that we have 
renamed the parameters within {\tt gcd1} for better 
clarity, but this was not necessary, it would just work 
the same if we had kept {\tt \$a} and {\tt \$b}.

The preceding code works perfectly well.

There may be a last problem, though. Someone being not 
careful enough (or wanting to be too clever) might decide 
to call directly {\tt gcd1}, thus annihilating the benefits 
of the checks made by {\tt gcd}. To prevent that, we can make good 
use of the fact that subroutines have lexical scope in Perl~6 
and can be made local to another subroutine: we can declare 
and define {\tt gcd1} within the body of the {\tt gcd} 
subroutine, so that {\tt gcd1} can be called only from 
within the {\tt gcd} subroutine:
\index{lexical subroutine}

\begin{verbatim}
sub gcd(Int $a is copy, Int $b is copy) { 
    sub gcd1($c, $d) {
        return gcd1($d, $c - $d) if $c > $d;
        return gcd1($c, $d - $c) if $d > $c;
        return $c;
    }
    $a = -$a if $a < 0;
    $b = -$b if $b < 0;
    return $a if $b == 0;
    return $b if $a == 0;
    return gcd1 $a, $b;
}

say gcd 125, 25;       # 25
say gcd 2048, 256;     # 256
say gcd 256, 4096;     # 256
say gcd 2048, 1;       # 1
say gcd 0, 256;        # 256
say gcd 33, 45;        # 3
say gcd -4, 6;         # 2
\end{verbatim}

Chapter~\ref{iteration} will come back to lexical scoping.

\index{gcd}
\index{Euclid's algorithm}
\index{algorithm!Euclid's}

You may be interested to know that there is a {\tt gcd} built-in 
function in Perl~6.


\section{Exercises of Chapter~\ref{iteration} (Iteration)}

\subsection{Exercise~\ref{test_sqrt}: square root (p.~\pageref{test_sqrt})}
\label{sol_test_sqrt}
\index{square root}

We need a subroutine to find the square root of a number by 
computing successively better approximations of the root, using 
Newton's method.
\index{Newton's method}
\index{Newton, Isaac}

For this exercise, we've made the following somewhat 
arbitrary decisions:
\begin{itemize}
\item We have chosen an \emph{epsilon} value of $10^{-11}$ (or {\tt 1e-11});
\item We have used \verb'$a/2' as the initial estimate 
of $\sqrt{\$a}$. 
\end{itemize}

Note that it might make more sense to make this 
initial estimate within the {\tt my-sqrt} subroutine, 
rather than having the caller pass it as an argument. The 
rationale for doing it in the caller is that, in some cases, 
the caller might have information on the range of the input 
value and might therefore be able to provide a better 
initial estimate leading the algorithm to converge toward 
the solution slightly faster.

Here's out implementation of Newton's method for computing 
the square root of a number:
\index{Newton's method}

\begin{verbatim}
sub my-sqrt ($a, $estimate is copy) {
    my $epsilon = 1e-11;
    while True {
        # say "-- Intermediate value: $estimate";
        my $y = ($estimate + $a/$estimate) / 2;
        last if abs($y - $estimate) < $epsilon;
        $estimate = $y;
    }
    return $estimate;
}

sub print-result ($a, $r, $s, $d) {
    printf "%d  %.13f  %.13f  %.6e \n", $a, $r, $s, $d;
}

sub test-square-root {
    say "a  mysqrt(a)\t    sqrt(a)\t     diff";
    for 1..9 -> $a {
        my $init-estimate = $a/2;
        my $result = my-sqrt $a, $init-estimate;
        my $sqrt = sqrt $a;
        my $diff = abs($result - $sqrt);
        print-result($a, $result, $sqrt, $diff);
    }
}
    
test-square-root;
\end{verbatim}

\index{printf function}
The {\tt printf} ("formatted print") function used in 
the {\tt print-result} subroutine is derived from the 
C~programming language. Its first argument is a 
\emph{format string} which describes how each of 
the following arguments should be formatted. Here, the 
format string requests the compiler to output the 
first subsequent argument as a signed integer (the 
\verb'%d' part of the format string), the 
next two arguments as floating-point numbers with
13~digits after the decimal point (the \verb'%.13f' part), and 
the last argument as a floating-point number in scientific 
notation with 6~digits after the decimal point (\verb'%.6e').



\subsection{Exercise~\ref{pi_estimate}: Pi estimate (p.~\pageref{pi_estimate})}
\label{sol_pi_estimate}
\index{Ramanujan, Srinivasa}
\index{Ramanujan, Srinivasa!pi estimate}
\index{pi!estimate}

Pi estimate according to Srinivasa Ramanujan's algorithm


\begin{verbatim}
sub factorial(Int $n) {
    return 1 if $n == 0;
    return $n * factorial $n-1;
}

sub estimate-pi {
    #`{ ======================================
        Algorithm by Srinivasa Ramanujan 
        (see http://en.wikipedia.org/wiki/Pi)
        ======================================
      }
    my $factor = 2 * 2.sqrt / 9801;
    my $k = 0;
    my $sum = 0;
    while True {
        my $num = factorial(4*$k) * (1103 + 26390*$k);
        my $den = factorial($k)**4 * 396**(4*$k);
        my $term += $factor * $num / $den;
        # say "Intermediate term = $term";
        last if abs($term) < 1e-15;
        $sum += $term;
        $k++;
    }
    return 1 / $sum;
}

say estimate-pi;
# say pi - estimate-pi;
\end{verbatim}

This prints: 3.14159265358979.

Notice how we have used here a multiline comment to 
give some additional information about the subroutine.

Uncommenting the intermediate print statement shows the 
steps of the progress toward to solution:

\begin{verbatim}
Intermediate term = 0.31830987844047
Intermediate term = 7.74332048352151e-009
Intermediate term = 6.47985705171744e-017
-4.44089209850063e-016
\end{verbatim}

\section{Exercises of Chapter~\ref{strings} (strings)}

\subsection{Exercise in Section~\ref{stringtraversal}: String traversal (p.~\pageref{stringtraversal})}
\label{sol_stringtraversal}

The backward traversal of a word with a {\tt while} loop may
be written as follows:

\begin{verbatim}
my $fruit = "banana";
my $index = $fruit.chars;
while $index > 0 { 
    $index--;
    my $letter = substr $fruit, $index, 1; 
    say $letter; 
}
\end{verbatim}
%

The {\tt chars} method returns the length of the string. The 
{\tt substr} function will find letters under \verb'$index' 
between 0 and \verb'$length' - 1. It is therefore practical 
to decrement the \verb'$index' variable before using it the
{\tt substr} function. 

The \emph{while} loop of the preceding code example can be 
written more concisely:

\begin{verbatim}
my $fruit = "banana";
my $index = $fruit.chars;
while $index > 0 { 
    say substr $fruit, --$index, 1; 
}
\end{verbatim}
%

Here, we print directly the value returned by {\tt substr}, 
without using a temporary variable, and we decrement the 
\verb'$index' variable within the expression using {\tt substr}.
We need to use the prefix form of the decrement operator 
because we need \verb'$index' to be decremented before it is 
used by {\tt substr}.
\index{prefix decrement operator}

The loop would be made even more concise using a {\tt while} with 
a statement modifier (or the postfix syntax of {\tt while}):
\index{statement modifier}
\index{postfix!syntax}

\begin{verbatim}
my $fruit = "banana";
my $index = $fruit.chars;
say substr $fruit, --$index, 1 while $index;
\end{verbatim}
%

This is the same idea, using the {\tt flip} function to 
reverse the string:
\index{flip function}

\begin{verbatim}
my $fruit = flip "banana";
my $index = 0;
say substr $rev_fruit, $index++, 1 while $index < $rev_fruit.chars;
\end{verbatim}
%

The aim of this exercise was to train you to use loops to 
traverse the string. Combining the {\tt flip} and {\tt comb} 
functions or methods would of course make our solution much 
simpler (and probably faster):

\begin{verbatim}
.say for "banana".flip.comb;
\end{verbatim}


\subsection{Exercise in Section~\ref{stringtraversal}: The ducklings (p.~\pageref{stringtraversal})}
\label{sol_ducklings}

The first idea that may come to mind for this exercise is to build 
a modified list of prefixes this way:

\begin{verbatim}
for 'J' .. 'N', 'Ou', 'P', 'Qu' -> $letter { #...}
\end{verbatim}
%

But that does not work properly because this creates a list 
of four elements in which the first element is itself a 
sublist ``J'' to ``N'':

\begin{verbatim}
> say ('J'..'N', 'Ou', 'P', 'Qu').perl;
("J".."N", "Ou", "P", "Qu")
\end{verbatim}
%

We will come back to this later in the book, but let us just say 
that we need to \emph{flatten} this combination of lists into 
one single iterable list, which can be done with the 
{\tt flat} method or function or the ``|'' operator:

\begin{verbatim}
for ('J' .. 'N', 'Ou', 'P', 'Qu').flat     -> $letter {#...}
# or: for flat 'J' .. 'N', 'Ou', 'P', 'Qu' -> $letter {...}
# or: for |('J' .. 'N'), 'Ou', 'P', 'Qu'   -> $letter {...}
# Note: parentheses needed in the last example above with | 
# to overcome precedence problem
\end{verbatim}
%

With this small difficulty removed, the solution is now easy:

\begin{verbatim}
my $suffix = 'ack';
for ('J' .. 'N', 'Ou', 'P', 'Qu').flat -> $letter {
    say $letter ~ $suffix;
}
\end{verbatim}
%

Here again, we could make the code slightly more concise with 
the postfix syntax of {\tt for} and the \verb'$_' topical 
variable:

\begin{verbatim}
my $suffix = 'ack';
say "$_$suffix" for flat 'J' .. 'N', 'Ou', 'P', 'Qu';
\end{verbatim}
%

Here, we have introduced another simple and common way of 
concatenating two strings: simply inserting the two variables one after the other within double quotes.


\subsection{Exercise in Section~\ref{stringtraversal}: Counting the letters of a string (p.~\pageref{stringtraversal})}
\label{sol_count_letters}

This subroutine counts the number of occurrences of 
a specific letter within a word (or any string):

\begin{verbatim}
sub count (Str $word, Str $letter) {
    my $count = 0;
    for $word.comb -> $letter {
        $count++ if $letter eq 'a';
    }
    return $count;   
}
say count "banana", "a";   # -> 3
\end{verbatim}
%

The solution to Exercise~\ref{count_a} 
(p.~\pageref{sol_count_a}) below uses the 
{\tt index} and {\tt substr} functions to perform 
the same count .


\subsection{Section~\ref{regex_loop}: Simulating a regex with a loop (p.~\pageref{regex_loop})}
\label{sol_regex_loop}

The aim is to find in a string any letter that is 
immediately preceded by the letter ``l'' and followed by 
the letter ``w''.

If you try to do the specified search with the techniques 
we've seen so far, you'll find out that there are a number 
of edge cases making it quite complicated.

This is a possible solution:

\begin{verbatim}
sub traverse (Str $word, Str $start_letter, Str $end_letter) {
    my $found_start = False;
    my $capture_next = False;
    my $target_letter;
    for 0..$word.chars - 1 -> $idx {
        my $letter = substr $word, $idx, 1;
        next unless $letter eq $start_letter or $found_start;
        if ($capture_next) {
            $target_letter = $letter;
            $capture_next = False;
            next;
        }
        if $letter eq $start_letter and not $found_start {
            $found_start = True;
            $capture_next = True;
            next;
        }
        # if we get there, we have found a candidate target letter
        if $letter eq $end_letter {
            return $target_letter
        } else {
            # wrong match, let's start again, we need to backup
            if $target_letter eq $start_letter {
                 $target_letter = $letter;
                 $capture_next = False;
            } elsif $letter eq $start_letter {
                 $capture_next = True;
            } else {
               $capture_next = False;
               $found_start = False;
            } 
        } 
    }
    return;  # not found!
}          

for <s b   l w   l o   s m   y l   a z> -> $st, $end {
    say "$st $end: ", traverse "yellow submarine", $st, $end;  
}
\end{verbatim}

As you can see, this is quite complicated because of 
the various edge cases that need to be handled. Compare this with 
the one-line regex that does the same:

\begin{verbatim}
say ~$0 if "yellow submarine" ~~ /l(.)w/;
\end{verbatim}

To tell the truth, we haven't chosen the simplest way of 
doing it.

It is much easier to loop on every letter of the string 
except the first one and the last one and, for each such 
letter, to check what the previous letter and the next 
are. Then you simply need to return the current letter if 
the previous and the next match the conditions:

\begin{verbatim}
sub traverse (Str $word, Str $start_letter, Str $end_letter) {
    my $found_start = False;
    my $capture_next = False;
    my $target_letter;
    for 1..$word.chars - 2 -> $idx {
        if $start_letter eq substr $word, $idx - 1, 1
          and $end_letter eq substr $word, $idx + 1, 1 {
            return substr $word, $idx, 1;
        }
     }
    return;  # not found!
}          

for <s b   l w   l o   s m   y l   a z> -> $st, $end {
    say "$st $end: ", traverse "yellow submarine", $st, $end;  
}
\end{verbatim}

In the test cases at the end, we use a {\tt for} loop in a 
pointy block construct in which we use each time through the 
loop \emph{two} of the items in the list. The numbers of 
spaces between the items of the list are technically useless 
and irrelevant to the 
way the syntactic construct works, they are just a formatting 
help for the reader to better see how the letters will 
be grouped in the process.
\index{for loop}
\index{pointy block}
\index{pointy block!using several items}

This displays:
\begin{verbatim}
s b: u
l w: o
l o: l
s m: Nil
y l: e
a z: Nil
\end{verbatim}

This is much simpler than the previous attempt, but it would 
still be quite difficult to change something, for example 
to add a new condition: the structure of the code would 
probably need to be reworked quite a bit.

Even compared with this simpler solution, the regex solution 
really shines orders of magnitude brighter.


\subsection{Exercises in Subsection~\ref{regex_exercises}: Regex exercises (p.~\pageref{regex_exercises})}
\label{sol_regex_exercises}
\index{regex}

As is often the case in Perl, and even more so with 
regexes, there is more than one way to do it (TIMTOWTDI). 
Most of the exercises suggested here have more than one 
solution (and sometimes many).
\index{TIMTOWTDI}

With regexes, you also have to think carefully about the 
input data to figure out what should be matched and what 
should be rejected.

\subsubsection{Ten digits in a row}

Here's a way to find ten consecutive digits in a string:

\begin{verbatim}
my $string = "567867 8778689 6765432 0123456789 897898";
say ~$0 if $string ~~ /(\d ** 10)/; # -> 0123456789
\end{verbatim}

We are simply using the \verb'\d' (digit) character class 
together with a quantifier specifying 10 times this class.

Note that we have used here capturing parentheses in order 
to populate the matched number into \verb'$0'. We could also 
omit parentheses and retrieve the number from the Match object:

\begin{verbatim}
my $string = "567867 8778689 6765432 0123456789 897898";
say ~$/ if $string ~~ /\d ** 10/; # -> 0123456789
\end{verbatim}
%

Also, the above solutions would match any ten-digit sequence 
within a longer sequence of digits, which may or may not 
be what you need. For example:

\begin{verbatim}
my $string = "567867 87786896765432 0123456789 897898";
say ~$0 if $string ~~ /(\d ** 10)/; # -> 8778689676
\end{verbatim}

If you want to match more precisely a sequence of ten 
digits (not more than 10), you need to specify what you 
want to have ``around'' the match. For example, to 
match the sole ten-digit sequence above, you might use 
the non-digit character class:

\begin{verbatim}
my $string = "567867 87786896765432 0123456789 897898";
say ~$0 if $string ~~ /\D (\d ** 10) \D/; # -> 0123456789
\end{verbatim}

But that would not match a ten-digit sequence at the start 
or the end of the string:

\begin{verbatim}
my $string = "5678670001 87786896765432 0123456789 897898";
say ~$0 if $string ~~ /\D (\d ** 10) \D/; # -> 0123456789
\end{verbatim}

A better solution might be to use word boundary anchors:
\index{anchor!word boundary}
\index{anchor!left word boundary}
\index{anchor!right word boundary}
\index{anchor!start of word}
\index{anchor!end of word}

\begin{verbatim}
my $string = "5678670001 87786896765432 0123456789 897898";
say ~$0 if $string ~~ /<< (\d ** 10) >>/; # -> 5678670001
\end{verbatim}

As it can be seen, quite a bit of reflection may 
sometimes be needed to ensure that we match exactly 
what we want. 

\subsubsection{An octal number}

Here's a possible solution for finding an octal number (i.e. 
a number composed only of digits between 0 and 7) in 
a string:

\begin{verbatim}
my $string = "567867 8778689 6765432 0123456789 897898";
say ~$0 if $string ~~ /\D (<[0..7]>+) \D/; # -> 6765432
\end{verbatim}

The character class is \verb'<[0..7]>' for digits between 
0 and 7. The \verb'+' quantifiers means: as many as possible 
of this character class. And the \verb'\D' (non digit) are 
there to prevent the regex from matching part of a larger 
number with non-octal digits (for example from matching 567 
in the first number). Depending on the exact requirement, 
using word boundary anchors as in the previous exercise's 
solution might be better.

\subsubsection{First word at the start of the string}

To find the first word in a string, we can just search the 
first sequence of word characters (characters belonging to 
the \verb'\w' character class) in the string:

\begin{verbatim}
my $string = "The greatest thing you'll ever learn " ~
             "is just to love and be loved in return" ~
             "(Nature Boy, Nat King Cole)";
say ~$0 if $string ~~ /(\w +)/;  # -> The
\end{verbatim}


\subsubsection{First word starting with an ``a''}

Here's a way to find the first word starting with the letter 
``a'' in a sentence:

\begin{verbatim}
my $string = "Four scores and seven years ago our fathers ...";
say ~$0 if $string ~~ /\W (a \w+)/;  # -> and
\end{verbatim}

\subsubsection{First word starting with a lower-case vowel}

To make sure that the match does not start with a vowel in 
the middle of a word, we might start the pattern with a 
\verb'\W' (non-word character) or, better, with a \verb'<<' 
left word boundary:

\begin{verbatim}
my $string = "Democracy is the worst form of government, " ~
             "except for all the others. (Churchill)";
say ~$0 if $string ~~ /<< (<[aeiouy]> \w*)/;  # -> is
\end{verbatim}

Here we use a \verb'*' (rather than \verb'+') quantifier 
because a word containing only one vowel is eligible as 
a word starting with a vowel.

\subsubsection{A mobile number}

For a ten-digit number starting with ``06'' or ``07'', the 
easiest solution is probably to use a \verb'<[67]>' character 
class:

\begin{verbatim}
my $string = "567867 8778689 0123456789 0723456789 3644";
say ~$0 if $string ~~ /(0<[67]>\d ** 8)/;  # -> 0723456789
\end{verbatim}

\subsubsection{First word starting with a vowel (lower- or upper-case)}

We can simply ignore case for the whole word:
\begin{verbatim}
my $string = " Ask not what your country can do for you — " ~
             " ask what you can do for your country. (JFK)";
say ~$0 if $string ~~ /:i << (<[aeiouy]> \w*)/;  # -> Ask
\end{verbatim}

\subsubsection{Repeated letters}

We can capture any letter and check whether the next one is 
the same as the capture:

\begin{verbatim}
say ~$0 if 'appeal' ~~ /((\w)$0);  # -> pp
\end{verbatim}

For capturing the second group of repeated letters:

\begin{verbatim}
say ~$1 if 'coffee' ~~ /(\w)$0.*((\w)$0)/;  # -> ee
\end{verbatim}

And for the third group:

\begin{verbatim}
say ~$2 if 'Mississippi' ~~ /(\w)$0.*(\w)$0.*((\w)$0)/; # -> pp
\end{verbatim}


\subsection{Exercise in Section~\ref{isreverse}: {\tt is-reverse} subroutine (p.~\pageref{isreverse})}
\label{sol_isreverse}
\index{is-reverse}

The second bug in the {\tt is-reverse} subroutine is located 
on this line:

\begin{verbatim}
    while $j > 0 {
\end{verbatim}
%

The {\tt \$j} index should be allowed to loop down until 0 
(included) if we want to compare the first letter of 
{\tt \$word2} with the last letter of {\tt \$word1}.

The corrected version of the  {\tt is-reverse} subroutine 
might be:

\begin{verbatim}
sub is-reverse(Str $word1, Str $word2) {
    return False if $word1.chars != $word2.chars;
    
    my $i = 0;
    my $j = $word2.chars - 1;

    while $j >= 0 {
        return False if substr($word1, $i, 1) ne substr($word1, $j, 1);
        $i++; $j--;
    }
    return True;
}
\end{verbatim}
%

\subsection{Exercise~\ref{count_a}: Counting letters (p.~\pageref{count_a})}
\label{sol_count_a}
\index{count letters}

Counting the number of letters ``a'' in a word with the {\tt index} 
function implies looking for the first ``a'' from the 
beginning of the string, then looking for the next one 
from the position immediately after, and so on until no 
more ``a'' is to be found.

Here, we make an infinite loop from which we break out with 
the {\tt last} statement when {\tt index} no longer finds 
an ``a''. The {\tt \$count} counter is incremented each time 
an ``a'' is found, and the {\tt \$idx} keeps track of the 
current position within the string.

\begin{verbatim}
sub count_a {
    my $word = "banana";
    my $count = 0;
    my $idx = 0;
    while True {
       $idx = index $word, 'a', $idx;
       last unless $idx.defined;
       $idx++;
       $count++;
    }
    return $count;
}
say count_a();  # -> 3
\end{verbatim}

Adapting it for any string and any letter is just a matter 
of passing the right arguments to the subroutine and using 
within the subroutine its parameters instead of hard-coded 
values.

\begin{verbatim}
sub count_index (Str $word, Str $letter) {
    my $count = 0;
    my $idx = 0;
    while True {
       $idx = index $word, $letter, $idx;
       last unless $idx.defined;
       $idx++;
       $count++;
    }
    return $count;
}
say count_index "When in the Course of human events...", "n"; # 5
\end{verbatim}

Counting a given letter in a given word with the {\tt substr} 
function is straight forward: we just need to loop over 
each letter of the word and increment a counter when needed.

\begin{verbatim}
sub count_substr (Str $word, Str $letter) {
    my $count = 0;
    for 0..$word.chars - 1 {
        $count++ if $letter eq substr $word, $_, 1;
    }
    return $count;
}
say count_substr "I have a dream that one day...", "a"; # -> 4
\end{verbatim}

\subsection{Exercise~\ref{islower}: Lower-case letters (p.~\pageref{islower})}
\label{sol_islower}
\index{count letters}

Only \verb'any_lowercase5' and \verb'any_lowercase7' are correcty 
checking whether the input string contains at least one lower 
case letter.

If you did not determine that yourself, really try to find by 
yourself the mistakes in the others before reading on, you should be 
able to find the errors in the other subroutines (except 
perhaps \verb'any_lowercase4' which is admittedly a bit tricky).

The \verb'any_lowercase5' and \verb'any_lowercase7' subroutines 
perform the search as follows:

\begin{itemize} 
\item \verb'any_lowercase5' sets {\tt \$flag} 
to {\tt False} before the loop, changes it to {\tt True} if 
any character in the string is {\tt True}, and returns 
{\tt \$flag} after the completion of the loop;

\item  \verb'any_lowercase7' is also correct (and probably slightly 
better than \verb'any_lowercase5'). It returns {\tt True} if 
any character is lower-case and return {\tt False} only if 
it gets a chance to go to the end of the loop.
\end{itemize}

The other subroutines have the following mistakes (some have 
arguably several mistakes, we're going to list at least one of 
them):
\begin{itemize} 

\item \verb'any_lowercase1' is only checking the first 
character of its argument and exiting the loop thereafter;

\item \verb'any_lowercase2' is calling the {\tt is-lower} 
subroutine on the string {\tt "char"}, not on the 
{\tt \$char} variable (it also has the same defect as 
\verb'any_lowercase1');

\item \verb'any_lowercase3' is returning {\tt True} or 
{\tt False} depending on only the last character of the 
input string;

\item \verb'any_lowercase4' suffers of a somewhat nasty 
operator precedence problem: the assignment \verb'$flag = $flag ...' 
is executed before the {\tt or} relational operator is executed, 
so that the latter part has no effect. 
Changing the faulty line to:
\index{precedence}
\index{operator!precedence}
\begin{verbatim}
$flag = $flag || is-lower $char;   # higher priority operator
# or
$flag = ($flag or is-lower $char); # parens to override  precedence
\end{verbatim}
would solve the problem;

\item \verb'any_lowercase6' is almost correct in terms of its 
algorithm, but returns the strings {\tt "True"} or 
{\tt "False"} instead of the boolean values {\tt True} or 
{\tt False}.

\item  \verb'any_lowercase8' returns {\tt False} if any 
character is not lower-case;

\item  \verb'any_lowercase9' also returns {\tt False} if any 
character is not lower-case.

\end{itemize}

The following is an example of the loop you could write to test 
each subroutine, each with three input strings:

\begin{verbatim}
for <FOO bar Baz> -> $str {
    say "1. $str: ", any_lowercase1 $str;
    say "2. $str: ", any_lowercase2 $str;
    say "3. $str: ", any_lowercase3 $str;
    say "4. $str: ", any_lowercase4 $str;
    say "5. $str: ", any_lowercase5 $str;
    say "6. $str: ", any_lowercase6 $str;
    say "7. $str: ", any_lowercase7 $str;
    say "8. $str: ", any_lowercase8 $str;
    say "9. $str: ", any_lowercase9 $str;
}
\end{verbatim}

It would be possible to replace the nine print statements 
with a simple loop, but this requires using features that we 
haven't studied yet.

You'll see in other chapters ways to better organize test cases,  
for example in Section~\ref{test_module} (p.~\pageref{test_module}).

\subsection{Exercise~\ref{rotate}: Caesar's cipher (p.~\pageref{rotate})}
\label{sol_rotate}
\index{letter rotation}
\index{rotation, letter}
\index{Caesar cipher}
\index{rot13}

Implementing a letter rotation cypher:

\begin{verbatim}
sub rotate-one-letter (Str $letter, Int $shift) {
    my $upper-end = 'Z'.ord;        # last upper-case letter
    my $lower-end  = 'z'.ord;       # last lower-case letter

    my $rotated-ord = $letter.ord + $shift;
    if $letter ~~ /<[a..z]>/ {                # lower case
        $rotated-ord -= 26 if $rotated-ord > $lower-end;
    } elsif $letter ~~ /<[A..Z]>/ {           # upper case
        $rotated-ord -= 26 if $rotated-ord > $upper-end;
    } else {
        return $letter;
    }
    return $rotated-ord.chr;
}

sub rotate-one-word (Str $word, Int $shift is copy) {
    $shift = $shift % 26;
    $shift = 26 + $shift if $shift < 0;
    my $rotated-word = "";
    for 0..$word.chars - 1 {
        $rotated-word ~=  rotate-one-letter substr($word, $_, 1), $shift;
    }
    return $rotated-word;
}

sub rot13 (Str $word) {
    return rotate-one-word $word, 13;
} 

say rotate-one-word "ABDCabcd", 25;
say rotate-one-word "cheer", 7;
say rotate-one-word "melon", -10;

say rot13("Fbzr cebsnavgl");
\end{verbatim}

If you are interested in decoding only ROT13, the {\tt tr} 
transliteration operator can give you much shorter code. 
For example {\tt tr/a..m/n..z/} will transliterate all 
letters in the {\tt a..m} range into their respective 
equivalents in the {\tt n..z} range.
\index{tr operator}
\index{operator!tr}
\index{one-liner mode}

We can code a ROT13 in a simple Perl one-liner (see Section~\ref{one-liner mode}):
\index{one-liner mode}
\label{rot13_oneliner}

\begin{verbatim}
$ perl6 -e 'my $w = "foobar"; $w ~~ tr/a..mn..z/n..za..m/; say $w;'
sbbone

$ perl6 -e 'my $w = "sbbone"; $w ~~ tr/a..mn..z/n..za..m/; say $w;"
foobar
\end{verbatim}

It is quite easy to add the ranges for capital letters. You might want to do it as a further exercise.


\section{Exercises of Chapter~\ref{wordplay} (Word play)}

\subsection{Exercise~\ref{cartalk}: Consecutive double letters (p.~\pageref{cartalk})}
\label{sol_cartalk}

With the looping techniques used in Chapter~\ref{wordplay}, 
we could write this:

\begin{verbatim}
sub is_triple_double (Str $word) {
    # Tests if a word contains three consecutive double letters.
    my $i = 0;
    my $count = 0;
    while $i < $word.chars - 1 {
        if substr($word, $i, 1) eq substr($word, $i + 1, 1) {
            $count++;
            return True if $count == 3;
            $i += 2;
        } else {
            $count = 0;
            $i++;
        }
    }
    return False;
}

for 'words.txt'.IO.lines -> $word {
    say $word if is_triple_double $word;
}
\end{verbatim}
%

This is, however, a typical case where regexes might prove 
more efficient than looping  (in terms of coding efficiency, 
i.e., as an approximation, how many code lines are needed 
for performing a given task).

We have seen in Chapter~\ref{strings} that regex captures are 
populating the \verb'$0', \verb'$1', \verb'$2', etc. special 
variables. A regex pattern matching any repeated letter might 
therefore be \verb'/(.) $0/', where the character found 
by \verb'$0' is the same as the character found by the dot. 

Similarly, a regex pattern matching three pairs of repeated 
letters in a row might be:
\begin{verbatim}
say ~$/ if "abbccdde" ~~ /(.)$0 (.)$1 (.)$2/; # -> bbccdd 
\end{verbatim}

With this, the program to find the words with three double 
letters in the {\tt words.txt} file takes just three code 
lines:

\begin{verbatim}
for 'words.txt'.IO.lines -> $word {
    say $word if $word ~~ /(.) $0 (.) $1 (.) $2/;
}
\end{verbatim}
%

Both programs find four words which are variations on 
``bookkeeper'' and ``bookkeeping''.

The regex version is so simple that you can code it directly 
at the operating system command line prompt as a one-liner (see Section~\ref{one-liner mode}):
\index{one-liner mode}

\begin{verbatim}
$ perl6 -ne '.say if /(.) $0 (.) $1 (.) $2/' words.txt
bookkeeper
bookkeepers
bookkeeping
bookkeepings
\end{verbatim}


\subsection{Exercise~\ref{cartalk2}: Palindromes in odometers (p.~\pageref{cartalk2})}
\label{sol_cartalk2}

The following is a possible program for solving the 
palindromic odometer puzzle:
\index{palindrome}
\index{odometer}

\begin{verbatim}
sub is-palindrome ($number, $start, $len) {
    # checks if the relevant substring is a palindrome
    my $substring = substr $number, $start, $len;
    return $substring eq flip $substring;
}
    
sub check ($num) {
    # Checks whether the integer num has the properties described
    return (is-palindrome($num, 2, 4) and
        is-palindrome($num + 1, 1, 5) and
        is-palindrome($num + 2, 1, 4) and
        is-palindrome($num + 3, 0, 6));
}

say 'The following are the possible odometer readings:';
for 1e5..1e6 - 4 -> $number {
    say $number if check $number;
}
\end{verbatim}

Another way to do it would be to use regexes to find out whether 
we have palindromes:

\begin{verbatim}
sub check ($num) {
    # Checks whether the integer num has the properties described
        $num ~~ /^..(.)(.)$1$0/ and 
        $num + 1 ~~ /^.(.)(.).$1$0/ and 
        $num + 2 ~~ /^.(.)(.)$1$0/ and 
        $num + 3 ~~ /^(.)(.)(.)$2$1$0/;
}

say 'The following are the possible odometer readings:';
for 1e5..1e6 - 4 -> $number {
    say $number if check $number;
}
\end{verbatim}

this code is shorter, but is also slower: it takes almost twice 
as long to execute on my computer. So there is a trade-off here: 
The first, faster, way is probably better if you need to run 
your program many times or often, but you might prefer the second 
version if this is just a one-off computation. It's up to you to 
decide.


\subsection{Exercise~\ref{cartalk3}: Palindromes in ages (p.~\pageref{cartalk3})}
\label{sol_cartalk3}

The following program iterates over possible age differences 
between 15 and 75 and, for each age, calculates all 
palindromic possibilities.

\begin{verbatim}
say 'diff #instances';
check_diffs();
say 'daughter  mother';
num_instances(18, True);

sub are_reversed(Int $i, Int $j) {
    # $j (mother's age) will always be 2 digits
    return $j eq flip sprintf '%02d', $i; # format $i on 2 digits
}

sub num_instances (Int $diff, Bool $flag) {
    # computes and counts all possibilities for one age difference
    my $daughter = 0;
    my $count = 0;
    while True {
        my $mother = $daughter + $diff;
        if are_reversed($daughter, $mother) or are_reversed($daughter, $mother+1) {
            $count++;
            printf "%02d\t%d\n", $daughter, $mother if $flag;
        }
        last if $mother > 99;
        $daughter++;
    }
    return $count;
}

sub check_diffs () {
    # enumerates all possible age differences
    for 15..75 -> $diff {
        my $nb_cases = num_instances $diff, False;
        say "$diff   $nb_cases" if $nb_cases > 0;
    }
}
\end{verbatim}

The {\tt while True} statement creates an infinite loop. The 
loop is stopped, however, by the {\tt last} control flow 
statement when the mother's age exceeds 99. We will see 
in section~\ref{C-style loop} a more idiomatic way to built 
an infinite loop, but this is sufficient for now.
\index{infinite loop}

The {\tt sprintf} function used here transforms any number 
below ten into a two-digit number string with a leading 0. 
Its syntax is similar to that of the {\tt printf} function 
seen earlier. The difference is that if only creates a new 
string, but does not print it.
\index{sprintf function}

Using the \verb'.fmt' method instead of the {\tt sprintf} 
function, as well as the method syntax for {\tt flip}, in 
may render the \verb'are_reversed' subroutine somewhat  
nicer:
\index{fmt method}

\begin{verbatim}
sub are_reversed(Int $i, Int $j) {
    return $j eq $i.fmt('%02d').flip; # format $i on 2 digits
}
\end{verbatim}


\section{Exercises of Chapter~\ref{arrays} (Arrays and lists)}

\subsection{Exercise of Section~\ref{stacks_queues}: Implementing a queue (p.~\pageref{exercise_queue})}
\label{sol_exercise_queue}

\index{queue}
This a somewhat simplistic implementation of a queue using 
an array and the {\tt unshift} and {\ pop} functions:
\index{enqueue}
\index{dequeue}

\begin{verbatim}
sub enqueue (@queue, $new_item) {
    unshift @queue, $new_item;
}
sub dequeue (@queue) {
    my $item = pop @queue;
    return $item;
}
my @line = 1, 2, 3, 4, 5;
enqueue @line, 6;
say @line;
say dequeue @line for 1..3;
\end{verbatim}

\subsubsection{Improving the queue subroutine signatures}

Let us try to make our queue a bit more robust.

First, we want to add some signatures to our subroutines. We might be tempted to write something like:

\begin{verbatim}
sub enqueue (Array @queue, $new_item) {
    unshift @queue, $new_item;
}
\end{verbatim}

But that does not work, because that would essentially tell 
Perl that the \verb'@queue' parameter is an array of arrays. 
What we need here is the following signature syntax:

\begin{verbatim}
sub enqueue (@queue where Array, $new_item) {
    unshift @queue, $new_item;
}
sub dequeue (@queue where Array) {
    my $item = pop @queue;
    return $item;
}
\end{verbatim}

We probably don't want here any type signature for the \verb'$new_item' parameter of {\tt enqueue}, because we want 
our queue to be able to operate on any data type in order to 
make it as generic as possible. But, just as we said it 
about stacks (Section~\ref{stacks_queues}), we might want to 
be able to add several items to the data structure in one go. 

\subsubsection{Slurpy (or variadic) parameters}

\label{slurpy_params}
\index{slurpy parameters}
\index{variadic parameters}
\index{signature}
There are several ways to insert several elements to the 
queue, but the simplest is probably to use a signature with 
a \emph{slurpy} parameter (or \emph{variadic} parameter): 
an array or hash parameter is marked as slurpy by a leading 
asterisk, which means it can bind to an arbitrary amount 
of arguments (zero or more). These are called "slurpy" because 
they slurp up any remaining arguments to a function, like 
someone slurping up noodles. This also means that a positional 
slurpy parameter can only be the last one in the signature.

\begin{verbatim}
sub enqueue (@queue where Array, *@new_items) {
    unshift @queue, $_ for @new_items;
    # or: unshift @queue, |@new_items;
}
sub dequeue (@queue where Array) {
    my $item = pop @queue;
    return $item;
}
my @line = 4, 5, 6, 7, 8;
enqueue @line, 3, 2, 1;
say @line;
say dequeue @line for 1..3;
\end{verbatim}

This will display:

\begin{verbatim}
[1 2 3 4 5 6 7 8]
8
7
6
\end{verbatim}

See also Section~\ref{slurpy_parameters} for more details 
on slurpy parameters.

Note that, for an \verb'enqueue' subroutine, we can't simply 
write:
\begin{verbatim}
sub enqueue (@queue where Array, *@new_items) {
    unshift @queue, @new_items;
}
\end{verbatim}
%
because, when given an array as a second argument, 
{\tt unshift} inserts the new items as a sub-list. 
Using the for loop or the ``|'' flattening operator 
solves this slight difficulty. 

Another possibility is to use the {\tt prepend} built-in 
function instead of {\tt unshift}, since it does add 
the flattened elements of the array at the beginning 
of the queue:

\begin{verbatim}
sub enqueue (@queue where Array, *@new_items) {
    prepend @queue, @new_items;
}
\end{verbatim}

\subsubsection{A queue using {\tt shift} and {\tt append}}

\index{append function}
\index{shift function}
\index{unshift function}
\index{push function}
\index{list flattening}
The order in which the arguments are passed is a bit 
counter-intuitive. Also, we might prefer not having to use 
a loop to add the new elements. It is slightly easier to use the {\tt push} 
and {\tt shift} combination, and to replace {\tt push} by 
{\tt append}, which does more or less the same thing as 
{\tt push} but flattens the list just as {\tt prepend} 
seen earlier:

\begin{verbatim}
sub enqueue (@queue where Array, *@new_items) {
    append @queue, @new_items;
}
sub dequeue (@queue where Array) {
    my $item = shift @queue;
    return $item;
}
my @line = 1, 2, 3, 4;
enqueue @line, 6, 7, 8;
say @line;
say dequeue @line for 1..3;
\end{verbatim}

This will display:

\begin{verbatim}
[1 2 3 4 6 7 8]
1
2
3
\end{verbatim}

\subsubsection{Exceptions}

\index{exception}
Finally, one additional weakness needs to be fixed: what 
happens if the queue is empty when we try to {\tt dequeue} 
an item? Raising an exception or aborting the program 
might be what's needed. We might also decide to return an 
undefined value and let the caller deal with it:
\index{queue}

\begin{verbatim}
sub enqueue (@queue where Array, *@new_items) {
    append @queue, @new_items;
}
sub dequeue (@queue where Array) {
    return unless @queue;
    my $item = shift @queue;
    return $item;
}
my @line;
enqueue @line, 1, 2, 3;
say @line;
for 1..4 -> $count {
    my $item = dequeue @line;
    if defined $item {
        say $item;
    } else {
        say "ERROR: The queue is empty !";
    }
}
\end{verbatim}

This produces the following output:

\begin{verbatim}
[1 2 3]
1
2
3
ERROR: The queue is empty !
\end{verbatim}

The {\tt dequeue} subroutine could be made simpler using 
the ternary conditional operator (see 
Section~\ref{ternary operator}) and returning the {\tt Nil} 
value is the queue is empty:
\index{Nil}

\begin{verbatim}
sub dequeue (@queue where Array) {
    @queue ?? @queue.shift !! Nil
}
\end{verbatim}


\index{stack}
\index{queue}
As a further exercise, you might want to apply to the 
example code for stacks (seen in Section~\ref{stack_code} 
on p.~\pageref{stack_code}) the changes we have made 
above to the management of queues.

\subsubsection{Encapsulating the data}
\label{functional_queue}

Another problem with our implementation of queues is that 
the \verb'@file' queue is fully accessible to the developer 
who might be tempted to peek directly into the array or, 
worse, to modify it, without using the {\tt enqueue} and 
{\tt dequeue} subroutines designed to keep the queue 
consistent. 

\index{encapsulation}
\index{data hiding}
We might want to prevent that and make it impossible for 
the user to tamper with the queue or otherwise access to it by 
any other mean than the adequate subroutines. Hiding the 
information about the implementation or otherwise making 
it inaccessible by other means than those that have been 
designed for that purpose is often called data 
\emph{encapsulation}. One common way to achieve data 
encapsulation is through object-oriented programming, which 
we cover in Chapter~\ref{objects}. 

\index{first-class object}
\index{object, first-class}
We can, however, obtain 
a similar result by combining variable scoping and some 
material briefly covered in Section~\ref{first_class} about 
subroutines as first-class objects.

Consider the following implementation of a queue:
\begin{verbatim}
sub create-fifo {
    my @queue;
    return (
        sub {return shift @queue;}, 
        sub ($item) {push @queue, $item;}
        ) ;
}
my ($fifo-get, $fifo-put) = create-fifo();
$fifo-put($_) for 1..10;
print " ", $fifo-get() for 1..5; # ->  1 2 3 4 5
\end{verbatim}
%
\index{closure}
\index{scope}
\index{FIFO}
\index{anonymous subroutine}
The center piece here is the {\tt create-fifo} subroutine.
The \verb'@queue' array holding the data is lexically 
scoped to this subroutine and cannot be accessed directly 
from anywhere else in the program. {\tt create-fifo} returns 
two anonymous subroutines, one to dequeue items and one 
to enqueue them. These subroutines are lexical \emph{closures}, 
which means in simple terms that they can access to 
\verb'@queue', because they have been defined within its 
scope, even if they are called from somewhere else. Even 
when {\tt create-fifo} has completed, those subroutines 
can still access to it because they sort of give an extra 
life to the array as long as the subroutines are accessible.

\index{function factory}
The rest of the code should be clear: when {\tt create-fifo} is 
called, it manufactures the two anonymous subroutines which 
are stored into the \verb'$fifo-get' and \verb'$fifo-put' 
variables. A subroutine such as {\tt create-fifo} is 
sometimes called a \emph{function factory} because it 
generates other subroutines at run time.
Finally, \verb'$fifo-put' is called ten times to 
populate the queue with integers from 1 to 10, and 
\verb'$fifo-get' is called five times to get the first five 
items of the queue. The queue is encapsulated: there is 
no way to access to its data other than using the two 
anonymous subroutines.

Making it possible to enqueue a list of items (rather 
than a single one) and managing exceptions (such as 
trying to get an item from an empty queue) are left 
as an exercise to the reader.

\index{functional programming}
\index{functional!programming}
The techniques used here borrow heavily on a programming 
paradigm called \emph{functional programming}, a model of 
programming used by languages such as Lisp, Caml, Clojure or 
Haskell. This paradigm is quite different from almost 
everything we have seen so far, just as object-oriented 
programming is yet another different paradigm. As you gain 
more experience as a programmer, you should make a point 
to understand these different paradigms, because they offer 
different ways of doing things, and the all have specific 
advantages for specific types of problems. Knowing all of them 
gives you more expressive power. One of the good things with 
Perl~6 is that it gives you a modern and powerful tool to use 
each of these programming paradigms. 
Chapter~\ref{functional programming} is all about functional 
programming, and make sure to read 
Subsection~\ref{array_functional_programming} in the 
array and list chapter.

\subsection{Exercise of Section~\ref{modify_array}: Other ways to modify an array (p.~\pageref{splice_exercise})}
\label{sol_splice_exercise}

\subsubsection{Simulating the {\tt pop} function}

The {\tt my-pop} subroutine uses {\tt splice} to simulate 
the {\tt pop} function:  

\begin{verbatim}
sub my-pop (@array where @array > 0) {
    my @result = splice @array, @array.end, 1;
    return @result[0];
}
my @letters = 'a'..'j';
my $letter = my-pop @letters;
say $letter;             # -> j
say @letters;            # -> [a b c d e f g h i]
\end{verbatim}

Here, the expression \verb'@array.end' returns the index of 
the last item of the array. It is also possible to count 
the array items from the end and to access to the last and 
penultimate items of a list or an array using the following 
syntax:

\begin{verbatim}
> say (1..9)[*-1];
9
> say (1..9)[*-2];
8
\end{verbatim}

The {\tt my-pop} subroutine could be rewritten as follows:

\begin{verbatim}
sub my-pop (@array where @array > 0) {
    my @result = splice @array, *-1, 1;
    return @result[0];
}
\end{verbatim}

You don't have to specify the number of elements with 
{\tt splice} if you just want the rest. We can also 
avoid using the \verb'@result'  intermediate array. 
So we could simplify {\tt my-pop} as:
\index{pop function}
\index{splice function}

\begin{verbatim}
sub my-pop (@array where @array > 0) {
    @array.splice(*-1)[0]
}
\end{verbatim}


\subsubsection{Simulating the {\tt push} function}

The only slight difficulty in this exercise is to manage a signature with a 
``variadic'' list of parameters (or slurpy parameters). This 
has been explained above in subSubsection~\ref{slurpy_params} 
(p.~\pageref{slurpy_params}).
\index{push function}
\index{slurpy parameters}
\index{variadic parameters}
\index{signature}

\begin{verbatim}
sub my-push (@array, *@list) {
    my @result = splice @array, @array.end + 1, 0, @list;
    return @array; # push returns the modified list
                   # (seldom used for arrays)
}
my @letters = 'a'..'j';
my-push  @letters, 'k', 'l', 'm';
say @letters;      # -> [a b c d e f g h i j k l m]
\end{verbatim}

\subsubsection{Simulating the {\tt unshift} function}

To simulate the \verb'unshift' function, we can use 
again slurpy parameters.
\index{slurpy parameters}
\index{unshift function}

\begin{verbatim}
sub my-unshift (@array, *@list) {
    my @result = splice @array, 0, 0, @list;
    return @array; # unshift returns the modified list
                   # (seldom used for arrays)
}
my @letters = 'd'..'j';
my-unshift @letters, 'a'..'c';
say @letters;      # -> [a b c d e f g h i j]
\end{verbatim}

\subsubsection{Simulating the {\tt delete} subscript adverb}

Remember the {\tt delete} adverb removes the value, but 
leaves the slot undefined within the array. The {\tt splice} 
function would also remove the slot, so this might not be 
what is really needed here if we want to simulate the 
behavior of {\tt delete} (although, in a sense, it might 
also be considered to be an improvement to remove the slot
altogether). To really simulate {\tt delete}, it is probably 
better to just ``undefine'' the value:
\index{delete adverb}

\begin{verbatim}
sub my-delete (@array, $idx) {
    my $item = @array[$idx];
    @array[$idx] = Nil;
    return $item;
}
my @letters = 'a'..'j';
my $letter = my-delete @letters, 4;
say $letter;       # -> e
say @letters;      # -> [a b c d (Any) f g h i j]
\end{verbatim}

\subsection{Exercise of Section~\ref{map_filter}: Mapping and filtering the elements of a list (p.~\pageref{exercise_squares})}
\label{sol_exercise_squares}

Producing an array containing the square of the numbers in the input list is very straight forward:
\index{map function}

\begin{verbatim}
my @squares = map { $_ ** 2 }, 3, 5, 7;     # -> [9 25 49]
\end{verbatim}
%

To keep the elements of a list that are perfect squares, one 
way is to check for each number whether its square root 
is an integer. For example:
\index{grep function}

\begin{verbatim}
my @filt = grep { my $sq = sqrt $_; $sq == $sq.Int}, 3, 9, 8, 16;
say @filt;                                  # -> [9 16]
\end{verbatim}
%

This is working fine with the sample data of the example test, 
but the program will abort if we try it with a negative input 
value. We might want to avoid that exception and just 
consider that a negative number is not a perfect square.

Since the code block here would be getting a bit more 
complicated, we might prefer to use a function 
reference instead:

\begin{verbatim}
sub is-square (Numeric $num} { 
    return False if $num < 0;
    my $sq = sqtr $num;
    return $sq == $sq.Int;
} 
my @filt = grep &is-square, 3, 9, -6, 16;   # -> [9 16]
\end{verbatim}
%
\subsection{Exercise of Section~\ref{advanced_sort}: Advanced sorting techniques (p.~\pageref{sort_exercise})}
\label{sol_sort_exercise}

The transformation subroutine that can extract the letter 
groups from the strings is quite straight forward:

\begin{verbatim}
sub my_comp (Str $str) {
    return $0 if $str ~~ /^\d+ (\w+)/; 
    Nil;   # returning Nil if the regex did not match
}
\end{verbatim}
%
\index{sort}
The sort is just the same as in the original chapter:
\begin{verbatim}
say sort &my_comp, <22ac 34bd 56aa3 12c; 4abc( 1ca 45bc>;
     # -> (56aa3 4abc( 22ac 45bc 34bd 12c; 1ca)
\end{verbatim}

The transformation subroutine is simple enough to be easily 
replaced by a code block:
\index{sort!transformation subroutine}

\begin{verbatim}
my @unsorted = <22ac 34bd 56aa3 12c; 42acd 12cd; 4abc( 1ca 45bc 3dab!>;
my @sorted = sort {/\d+ (\w+)/; $0 // Nil}, @unsorted;
say @sorted; 
     # -> [56aa3 4abc( 22ac 42acd 45bc 34bd 12c; 1ca 12cd; 3dab!]
\end{verbatim}
%
This can also be written with a method invocation syntax:
\begin{verbatim}
my @sorted = @unsorted.sort: {/\d+ (\w+)/; $0 // Nil};
\end{verbatim}
%

\subsection{Exercise~\ref{nested_sum}: Nested sum (p.~\pageref{nested_sum})}
\label{sol_nested_sum}

The most obvious way to compute the sum of all values contained 
in nested lists or arrays is to use nested loops. For example:
\index{nested list}

\begin{verbatim}
my @AoA = [[1, 2], [3], [4, 5, 6]];
sub nested-sum (@array) { 
    my $sum; 
    for @array -> $item { 
        for $item.flat -> $nested_item {
            $sum += $nested_item;
        }
    } 
    return $sum
}
say nested-sum @AoA;  # -> 21
\end{verbatim}

The only slight syntactical difficulty here is that 
we need to flatten the \verb'$item' sublists in order 
to traverse them. This could also be done with the ``|'' 
operator:
\begin{verbatim}
        for |$item -> $nested_item {
\end{verbatim}

This is another way to do it, using a {\tt for} loop for 
traversing the outer array and a reduction operator 
for adding the elements of the nested lists:

\begin{verbatim}
my @AoA = [[1, 2], [3], [4, 5, 6]];
sub nested-sum (@array) { 
    my $sum; 
    for @array -> $item { 
        $sum += [+] $item;
    } 
    return $sum
}
say nested-sum @AoA;  # -> 21
\end{verbatim}

Using {\tt map} for flattening the nested lists and 
the reduction operator can make this code considerably 
shorter:

\begin{verbatim}
my @AoA = [[1, 2], [3], [4, 5, 6]];
sub nested-sum (@array) { 
    return  [+]  map {|$_}, @array;
}
say nested-sum @AoA;  # -> 21
\end{verbatim}

Comparing this solution needing one line of actual code 
with the first one shows how expressive the functional 
programming style can be for handling arrays and lists 
and hopefully tells you one of the reasons why we have 
been insisting on this programming style in this 
chapter.

These solutions work well because it is known that there 
are at most lists and nested lists (lists of lists). What 
if the level of ``nestedness'' is not known in advance 
and can be higher than two? A solution would be to use 
a recursive subroutine to explore the tree of lists:
\index{recursion}

\begin{verbatim}
my @AoA = [[1,2], [3], [4,5,6], [3, [7,6, [3,2]]]];
sub nested-sum ($input) { 
    my $sum = 0; 
    for |$input -> $item { 
        if $item.WHAT ~~  Int {
            $sum += $item;
        } else {
            $sum += nested-sum $item;
        }
    } 
    return $sum;
}
say nested-sum @AoA;  # -> 42
\end{verbatim}

Remember that a recursive approach if often an efficient 
tool when dealing with nested or chained data. 

\subsection{Exercise~\ref{cumsum}: Cumulative sum (p.~\pageref{cumsum})}
\label{sol_cumsum}

To compute the cumulative sum of a list of numeric values, 
we just need an accumulator and we push the value of 
the accumulator each time through the iteration 
on the array:
\index{accumulator!list}

\begin{verbatim}
my @numbers = <2 5 7 6 5 3 6 8>;
say cumul-sum(@numbers); # -> [2 7 14 20 25 28 34 42]

sub cumul-sum (@array) {
    my @cumulative;
    my $partial_sum = 0;
    for @array -> $element {
        $partial_sum += $element;
        push @cumulative, $partial_sum;
    }
    return @cumulative;
}
\end{verbatim}

But guess what? The code can be much shorter with 
functional programming. Remember that the reduction 
metaoperator can give you a list of partial results:
\index{reduction operator}
\index{metaoperator}

\begin{verbatim}
my @numbers = <2 5 7 6 5 3 6 8>;
say [\+] @numbers;    # -> (2 7 14 20 25 28 34 42)
\end{verbatim}

You might think at this point that I have designed these 
exercises to make my point about the expressive power 
of functional programming. Not at all! Both this 
exercise and the previous one are straight from 
the list chapter of Allen Downey's \emph{Think Python} 
book on which this book is loosely based. 
I haven't written these two exercises, but only the 
solutions presented here.


\subsection{Exercise~\ref{middle}: Middle (p.~\pageref{middle})}
\label{sol_middle}

The easiest way to produce a new list that contains 
all but the first and last elements of a given list 
is probably to use simply a slice:

\begin{verbatim}
say middle(5..10);    # -> (6 7 8 9)
sub middle (@array) { 
    return @array[1..*-2] 
}
\end{verbatim}

Note that \verb'*-1' refers to index of the last element 
of an array. To discard the last element, we limit the 
range to \verb'*-2'.

\subsection{Exercise~\ref{chop}: Chop (p.~\pageref{chop})}
\label{sol_chop}

The basic difference with the previous exercise is that 
the array should be modified in place, rather than 
returned from the function.

Here's one possible way, which uses the {\tt shift} and 
{\tt pop} functions to remove respectively the first 
and the last element of the array:
\index{shift function}
\index{pop function}

\begin{verbatim}
my @nums = 5..10;
chop-it(@nums); 
say @nums;   # -> [6 7 8 9]

sub chop-it (@array) { 
    shift @array; 
    pop @array; 
    return;
}
\end{verbatim}

Using a slice is somewhat simpler, just make sure to 
assign the slice to the array in order to modify the 
array in-place:
\index{slice}

\begin{verbatim}
sub chop-it (@array) { 
    @array = @array[1..*-2];
    return;
}
\end{verbatim}
%

\subsection{Exercise~\ref{is_sorted}: Subroutine is-sorted (p.~\pageref{is_sorted})}
\label{sol_is_sorted}
\index{is-sorted}

To check whether a list is sorted, we just need to iterate 
over its items, keep track of the previous value and compare 
the current value with the previous one. Return false if 
any pair of values does not satisfy the comparison, and 
return true upon getting at the end of the iteration.

\begin{verbatim}
sub is-sorted (@array) {
    my $previous = @array[0];
    for @array -> $current {
        return False if $current < $previous;
        $previous = $current;
    }
    return True;
}
say is-sorted < 2 4 5 7 7 8 9>;    # -> True
say is-sorted < 2 4 5 7 6 8 9>;    # -> False
\end{verbatim}

Another approach might be to simply compare the input 
list with a sorted version of the same:
\index{sort}

\begin{verbatim}
sub is-sorted (@array) {
    return @array eqv @array.sort;
}
say is-sorted < 2 4 5 7 7 8 9>;    # -> True
say is-sorted < 2 4 5 7 6 8 9>;    # -> False
\end{verbatim}

While this leads to short and simple code, this is not 
an optimal solution, because it forces the program to 
sort the input array, which is significantly more costly 
than just traversing the array, at least when the array 
to be checked is large.

Once again, functional programming and especially the 
reduction hyperoperator can lead to much shorter code 
than the first solution, without incurring the cost 
of an additional sort:
\index{functional programming}
\index{reduction metaoperator}

\begin{verbatim}
sub is-sorted (@array) {
    return [<=] @array }
}
say is-sorted < 2 4 5 7 7 8 9>;    # -> True
say is-sorted < 2 4 5 7 6 8 9>;    # -> False
\end{verbatim}

By the way, this last version of {\tt is-sorted} will 
``short-circuit'' and return False as soon as it has 
found values not in the proper order, without iterating 
over the rest of the list.

\subsection{Exercise~\ref{is_anagram}: Subroutine is-anagram (p.~\pageref{is_anagram})}
\label{sol_is_anagram}
\index{anagram}
\index{is-anagram}

When comparing two words to see if they are anagram, we 
can start by returning false if they are not the same 
length, since anagrams obviously have the same letter 
count. This might make the process faster if the detailed 
process to compare two strings is time consuming, by 
avoiding the time consuming part for cases that will 
obviously not match.

We don't want to try every permutation of letters. The 
easiest way to check for anagrams is probably to start 
by \emph{normalizing} the input strings, i.e. 
reorganizing them in such a way that they can easily 
be compared. The most obvious way is just to sort the 
letters of the two words and compare the results:
\index{sort}

\begin{verbatim}
sub is-anagram (Str $word1, Str $word2) {
    return False if $word1.chars != $word2.chars;
    return False if $word1.comb.sort ne $word2.comb.sort;
    True;
}
for <ban bane post stop pots stop post pots pots taps> -> $w1, $w2 {
    say "$w1 $w2:\t", is-anagram $w1, $w2;
}
\end{verbatim}

which produces the following output:
\begin{verbatim}
$ perl6 anagrams.pl6
ban bane:       False
post stop:      True
pots stop:      True
post pots:      True
pots taps:      False
\end{verbatim}

Note that this works correctly because the {\tt ne} 
operator coerces its argument into a string before 
performing the comparison.
\index{ne operator}

This code can be made shorter (but possibly slightly less 
efficient) by returning directly the comparison of 
the sorted versions:

\begin{verbatim}
sub is-anagram (Str $word1, Str $word2) {
    return $word1.comb.sort eq $word2.comb.sort;
}
\end{verbatim}


\subsection{Exercise~\ref{has_duplicates}: Subroutine has-duplicates (p.~\pageref{has_duplicates})}
\label{sol_has_duplicates}
\index{duplicate}
\index{has-duplicates}

Within the context of what we have seen so far, the 
easiest way to find out if a list of strings has 
duplicates is probably to sort the list, so that 
possible duplicates will be adjacent, and to 
compare each item of the sorted array with the 
previous (or next) one.
\index{sort}

\begin{verbatim}
say has-duplicates( < a b c df g xy z r e >); # -> False
say has-duplicates( < a b c df g xy z c e >); # -> True

sub has-duplicates (@array) {
    my @sorted = sort @array;
    for 1..@sorted.end -> $i {
        return True if @sorted[$i] eq @sorted[$i - 1];
    }
    False;
}
\end{verbatim}
%

Here, the loop starts on index 1 (and not 0) because 
each item is compared with the previous one.

Another way is to iterate on the elements of the sorted 
array and to keep track of the previous item to enable 
the comparison:

\begin{verbatim}
say has-duplicates( < a b c d f y z r e >); # -> False
say has-duplicates( < a b c d f y z c e >); # -> True

sub has-duplicates (@array) {
    my @sorted = sort @array;
    my $previous = shift @sorted;
    for @sorted -> $item {
        return True if $item eq $previous;
        $previous = $item;
    }
    False;
}
\end{verbatim}
%

Another possibility is to use the {\tt unique} function 
of Perl~6, which returns a sequence of unique values 
from the input list or array. Comparing the item count 
of the output of {\tt unique} with the element count of 
the original list will tell us whether some duplicates were 
removed by {\tt unique}:
\index{unique function}

\begin{verbatim}
sub has-duplicates (@array) {
    my @unique-items = unique @array;
    return False if @unique-items.elems == @array.elems;
    True;
}
\end{verbatim}

This could be rewritten more concisely by chaining the 
method invocations:

\begin{verbatim}
sub has-duplicates (@array) {
    @array.unique.elems != @array.elems;
}
\end{verbatim}

Note that Perl also has a {\tt repeated} built-in 
function, which is the counterpart of {\tt unique} 
and returns the duplicates of a list:
\begin{verbatim}
say <a b c d b f d>.repeated;  # -> (b d)
\end{verbatim}

The {\tt has-duplicates} subroutine can just coerce 
the output of {\tt repeated} into a Boolean:
\begin{verbatim}
sub has-duplicates (@array) {
    ?@array.repeated
}
\end{verbatim}


Another efficient way of finding or removing duplicates 
from a list or an array is to use \emph{hashes}, a 
built-in data structure which we cover in 
Chapter~\ref{hashes} (see Exercise~\ref{has_duplicates_hash}).
\index{hash}

\subsection{Exercise~\ref{birthdays}: Simulating the birthday paradox (p.~\pageref{birthdays})}
\label{sol_birthdays}
\index{birthday paradox}
\index{duplicate}

For simulating the birthday paradox, we need to generate 
random integers between 1 and 365 (each integer 
representing a date in the year). For the sake of 
simplicity, we will generate random integers between 
0 and 364, which is equivalent for our purpose.

And we will run the simulation 1,000 times.

\begin{verbatim}

sub has-duplicates (@array) {
    return ?@array.repeated
}

sub has-duplicate-birthdays (Int $num-students) {
    my @blist;
    for 1..$num-students {
        push @blist, 365.rand.Int; # numbers between 0 and 364
    }
    return has-duplicates(@blist);
}

my $dupl-count = 0;
my $nb-tests = 1000;
for 1..$nb-tests {
    $dupl-count++ if has-duplicate-birthdays 23; # 23 students
}
say "On $nb-tests tests, $dupl-count had at least one duplicate birthday";
\end{verbatim}

Note that we have reused here the {\tt has-duplicates} 
subroutine of the previous exercise. It is so short 
that its code could have been in-lined in the 
\verb'populate-birthdays'  
subroutine, but it is 
generally considered good practice to reuse software 
components that have been developed and tested.

Running the program four times gave the following 
results:

\begin{verbatim}
$ perl6 birthdays.pl6
On 1000 tests, 498 had at least one duplicate birthday

$ perl6 birthdays.pl6
On 1000 tests, 505 had at least one duplicate birthday

$ perl6 birthdays.pl6
On 1000 tests, 527 had at least one duplicate birthday

$ perl6 birthdays.pl6
On 1000 tests, 491 had at least one duplicate birthday
\end{verbatim}

\index{birthday paradox}
This simulation confirms that on a sample of 23 persons, 
there is an approximate 50\% probability that at least 
two will have the same birthday.

Note that Perl has a {\tt roll} built-in which returns 
randomly selected elements from a list. This can make 
the {\tt populate-birthdays} subroutine significantly 
more concise:

\begin{verbatim}
sub has-duplicate-birthdays (Int $num-students) {
    has-duplicates( (^365).roll($num-students) )
}
\end{verbatim}

\subsection{Exercise~\ref{push_unshift}: Comparing push and unshift (p.~\pageref{push_unshift})}
\label{sol_push_unshift}
\index{push function}
\index{unshift function}
\index{now}

Populating an array with either {\tt push} or 
{\tt unshift} is something you've seen before. The only new thing 
here is to compare run times of various solutions.
The {\tt now} function returns the number of seconds 
elapsed since a theoretical start point called ``the 
Epoch'', usually Jan 1, 1970. Calling {\tt now} once 
before running a script and once after will tell 
us how long it ran through a simple subtraction.

\begin{verbatim}
my $start_push = now;
my @push_array;
for 'words.txt'.IO.lines -> $line {
    push @push_array, $line;
}
say "push took " ~ now - $start_push ~ " seconds.";
@push_array = ();

my $start_unsh = now;
my @unsh_array;
for 'words.txt'.IO.lines -> $line {
    unshift @unsh_array, $line;
}
say "unshift took " ~ now - $start_unsh ~ " seconds.";
@unsh_array = ();
\end{verbatim}

This is a sample run of this program:
\begin{verbatim}
push took 1.870107 seconds.
unshift took 2.2291266 seconds.
\end{verbatim}

Try it for yourself, run it several time, you should 
probably notice that {\tt push} is consistently 
faster than {\tt unshift}, even though the difference 
is not that large.

The reason is presumably that since {\tt unshift} is 
inserting items at the start of the array, Perl has 
to move data around in order to reorganize the whole 
array many times over, whereas, using {\tt push} 
for inserting items at the end of the array implies 
less internal house keeping.

As a further exercise, you may try to explore other 
ways to populate an array, such as {\tt append} or 
{\tt splice}.
\index{splice function}
\index{append}

If you are just going to insert each line from the 
input file into an array without changing anything to 
those lines, then slurping the data into the array 
without a loop will be simpler and much faster:
\index{slurp}

\begin{verbatim}
my $start_slurp = now;
my @slurp_array = 'words.txt'.IO.lines;
say "slurp took " ~ now - $start_slurp ~ " seconds.";
\end{verbatim}

This is four to five times faster:
\begin{verbatim}
slurp took 0.42602506 seconds.
\end{verbatim}

\subsection{Exercise~\ref{bisection}: Bisection search in a list (p.~\pageref{bisection})}
\label{sol_bisection}
\index{bisect}
\index{membership!bisection search}
\index{bisection search}
\index{search!bisection}
\index{membership!binary search}
\index{binary search}
\index{search!binary}
\index{half-interval search}

We can start with a recursive bisection algorithm:
\index{recursion}


\begin{verbatim}
sub bisect (@word_list, Str $word) {
    my $index = (@word_list.elems / 2).Int;
    return False if $index == 0 and @word_list[$index] ne $word;
    my $found = @word_list[$index];
    if $word lt $found {
        # search the first half
        return bisect @word_list[0..$index-1], $word;
    } elsif $word gt $found {
        # search the second half
        return bisect @word_list[$index+1..*-1], $word;
    }
    True;     # if we get there, we've found the word
}

for <a f w e q ab ce> -> $search { 
    if bisect [<a b d c e f g>], $search {
        say "found $search";
    } else {
        say "did not find $search";
    }
}
\end{verbatim}

This will display the following output:
\begin{verbatim}
found a
found f
did not find w
found e
did not find q
did not find ab
did not find ce
\end{verbatim}

There are a couple of weaknesses, though, in this 
implementation. First, on each recursive call, 
{\tt bisect} passes as an argument an array 
which may be quite large, and this is not very 
efficient both in terms of memory usage (to store 
the successive subsets of the original array) and 
in terms of CPU cycles (the time to copy these 
arrays).

In addition, we can figure out whether the target word can 
be found in the list (and there are many cases where 
we don't need more information than that), but we don't 
know where it was found (i.e. on which subscript of 
the original array), which is often what is really 
needed.

A better option might be to have only one copy of the 
original array, say as global variable, and to pass 
around subscript ranges. But global variables are 
usually very much frowned upon because they tend to go 
against the tenets of structured programming and can 
be dangerous (even though this would arguably be a 
case where a global variable does make some sense). 
We can actually do better than global 
variables and still have the benefit of not passing 
the whole array around again and again thanks to the 
fact that, in Perl~6, subroutines are \emph{closures}, 
which means that they can use variables that exist in 
the environment where they are created.
\index{closure}

In the following code, {\tt bisect} is no longer a 
recursive subroutine, it is a very simple subroutine 
that just sets up the environment for {\tt bisect2},
which is the recursive routine and is defined within 
the body of {\tt bisect}. Because the array and the 
searched word exist within {\tt bisect}, {\tt bisect2} 
will be able to access to them. The parameters to 
{\tt bisect2} are now just two subscripts representing 
the range in which it will have to look up for the 
word.
\index{recursion}

\begin{verbatim}
sub bisect (Str $word, @word_list) {
    sub bisect2 ($low_idx, $high_idx) {
        my $mid_idx = (($low_idx + $high_idx) /2).Int;
        my $found = @word_list[$mid_idx];
        return $mid_idx if $word eq $found;
        return -1 if $low_idx >= $high_idx;
        if $word lt $found {
            # search the first half
            return bisect2 $low_idx, $mid_idx - 1;
        } else {
            # search the second half
            return bisect2 $mid_idx+1, $high_idx;
        }
    }
    my $max_index = @word_list.end;
    return bisect2 0, $max_index;
}

for <a f w e q ab ce g> -> $search { 
    my $result = bisect $search, [<a b d c e f g>];
    if $result == -1 {
        say "did not find $search";
    } else {
        say "found $search on position $result";
    }
}
\end{verbatim}

As a further exercise, adapt the above program to 
search for English words in {\tt words.txt}. Notice 
how fast this is. Please 
be aware that it works correctly because the words 
in this file are listed in alphabetical order.

Try to change 
the code to count and display the number of steps 
necessary to find a given word. Compare with the number 
of steps it would take on average for a linear search 
(i.e. traversing the array linearly until the word 
is found or can be declared to be absent. Can you 
guess why this search algorithm is sometimes 
called \emph{logarithmic search}?
\index{logarithmic search}

You may also want to try to write a non-recursive solution 
using a loop.


\subsection{Exercise~\ref{reverse_pair}: Reverse pairs (p.~\pageref{reverse_pair})}
\label{sol_reverse_pair}
\index{reverse word pair}

Finding reverse pairs requires reading each word of 
the list and checking the list to see whether the reversed 
words exist in the list. This means that you are 
going to look up about 113,000 words in a list having 
113,000 words. Your lookup method needs to be efficient.
The obvious solution is to use the bisection search 
implemented in the previous exercise.

\begin{verbatim}
sub bisect (Str $word, @word_list) {
    # see the code in the previous exercise
}

my @array = 'words.txt'.IO.lines;

for @array -> $word {
    my $reverse = $word.flip;
    my $res = bisect $reverse, @array;
    say "$word and $reverse form a reverse pair" if $res >= 0;
}
say now - INIT now;
\end{verbatim}

On my laptop (a decent box, but not a racehorse), 
the whole process ran in about 42~seconds, i.e. 
less than 0.4~millisecond per look-up.

If you think about it, the {\tt for} loop in the code 
above is really filtering from the word list those words 
belonging to a reverse pair. This could be implemented 
with a {\tt grep} using the {\tt bisect} subroutine for 
selecting the matches:

\begin{verbatim}
say "$_ and $_.flip() form a reverse pair" 
    for @array.grep( { bisect( .flip, @array ) >= 0 } );
\end{verbatim}

With the algorithm used here, each reverse pair is found 
twice (once for each word of the pair). When examining 
any given word from the list, we actually don't need to 
search backward in the part of the list before that word because 
if that word forms a pair with another word that comes before in 
alphabetic order, we've already found the pair when 
processing that other word. So it would be more efficient and 
faster to search only forward, i.e. to look for the reverse 
word in the part of the list coming after the word being 
examined. As a further exercise, modify the {\tt for} loop 
to search words forward.

\subsubsection{Comparing bisection search with hash lookup}

Bisection search is fairly fast, but hash lookup is 
even faster. Although we haven't studied hashes yet, 
try the following code:
\index{hash}

\begin{verbatim}
my %hash = map { $_ => 1}, 'words.txt'.IO.lines;
for %hash.keys -> $word {
    my $reverse = $word.flip;
    say "$word and $reverse form a reverse pair" 
        if %hash{$reverse}:exists;
}
say now - INIT now;
\end{verbatim}

Don't worry about understanding the code for the time 
being, but notice how much shorter it is. And how much 
faster it runs: on the same laptop, the execution time is 
about 16~seconds (less than 0.15 millisecond per lookup). 
I hope this will whet your appetite for Chapter~\ref{hashes}.

Note that the output of this example is not sorted because 
a hash does not keep the order of the input data, as we 
will see in Chapter~\ref{hashes}. It would be fairly easy to 
keep the sort order, for example by using an array in 
addition to the hash, but that is not really the 
subject here.

\subsubsection{Creating and using a module}

\index{module}
\label{bisect_module}
Coming back to the {\tt bisect} subroutine, copying 
and pasting this subroutine from the program of the 
previous exercise into the code of this exercise is 
not the best way to reuse code. Suppose a bug is 
found in that subroutine, it now needs to be fixed 
in two different programs; the chance that the bug gets 
corrected in one program and forgotten for the other 
is quite significant. Even if it is not forgotten, this 
is twice the same work, and this increases the chance 
of making another mistake in the process. The bug fix 
also needs testing twice. Even if there is no bug, we 
might need an enhancement and this again has to be done 
twice. 
\index{code reuse}

A good way to reuse software while maintaining only 
one copy of the reused subroutine is to insert it into 
a Perl module, i.e. in a separate file that will be 
loaded into our programs needing to use it.

The module file might be named \emph{BisectSearch.pm} 
and contain the following code:
\index{BisectSearch module}
\index{module!BisectSearch}
\index{module!creating a module}
\index{is export!trait}
\index{trait!is export}

\begin{verbatim}
unit module BisectSearch;

sub bisect (Str $word, @word_list) is export {
    sub bisect2 ($low_idx, $high_idx) {
        my $mid_idx = (($low_idx + $high_idx) /2).Int;
        my $found = @word_list[$mid_idx];
        return $mid_idx if $word eq $found;
        return -1 if $low_idx >= $high_idx;
        if $word lt $found {
            # search the first half
            return bisect2 $low_idx, $mid_idx - 1;
        } else {
            # search the second half
            return bisect2 $mid_idx+1, $high_idx;
        }
    }
    my $max_index = @word_list.end;
    return bisect2 0, $max_index;
}

sub some-other-sub is export {
    # does something useful
}
\end{verbatim}

Note that the module name given at the top of the code 
and the root file name have to correspond. The only other 
change to the code of the subroutine is adding the 
{\tt is export} trait to the signature of the subroutine.

\index{module!using a module}
Now a Perl program will be able to load this module 
and to use the {\tt bisect} and {\tt some-other-sub} 
subroutines. For example:
\index{use lib}
\index{module!use}
\index{use module}
\begin{verbatim}
use lib ".";  # tells Perl to look for modules in the current dir
use BisectSearch;

my @array = 'a'..'m';
for < a f w e q ab ce g > -> $search { 
    my $result = bisect $search, @array;
    if $result == -1 {
        say "did not find $search";
    } else {
        say "found $search : item # $result";
    }
}
\end{verbatim}
%

Perl has a list of places where to look for modules, 
which may vary from a Perl installation to another. 
The first line {\tt use lib ".";} tells Perl to also 
look for modules into the current directory. This is 
just an example; you might prefer using a dedicated 
directory for your modules. The 
second line {\tt use BisectSearch;} tells Perl to 
load module and import the exported subroutines. Now, 
the program can use the {\tt bisect} subroutine just as if 
it had been defined within the program.

That's it, folks! Simple, isn't it? Just try it! Well, there 
are a few more things to know about modules, but you 
already know enough to create and use modules. 

You might want to review some of the other subroutines we 
have written so far and stick into a module those which 
might be useful again. Hint: some of the array and string 
subroutines we've seen are likely candidates.
\index{module}

\subsection{Exercise~\ref{interlock}: Interlocking words (p.~\pageref{interlock})}
\label{sol_interlock}

First, it seems that it was a good idea creating the 
BisectSearch module, it's going to be reused immediately.

Second, we need some thinking. The first idea that might 
come to mind to solve the problem might be to have a 
nested loop on the word list in order to find all pairs 
of two words, to interlock them and to see if the resulting 
combined string exists in the list. But this is quite bad 
because this means creating 113-k squared pairs, i.e. more 
than 12.5~billion pairs. Even if a relatively large part 
of these pairs can be eliminated before having to look up in 
the word list since a pair can be interlocked only if the 
letter count difference between the two words is 0 or 1, 
checking all these pairs will take ages.

Let us see what happens if we work the other way around: for 
each word on the word list, we ``intersplit'' the word into 
one string with the even-rank letters and one with the odd-rank 
letters, and then check if these substrings belong to the 
list. At most, we will need 226-k~searches--in fact much less 
because we don't need to look up for the second string if the 
first string did not match anything.

This is our suggested solution:
\begin{verbatim}
use lib ".";
use BisectSearch;

my @array = 'words.txt'.IO.lines;
for @array -> $word {
    my ($word1, $word2) = intersplit($word);
    say "$word: $word1, $word2" if bisect($word1, @array) >= 0 
        and bisect($word2, @array) >= 0;
}

sub intersplit (Str $word) {
    my @letters = $word.comb;
    my $evens = join '', map {@letters[$_] if $_ %% 2}, @letters.keys;
    my $odds = join '', map {@letters[$_] if $_ % 2}, @letters.keys;
    return ($evens, $odds);
}
\end{verbatim}

The {\tt interplit} subroutine is not optimal in the 
sense that it traverses the \verb'@letters' array twice 
each time it is called. We can improve it using a pointy 
block taking two parameters (one odd- and one even-rank 
letters):
\begin{verbatim}
sub intersplit (Str $word) {
    my (@evens, @odds);
    for $word.comb -> $even, $odd {
        push @evens, $even;
        push @odds, $odd;
    }
    @evens.join, @odds.join;
}
\end{verbatim}

As a further exercise, can you find any words that are 
three-way interlocked; that is, every third letter forms a 
word, starting from the first, second or third? Hint: it 
will probably be easier if you start from the revised 
version of {\tt intersplit} just above.

\section{Exercises of Chapter~\ref{hashes} (Hashes)}

\subsection{Exercise at the end of Section~\ref{hash_descr}: A hash is a mapping (p.~\pageref{ex_employees})}
\label{sol_ex_employees}

Populating one pair at a time:

\begin{verbatim}
my %wages;
%wages{"Liz"} = 3000;
%wages{"Bob"} = 2500;
%wages{"Jack"} = 2000;
%wages{"Betty"} = 1800;
say "Bob's salary is %wages{'Bob'}";
for <Liz Jack> -> $employee {
    say "The salary of $employee is %wages{$employee};
}
\end{verbatim}

You can avoid quote marks around the keys using 
the \verb'<...>' angle brackets operator:

\begin{verbatim}
my %wages;
%wages<Liz> = 3000;
%wages<Bob> = 2500;
# ...
say "Bob's salary is %wages<Bob>";
\end{verbatim}

Assigning the full hash in one go:

\begin{verbatim}
my %wages = Liz => 3000, Bob => 2500, Jack => 2000, Betty => 1800;
say %wages; # -> Betty => 1800, Bob => 2500, Jack => 2000, Liz => 3000
\end{verbatim}

\subsection{Exercise~\ref{wordlist2}: Storing the word list into a hash (p.~\pageref{wordlist2})}
\label{sol_wordlist2}

The standard way might be to read each line of the file 
in a {\tt for} loop and store each word as the key of the 
hash. The content of the value is not important, we will 
store 1.

\begin{verbatim}
my %words;
for 'words.txt'.IO.lines -> $line {
    %words{$line} = 1
}
\end{verbatim}

An alternative approach is to assign to the hash the 
output of a {\tt map} expression returning a pair for each 
line of the file:

\begin{verbatim}
my %hash = map { $_ => 1}, 'words.txt'.IO.lines;
\end{verbatim}
%

\subsection{Exercise~\ref{mem_ackerman}: Memoizing the Ackermann function (p.~\pageref{mem_ackerman})}
\label{sol_mem_ackerman}

The original implementation of the Ackermann function looked 
like this:

\begin{verbatim}
sub ack ($m, $n) {
    return $n + 1 if $m == 0;
    return ack($m - 1, 1) if $n == 0;
    return ack($m - 1, ack($m, $n-1));
}
\end{verbatim}

It is not possible to memoize the cases where either \verb'$m' 
or \verb'$n' is zero, because the other value is unknown. Only 
the code corresponding to the last code line can be memoized, 
but that's okay because it does the bulk of the work anyway.

The next problem is that the hashes seen so far had only one 
key, but the Ackermann function takes two parameters. The 
simple work-around is to create a composite key, i.e. to 
concatenate the two parameters with a separator to create 
the keys of the hash. This leads to this possible solution:

\begin{verbatim}
my %ack-memo;
sub mem-ack (Int $m, Int $n) {
    return $n + 1 if $m == 0;
    return mem-ack($m - 1, 1) if $n == 0;
    %ack-memo{"$m;$n"} = mem-ack($m - 1, mem-ack($m, $n-1))
        unless %ack-memo{"$m;$n"}:exists;
    return %ack-memo{"$m;$n"};
}
say mem-ack 3, 4;
\end{verbatim} 

To benchmark the two solutions, you may use the following 
code:

\begin{verbatim}
my %ack-memo;
sub mem-ack (Int $m, Int $n) {
    return $n + 1 if $m == 0;
    return mem-ack($m - 1, 1) if $n == 0;
    %ack-memo{"$m;$n"} = mem-ack($m - 1, mem-ack($m, $n-1)) 
        unless %ack-memo{"$m;$n"}:exists;
    return %ack-memo{"$m;$n"};
}
my $start = now;
say mem-ack 3, 4;
say "mem-ack runtime: ", now - $start;
dd %ack-memo;

sub ack ($m, $n) {
    return $n + 1 if $m == 0;
    return ack($m - 1, 1) if $n == 0;
    return ack($m - 1, ack($m, $n-1));
}
$start = now;
say ack 3, 4;
say "ack runtime: ", now - $start;
\end{verbatim}

But don't try to run it with values of \verb'$m' greater 
than 3, it is useless. If we were to find an Ackermann 
value for a pair of numbers already seen, that would 
mean that we have entered an infinite loop. So there is 
no point memoizing the Ackermann function.

\index{multidimensional array}
\index{array!multidimensional}
We have used composite keys for \verb'%ack-memo', but we 
can have multidimensional hashes just as there are 
multidimensional arrays (see Section~\ref{multidimensional_array}. 
We only need to have two keys, each between its pair of 
curly brackets:
\index{multidimensional hash}
\index{hash!multidimensional}
\index{bracket!curly}
\index{curly bracket}
\begin{verbatim}
my %h;
%h{'a'}{'b'}= 'ab';
%h{'a'}{'c'}= 'ac';
%h{'a'}{'d'}= 'ad';
%h{'b'}{'c'}= 'bc';
dd %h; 
# -> Hash %h = {:a(${:b("ab"), :c("ac"), :d("ad")}), :b(${:c("bc")})}
\end{verbatim}
%

or to use a semi-colon to separate the keys:

\begin{verbatim}
my %i;
%i{'a';'b'} = 'ab';
%i{'a';'c'} = 'ac';
%i{'b';'c'} = 'bc';
dd %i; # -> Hash %i = {:a(${:b("ab"), :c("ac")}), :b(${:c("bc")})}
\end{verbatim}
%

\subsection{Exercise~\ref{has_duplicates_hash}: Finding duplicates with a hash (p.~\pageref{has_duplicates_hash})}
\label{sol_has_duplicates_hash}

We need to loop on the array, store the array elements into 
a hash and detect if an element is found in the hash.

\begin{verbatim}
sub has-duplicates (@array) {
    my %seen;
    for @array -> $elmt {
        return True if %seen{$elmt}:exists;
        %seen{$elmt} = 1;
    }
    return False;
}
\end{verbatim}

As a further exercise, generate a list of 50,000~random 
integers between 0 and 1,000,000,000, check on the same 
input list whether there are some duplicates with the 
various methods we have used and measure the run time of 
the various methods. If you encounter difficulties doing 
that, take a look at the solutions to the ``has duplicates'' 
(see Subsection~\ref{sol_has_duplicates}) and ``birthday paradox'' (see 
Subsection~\ref{sol_birthdays}) exercises to get some coding clues. An 
example of simple benchmarking is presented in the exercise just 
above.

Once your subroutines are working properly, launch the 
whole process at least 10 times to see if the differences 
are significant.

\subsection{Exercise~\ref{exrotatepairs}: Rotate pairs (p.~\pageref{exrotatepairs})}
\label{sol_exrotatepairs}

Consider the word ``sake'' and rotate it by 14~letters. This gives the word ``goys''. This also means that if ``goys'' 
is rotated by 12~letters, this will produce ``sake''. Since 
we are going to scan all the words of our word list, we will 
find this ``rotate pair'' when we try a shift of 12 letters 
on ``goys'', so that there no need to try a 14-letter rotation 
on ``sake''. More generally, we need to try only rotations 
between 1 and 13~letters. 

We iterate through the words of the list, rotate each of 
them by every shift between 1 and 13, and lookup the result 
in the hash.

\begin{verbatim}
sub rotate-one-letter (Str $letter, Int $shift) {
    my $last  = 'z'.ord;       # last lower-case letter
    my $rotated-ord = $letter.ord + $shift;
    if $letter ~~ /<[a..z]>/ { 
        $rotated-ord -= 26 if $rotated-ord > $last;
    } else {
        return $letter;
    }
    return $rotated-ord.chr;
}

sub rotate-one-word (Str $word, Int $shift) {
    my $rotated-word = "";
    for 0..$word.chars - 1 {
        $rotated-word ~=  rotate-one-letter substr($word, $_, 1), $shift;
    }
    return $rotated-word;
}

my %words = map { $_ => 1}, 'words.txt'.IO.lines;

for %words.keys -> $string {
    for 1..13 -> $shift {
        my $rotated = rotate-one-word $string, $shift;
        say " $string and $rotated are shifted by $shift"
            if %words{$rotated}:exists;
    }
}
\end{verbatim}

Rotating each word of a 113-k list by each shift between 
1 and 13 is quite long. Running the program on the word list 
will take some time, probably in the order of 10 to 15~minutes.
Using the \verb'.trans' built-in might speed up the process 
significantly.

\subsection{Exercise~\ref{homophones}: Homophones (p.~\pageref{homophones})}
\label{sol_homophones}

This is a solution using both the {\tt words.txt} word list 
used before and the CMU phonetic dictionary.

\begin{verbatim}
my %phonetic;

sub load-phonetic ($file-name) {
    for $file-name.IO.lines -> $line {
        next if $line !~~ /^\w/; 
        my ($key, $val) = $line.split("  ", 2);
        $key = lc $key;
        %phonetic{$key} = $val;
    }
}

load-phonetic('cmu_dict.txt');
my %words = map { $_ => 1}, 'words.txt'.IO.lines;

say "Starting the search";

for %words.keys -> $word {
    next unless %phonetic{$word}:exists;
    my $shorter = $word.substr(1);
    next unless %words{$shorter}:exists;
    next unless %phonetic{$shorter}:exists;
    next unless %phonetic{$word} eq %phonetic{$shorter};
    my $other-shorter = $word.substr(0, 1) ~ $word.substr(2);
    next unless %words{$other-shorter}:exists;
    next unless %phonetic{$other-shorter}:exists;
    next unless %phonetic{$other-shorter} eq %phonetic{$shorter};
    say "$word $shorter $other-shorter %phonetic{$shorter}"
}
\end{verbatim}

But this is somewhat inefficient because we don't actually 
need the word list, since the CMU dictionary is another 
word list that we can use. The following program uses 
only the CMU dictionary and saves the time to load the 
word list and to do checks on it:

\begin{verbatim}
my %phonetic;

sub load-phonetic ($file-name) {
    for $file-name.IO.lines -> $line {
        next if $line !~~ /^\w/; 
        my ($key, $val) = $line.split("  ", 2);
        $key = lc $key;
        %phonetic{$key} = $val;
    }
}

load-phonetic('cmu_dict.txt');

for %phonetic.keys -> $word {
    my $shorter = $word.substr(1);
    next unless %phonetic{$shorter}:exists;
    next unless %phonetic{$word} eq %phonetic{$shorter};
    my $other-shorter = $word.substr(0, 1) ~ $word.substr(2);
    next unless %phonetic{$other-shorter}:exists;
    next unless %phonetic{$other-shorter} eq %phonetic{$shorter};
    say "$word $shorter $other-shorter %phonetic{$shorter}"
}
\end{verbatim}

\section{Exercises of Chapter~\ref{data_struct_sel}}

\subsection{Exercise in Section~\ref{given_when}: the {\tt given ... when} switch statement (p.~\pageref{proceed_ex})}
\label{sol_proceed_ex}

To test the switch statement with various values, you might 
write something like this:

\begin{verbatim}
for <5 42 43 101 666 1024 2048> -> $value {
    given $value {
        when 0..9      { say "$_: One digit"}
        when 10..99    { say "$_: Two digits" ; proceed; }
        when 42        { say "$_: Response to the question" }
        when /^\d**3$/ { say "$_: Three digits" }
        default        { say "$_: More than three digits" }
    }
    say '';
}
\end{verbatim}

This will display the following result:

\begin{verbatim}
5: One digit

42: Two digits
42: Response to the question

43: Two digits
43: More than three digits

101: Three digits
(...)
\end{verbatim}

You can see the error when the input value is 43.

As a solution, it is possible to change the order of the 
{\tt when} clauses:

\begin{verbatim}
for <5 42 43 101 666 1024 2048> -> $value {
    given $value {
        when 0..9      { say "$_: One digit"}
        when 42        { say "$_: Response to the question"; proceed; }
        when 10..99    { say "$_: Two digits"}
        when /^\d**3$/ { say "$_: Three digits" }
        default        { say "$_: More than three digits" }
    }
    say '';
}
\end{verbatim}

This now works correctly, but the output for 42 is no longer 
in the same order. If we want to keep the original order,  
we may need to add a {\tt when} statement with an empty block:

\begin{verbatim}
for <5 42 43 101 666 1024 2048> -> $value {
    given $value {
        when 0..9      { say "$_: One digit"}
        when 10..99    { say "$_: Two digits"; proceed}
        when 42        { say "$_: Response to the question"; }
        when 10..99    { }
        when /^\d**3$/ { say "$_: Three digits" }
        default        { say "$_: More than three digits" }
    }
    say '';
}
\end{verbatim}

Or we could remove the need for {\tt proceed} by inserting 
the code for the 42 case into the two-digit block: 

\begin{verbatim}
for <5 42 43 101 666 1024 2048> -> $value {
    given $value {
        when 0..9      { say "$_: One digit"}
        when 10..99    { say "$_: Two digits"; 
                         say "$_: Response to the question" if $_ == 42
                       }
        when /^\d**3$/ { say "$_: Three digits" }
        default        { say "$_: More than three digits" }
    }
    say '';
}
\end{verbatim}

It would also be possible to nest a \verb'when' sub-expression 
within the \verb'when 10..99' expression:

\begin{verbatim}
for <5 42 43 101 666 1024 2048> -> $value {
    given $value {
        when 0..9      { say "$_: One digit"}
        when 10..99    { say "$_: Two digits"; 
                         when 42 {say "$_: Response to the question";}
                       }
        when /^\d**3$/ { say "$_: Three digits" }
        default        { say "$_: More than three digits" }
    }
    say '';
}
\end{verbatim}

\subsection{Exercise in Section~\ref{operator_construction}: constructing new operators (p.~\pageref{fact_operator})}
\label{sol_fact_operator}
\index{factorial}
\index{factorial!operator}

The ``!'' negation operator is a prefix operator (i.e. placed 
before the term that it negates). For the factorial operator, 
we need a postfix operator (placed after the term upon which it 
acts), so this difference will be sufficient to enable the 
Perl compiler to distinguish between the two operators.

We use the reduction metaoperator to compute the result:

\begin{verbatim}
sub postfix:<!> (Int $n) {
    [*] 2..$n;
}
say 5!; # -> 120
\end{verbatim}

The signature ensures the operand is an integer (failing which 
we get an error). We may want to guard against an 
negative integer, which we can do by raising an error if 
\verb'$n' is negative. In addition, we can use the {\tt Test} 
standard module to automatize our tests:
\index{test module}
\index{testing}

\begin{verbatim}
sub postfix:<!> (Int $n) {
    fail "The operand is not a positive integer" if $n < 0;
    [*] 2..$n
}
use Test;
plan 5;
dies-ok {(-1)!}, "Factorial fails for -1";
eval-dies-ok "(2.5)!", "Factorial fails for 2.5";
ok 0! == 1, "Factorial 0";
ok 1! == 1, "Factorial 1";
ok 5! == 120, "Factorial of a larger integer";
done-testing;
\end{verbatim}

\index{test plan}
The {\tt plan 5;} line says that the test plan contains 
5~individual tests. Then the two first tests check that 
the factorial operator fails for invalid input values. And 
it checks the output for some valid input.

The {\tt done-testing} specifies that the test has finished. 
This function is really useful when you don't have a plan 
for example when you don't know yet how many test you'll run. 
Here, we have a plan, so using done-testing isn't necessary.
\index{done-testing}

The following is the output of the tests:

\begin{verbatim}
1..5
ok 1 - Factorial fails for -1
ok 2 - Factorial fails for 2.5
ok 3 - Factorial 0
ok 4 - Factorial 1
ok 5 - Factorial of a larger integer
\end{verbatim}

If we had a test error on test~3, we would have obtained 
something like this:

\begin{verbatim}
ok 1 - Factorial fails for -1
ok 2 - Factorial fails for 2.5
not ok 3 - Factorial 0

# Failed test 'Factorial 0'
# at test_fact.pl6 line 8
ok 4 - Factorial 1
ok 5 - Factorial of a larger integer
1..5
# Looks like you failed 1 test of 5
\end{verbatim}

Here, we have put the tests in the same file as the subroutine 
definition for the sake of simplicity of the example. Normally, 
the tests would be in a separate file, usually in a ``t'' 
directory and with a \verb'.t' extension.

Testing and the {\tt Test} module are further discussed  
in Section~\ref{test_module} (p.~\pageref{test_module}).
More information about testing can be found at: 
\url{https://doc.perl6.org/language/testing}.

\subsection{Exercise in Section~\ref{sets_and_bags}: Sets, bags and mixes (p.~\pageref{diff_with_set})}
\label{sol_diff_with_set}

We can't just replace the \verb'%histogram' with a bag, because 
bags are immutable (i.e. cannot be changed after creation time) 
and the \verb'%histogram' hash is populated progressively as 
the lines of the book are being read from the file. You may use 
a baghash (the mutable version of a bag) and are encouraged 
to try it.

However, the aim here is to extract the words of the book that 
are not in the word list. In other words, we no longer care 
about word frequencies, but just need a unique list of words 
that appear at least once in the book, so that a set would 
be sufficient to satisfy our needs. The question is how to 
populate the set at creation time.

We can change the {\tt process-line} subroutine so that it 
process the line as previously but, instead of populating a 
hash, just returns the list of words. And create the set 
with a {\tt map} function calling that subroutine.

\begin{verbatim}
my $skip = True;                    # flag to skip the header
sub process-line(Str $line is copy) {
    $skip = False if defined index $line, "*END*THE SMALL PRINT!";
    next if $skip;
    $line ~~ s:g/<[-']>/ /;         # Replacing dashes and 
                                    # apostrophes with spaces
    $line ~~ s:g/<[;:,!?.()"_`]>//; # removing punctuation symbols
    $line = $line.lc;               # setting string to lower case
    return $line.words;
}

my $book-set = set map { process-line $_},  "emma.txt".IO.lines; 
my $word-list = set "words.txt".IO.lines;
my $unknown-words = $book-set (-) $word-list;
say $unknown-words.keys.head(20);
\end{verbatim}

This works well, but once we are there, we can also 
get rid of the \verb'$book-set' data structure and 
just filter directly the words extracted from the book:

\begin{verbatim}
my $skip = True; # flag to skip the header

sub process-line($line is copy) {
    # (same as above)
}

my $word-list = set "words.txt".IO.lines;
my @unknown-words = unique grep {$_ ∉ $word-list}, 
                    grep { $_ }, 
                    map { | process-line $_},  
                    "emma.txt".IO.lines; 
say @unknown-words.head(20);
\end{verbatim}

Testing such a program may take some time, because it has to 
process the full book each time. For the purpose of initial 
testing, one tip is to reduce the amount of input 
data to speed up the tests. You may prepare a smaller file with 
just a limited number of lines from the original {\tt emma.txt} 
file. Another simple way is to read only some lines from the 
book, which you can do with a slice on the code line which 
reads the file. For example, to read only the first 2,000~lines 
of the book file:

\begin{verbatim}
my @unknown-words = unique grep {$_ ∉ $word-list}, 
                    grep { $_ }, 
                    map { | process-line $_},  
                    ("emma.txt".IO.lines)[0..1999]; 
\end{verbatim}

This can also be used to get rid of the header. Since the 
actual text of the book starts on line 254, we can have:

\begin{verbatim}
my @unknown-words = unique grep {$_ ∉ $word-list}, 
                    grep { $_ }, 
                    map { | process-line $_},  
                    ("emma.txt".IO.lines)[253..1999]; 
\end{verbatim}
and remove from {\tt process-line} the code to skip the header.

\subsection{Exercise in Section~\ref{randomwords}: Random words (p.~\pageref{randhist})}
\label{sol_randhist}
\index{histogram!random choice}

We have made a {\tt BisectSearch} module containing a {\tt bisect} 
subroutine. It would be great to reuse it, but we can't really 
as it is: it is doing string comparisons and we need numerical  comparisons.
 
The best is probably to make a copy of the subroutine and modify it 
to make numeric comparisons. The subroutines can have the same 
name provided they are declared as multi subroutines and have 
a different signature: the first parameter of the new multi 
subroutine should be an {\tt Int} instead of a {\tt Str}. Since 
the changes to be made are quite small and easy, this is 
left as an exercise to the reader.

The program using that module might look like this:

\begin{verbatim}
use lib ".";
use BisectSearch;
my %histogram;

sub process-line(Str $line is copy) {
    $line ~~ s:g/<[-']>/ /; 
    $line ~~ s:g/<[;:,!?.()"_`]>//; 
    $line = $line.lc; 
    return $line.words;
}
%histogram{$_}++ for grep {$_},
                     map { | process-line $_}, 
                     ("emma.txt".IO.lines)[253..*]; 
my (@words, @freqs);
my $total_freq = 0;
for %histogram.kv -> $word, $freq {
    $total_freq += $freq;
    push @words, $word;
    push @freqs, $total_freq;
}
my $rand_int = $total_freq.rand.Int;
my $idx = bisect $rand_int, @freqs;
say @words[$idx];
\end{verbatim}

\subsection{Exercise in Section~\ref{markov}: Markov analysis (p.~\pageref{markov_analysis})}
\label{sol_markov_analysis}
\index{Markov analysis}

Before we present our solution to the exercise,, we want 
to briefly introduce a functionality that is useful for 
retrieving and validating the command-line arguments passed 
to a program, the {\tt MAIN} subroutine.

\subsubsection{The {\tt MAIN} subroutine}
\label{MAIN_sub}
\index{MAIN}
\index{program!argument}
\index{argument!to the program}
\index{command-line argument}
\index{argument!command-line}

The arguments passed to a program are usually stored in the 
\verb'@*ARGS' special array. You can just browse the items of this 
array to retrieve the arguments. The following one-liner is 
an example of this:
\index{one-liner mode}

\begin{verbatim}
$ perl6 -e 'say $_ for reverse @*ARGS' one two three
three
two
one
\end{verbatim}

There is however another way to do it, the {\tt MAIN} 
subroutine that we have briefly seen in Section~\ref{MAIN} 
(p.~\pageref{MAIN}). If there is a subroutine 
called {\tt MAIN} in the program, then the program will start 
by executing this subroutine, whose parameters will be the 
arguments passed to the program. This means that the signature 
of the {\tt MAIN} subroutine will make it possible to retrieve the 
parameters and check their validity.

In our example solution below, the {\tt MAIN} subroutine is declared 
as follows:

\begin{verbatim}
sub MAIN (Str $book, Int $word-count, Int $order = 2, 
            Int $start-line = 0) {
    # body of subroutine here
}
\end{verbatim}

The program will thus check that the arguments passed to it match 
the {\tt MAIN} subroutine signature. In the example, the first 
parameter has to be a string and the second one an integer; the 
third and fourth parameters are optional and will be defaulted 
respectively to 2 and 0 if the corresponding arguments are not 
provided.

If the arguments passed to the program don't match the {\tt MAIN} 
signature, the program will die after having printed an 
automatically-generated usage message:

\begin{verbatim}
$ perl6  markov.pl6 emma.txt 100 2 foo
Usage:
  markov.pl6 <book> <word-count> [<order>] [<start-line>]
\end{verbatim}
The \verb'$start-line' parameter has to be an integer. Since 
the corresponding argument (``foo'') is not an integer, the 
program displays a message showing the program usage.

Validating the command-line arguments passed to a program 
can sometimes be a relatively tedious task. With this 
{\tt MAIN} subroutine signature mechanism, it can often be reduced 
to a single line of code, the {\tt MAIN} signature.

\subsubsection{Solution to the Markov analysis exercise}
\index{Markov analysis}

This is a possible way to perform a Markov analysis of a 
text file:

\begin{verbatim}
my %prefixes;

sub MAIN (Str $book, Int $word-count, Int $order = 2, 
              Int $start-line = 0) {
    process-line($order, $_) for ($book.IO.lines)[$start-line..*]; 
    say make-text($order, $word-count);
}

sub process-line($order, Str $line is copy) {
    $line ~~ s:g/<[-']>/ /; 
    $line ~~ s:g/<[;:,!?.()"_`]>//; # removing punctuation symbols
    $line = $line.lc;               # setting string to lower case
    return unless $line ~~ /\w/;
    process-words($order, $line.words);
}

sub process-words ($order, @new-words) {
    state @word-buffer = ();
    push @word-buffer, |@new-words;
    while (@word-buffer.elems >= $order * 2) {
        my $key = @word-buffer.shift ~ " " ~ 
             (join ' ', @word-buffer[0..$order - 2]);
        my $value = @word-buffer[$order -1];
        push %prefixes{$key}, $value;
    }
}

sub make-text (Int $order, Int $w-count) {
    my @prefix = %prefixes.keys.pick.words;
    my $count = 0;
    my $text = join " ", @prefix;
    while $count <= $w-count {
        my @possible-suffixes = |%prefixes{join " ", @prefix};
        last unless @possible-suffixes;
        my $new-word = |@possible-suffixes.pick;
        $text ~= " $new-word";
        shift @prefix;
        push @prefix, |$new-word;
        $count++
    }
    return $text;
}     
\end{verbatim}

This program may be called on the ``emma.txt'' file with 
the following syntax:

\begin{verbatim}
$ perl6  markov.pl6 emma.txt 100 2 253
\end{verbatim}

\subsection{Exercises on the Huffman Code in Section~\ref{huffman_exercise} (p.~\pageref{huffman_exercise})}

\subsubsection{The frequency table (Section~\ref{letter_frequency})}
\label{sol_letter_frequency}
\index{frequency table}

We have already seen similar problems, there is nothing complicated. 
This is a possible solution using the pipeline programming model 
described in Section~\ref{most_common_words}:
\index{pipe-line programming}

\begin{verbatim}
my %frequencies;
%frequencies{$_}++ for grep {/<[a..z]>/}, map {.lc}, 
    "goldbug.txt".IO.lines.comb;
my $total_count = [+] values %frequencies;
say "$_ :\t%frequencies{$_} \t", 
    sprintf "%5.2f", %frequencies{$_}*100/$total_count  
    for reverse sort {%frequencies{$_}}, %frequencies.keys;
\end{verbatim}   

which displays:
\begin{verbatim}
e :     7625    13.10
t :     5485     9.42
a :     4477     7.69
o :     4208     7.23
i :     4183     7.18
n :     3912     6.72
s :     3516     6.04
h :     3372     5.79
r :     3278     5.63
d :     2533     4.35
l :     2324     3.99
u :     1893     3.25
c :     1523     2.62
m :     1499     2.57
f :     1392     2.39
w :     1303     2.24
p :     1169     2.01
y :     1146     1.97
g :     1143     1.96
b :     1031     1.77
v :     525      0.90
k :     351      0.60
x :     120      0.21
j :     111      0.19
q :     60       0.10
z :     44       0.08
\end{verbatim}

Remember that Edgar Poe's character claimed the succession 
of the most commonly used letters in English ran as 
follows:
\index{Poe, Edgar Allen}

\begin{verbatim}
e a o i d h n r s t u y c f g l m w b k p q x z
\end{verbatim}

So it appears that Poe's character was approximately 
right, but certainly not very accurate, in his estimates of the 
letter frequencies in an English text. Especially, it appears 
that he grossly underestimated the frequency of the ``t'' letter. 
Running the same program against the text of Jane Austen's 
``Emma'' novel that we have used previously produces very 
close results:

\begin{verbatim}
e :     87029   12.57
t :     60035    8.67
a :     54884    7.93
o :     53877    7.78
n :     47773    6.90
i :     47172    6.82
s :     42920    6.20
h :     42819    6.19
r :     41453    5.99
d :     28870    4.17
l :     27971    4.04
(...)
\end{verbatim}

\subsubsection{Huffman coding of a DNA strand (Section~\ref{huffman_code_2})}
\label{sol_huffman_code_2}
\index{Huffman!code}

At each step in the algorithm, we need to look for 
the two letters with the lowest frequencies. Rather than 
having to repeatedly go through all the items in the 
frequency hash (or to sort the values each time), we 
will use a data structure maintaining the values 
sorted according to our needs.

We start with the \verb'%frequencies' hash built in the 
previous exercise and transform it into a sorted collection 
of pairs mapping each letter to its frequency. We create 
a {\tt insert-pair} subroutine which adds the newly created 
pairs (the dummy letters) at the right place in the pair array 
to keep the array sorted according to our needs.
\index{array of pairs}
\index{array!of pairs}

\begin{verbatim}
my %code;
my @pairs;
push @pairs, $_ => %frequencies{$_} for 
    sort {%frequencies{$_}}, %frequencies.keys;

sub insert-pair (@list, $new-elem) {
    my $val = $new-elem.value;
    for @list.keys -> $i {
        if @list[$i].value >= $val {
            splice @list, $i, 0, $new-elem;
            return;
        }
    }
    push @list, $new-elem; # putting the new element at the end of 
                           # the list if right place not found earlier
}
\end{verbatim}

We loop over the pairs, pick up the two with the smallest 
frequencies, merge them into a new pair and add it at the right 
place with the {\tt insert-pair} subroutine. The loop ends when 
there are only two pairs left. At the same time, we populate at 
each step of the loop new \verb'%code' hash with the partial codes 
found.

\begin{verbatim}
loop {
    my $least1 = shift @pairs;
    my $least2 = shift @pairs;
    my $new-pair = $least1.key ~ $least2.key => $least1.value + $least2.value;
    insert-pair @pairs, $new-pair;
    %code{$least1.key} =  $least1.key ~ $least2.key ~ "|.";
    %code{$least2.key} =  $least1.key ~ $least2.key ~ "|-";
    last if @pairs <= 2;
}
%code{@pairs[0].key} = ".";
%code{@pairs[1].key} = "-";
\end{verbatim}

At the end of the loop, the pair array contains two pairs:

\begin{verbatim}
[c => 10 tga => 11]
\end{verbatim}

and the \verb'%code' hash contains the partial codes for each 
letter or dummy letter:

\begin{verbatim}
{a => ga|-, c => ., g => ga|., ga => tga|-, t => tga|., tga => -}
\end{verbatim}

We then use another loop to substitute the pseudo letters and get 
rid of them, until we are left with only the actual letters of the 
original input string:

\begin{verbatim}
loop {   
    my $done = True;
    for %code.keys -> $letter {
        next if $letter.chars > 1;
        my ($val, $code) = split '|', %code{$letter};
        next unless defined $val and defined $code;
        $done = False;
        my $result = %code{$val} ~ $code;
        %code{$letter} = $result;
    }
    last if $done;
}
my %encode;
%encode{$_} = %code{$_} for grep {$_.chars < 2 }, %code.keys;
\end{verbatim}

The \verb'%encode' hash contains the Huffman table:
\index{Huffman!table}

\begin{verbatim}
c => .
t => -.
g => --.
a => ---
\end{verbatim}

\subsubsection{Huffman coding of a more complex string (Section~\ref{huffman_code_2})}

For this question, we will use a small paragraph specially written 
to contain only a few letters of the alphabet:

\begin{quote}
``Eastern Tennessee anteaters ensnare and eat red ants, detest ant
antennae (a tart taste) and dread Antarean anteater-eaters. Rare
Andean deer eat tender sea reeds, aster seeds and rats’ ears. Dessert?
Rats’ asses.''
\end{quote}

As a first step, we will fold all letters to lower-case and use 
only the letters, eliminating spaces and punctuation symbols from 
the computation of the frequency table:

\begin{verbatim}
my $string = "Eastern Tennessee anteaters ensnare and eat red ants, detest ant
antennae (a tart taste) and dread Antarean anteater-eaters. Rare
Andean deer eat tender sea reeds, aster seeds and rats’ ears. Dessert?
Rats’ asses."; 

my %frequencies;
%frequencies{$_}++ for grep { /\w/ }, $string.lc.comb;
\end{verbatim}

This eloquent treatise on the eating habits of various fauna 
yields the following frequency table:

\begin{verbatim}
e :     40      23.53
a :     32      18.82
t :     24      14.12
s :     22      12.94
n :     20      11.76
r :     19      11.18
d :     13      7.65
\end{verbatim}

Using the same code as in the previous question generates the 
following Huffman table:
\index{Huffman!table}

\begin{verbatim}
a => ..
e => .-
s => -.-
n => -..
t => --.
d => ---.
r => ----
\end{verbatim}

\subsubsection{Encoding the input string (Section~\ref{huffman_code_2})}
\index{Huffman!encoding}

We want not only to encode an input string with the Huffman code, but 
we also want to then be able to decode it and to recognize the original 
input. Because of that, we no longer want to filter out punctuation signs 
from the translation table, which will therefore grow much larger than 
before. Spaces (both horizontal spaces and line returns) will be handled 
differently: we'll keep them unchanged in the encoded pseudo-Morse 
string, as this will make it easier to check and to display the result.

The frequency table now includes punctuation characters which exist 
in the input string:

\begin{verbatim}
%frequencies{$_}++ for grep {/<[\w] + [.,()’?-]>/}, $string.lc.comb;
\end{verbatim}

The frequency table now has 14 entries:

\begin{verbatim}
e :     40      22.10
a :     32      17.68
t :     24      13.26
s :     22      12.15
n :     20      11.05
r :     19      10.50
d :     13       7.18
. :     3        1.66
, :     2        1.10
’ :     2        1.10
) :     1        0.55
- :     1        0.55
? :     1        0.55
( :     1        0.55
\end{verbatim}

And the Huffman table (\verb'%encode' hash) now looks like this:
\index{Huffman!table}

\begin{verbatim}
e => .-
a => ---
s => -..
n => ..-
t => --.
r => ...
d => -.--
. => -.-.-.
( => -.-..-.
’ => -.-.--.
? => -.-..--
) => -.-...-
- => -.-....
, => -.-.---
\end{verbatim}

The encoding subroutine is very simple:

\begin{verbatim}
sub encoding (Str $input, %encode) {
    my $output;
    for $input.lc.comb -> $letter {
        $output ~= %encode{$letter} // $letter;
    }
    return $output;
}
\end{verbatim}

Each letter of the input is converted to lower case (since we have 
limited our table to lower case), translated into its pseudo-Morse 
code equivalent and concatenated to the output string. If a letter
is not found in the \verb'%encode' hash, then it is stored into 
the output as it is: this makes it possible to insert the spaces 
and end-of-line characters into the output string.

The result is as follows (slightly reformatted to fit in this book):
\begin{verbatim}
.-----..--..-.....- --..-..-..-.--..-...-.- ---..---..------..-...-..
.-..--....----....- ---..--.-- .------. ....--.-- ---..---.-..-.-.---
-.--.---..--..--. ---..---.
---..---..-..-..----.- -.-..-.--- --.---...--. --.----..--..--.-...- 
---..--.-- -.--....-----.-- ---..---.---....----..-
---..---..------..-...-.-.....------..-...-..-.-.-. ...---....-
---..--.--.----..- -.--.-.-... .------. --..-..--.--.-... -...----
 ....-.--.---..-.-.--- ----..--..-... -...-.--.---.. ---..--.-- 
 ...-----.-..-.-.--. .----...-..-.-.-. -.--.--..-...-...--.-.-..--
...-----.-..-.-.--. ----..-...--..-.-.-.
\end{verbatim}

Interestingly, the input string has 213~characters and the output 
string has 589. If we were storing the 14~different characters of 
the input with equal-length codes, we would need four bits per 
character, which would require 1052~bits. So Huffman coding 
achieved a compression ratio 1.78 times better than the best possible 
equal-length codes. And the ASCII encoding of the input string 
required 213~bytes, i.e. 1704~bits; the Huffman-encoded output 
required almost 3 times less. 

\subsubsection{Decoding the pseudo-Morse string (Section~\ref{huffman_code_2})}
\index{Huffman!decoding}

For decoding efficiently the pseudo-Morse string, we need to reverse 
the Huffman table, i.e. create a hash in which the pseudo-Morse codes 
are the keys and the letters the values. Reversing the \verb'%encode' 
hash is straight forward:

\begin{verbatim}
my %decode = reverse %encode.kv;
\end{verbatim}

\index{reverse}
The \verb'%encode.kv' expression produces a list of keys and values, 
and the reverse statement transforms it into a list of values and 
keys. Assigning that list to a new hash produces a new hash in 
which keys and values are swapped. Note that this works because 
we know that the values are unique, so that there is no problem 
of duplicates when promoting them to hash keys.

Decoding the pseudo-Morse string is a bit more complicated than 
its encoding, because we don't know in advance how many dots and 
dashes will be needed to obtain a letter. So we need to look at the 
first character (say a dot) of the pseudo-Morse string; if it is alone an 
entry in the translation table, then we have found our first 
letter, and we can start afresh with the next character as a starting 
point of a new letter; if not, we need to pick up the next character 
and see if the two first characters together form an entry; if yes, 
we have found a letter and can start from the beginning again; if 
not we need to see if the first three characters together form an 
entry in the table, and so on.

For example, with the beginning of the pseudo-Morse string:
\begin{verbatim}
.-----..--..-.....-
\end{verbatim}
the first dot is not an entry but the ``.-'' combination is a ``e''.
The next dash is not an entry, ``\verb"--"'' also not, but 
``\verb"---"'' is a ``a''.
The next dash is not an entry, ``-.'' also not, but ``-..'' is a ``s''. 
Similarly, the next three characters, ``\verb"--."'', form a ``t'', and we 
can go on to decode the word ``eastern''.

We might implement that with two nested loops: one to go through the 
string and the second one to consume the necessary number of dots and 
dashes until the end of a letter:

\begin{verbatim}
sub decoding (Str $input, %decode) {
    my @codes = $input.comb;
    my $output;
    loop {
        last unless @codes;
        my $current = shift @codes;
        $output ~= $current and next if $current ~~ /\s/;
        $output ~= %decode{$current} and next if %decode{$current}:exists;
        loop {           # we need more characters to complete a letter
            $current ~= shift @codes;
            if %decode{$current}:exists {
                $output ~= %decode{$current};
                last;    # we're done with a letter, go back to main loop
            }
        }
    }
    return $output;
 }
\end{verbatim}

This works properly and the output is the same as the original 
input (except for the fact that we have folded everything to 
lower-case):

\begin{verbatim}
eastern tennessee anteaters ensnare and eat red ants, detest ant
antennae (a tart taste) and dread antarean anteater-eaters. rare
andean deer eat tender sea reeds, aster seeds and rats’ ears. dessert?
rats’ asses.
\end{verbatim}

However, if you think about it, we don't really need two 
nested loops in the {\tt decoding} subroutine, which can be 
made a bit more concise as follows:

\begin{verbatim}
sub decoding (Str $input, %decode) {
    my ($output, $current);
    for $input.comb -> $in-code {
        $output ~= $in-code and next if $in-code ~~ /\s/;
        $current ~= $in-code;
        if %decode{$current}:exists {
            $output ~= %decode{$current};
            $current = "";
        }
    }
    return $output;
}
\end{verbatim}

Here, the \verb'$current' variable accumulates the dots and 
dashes from the input until it is found to be an entry in 
the translation table, at which point it is reset to an 
empty string to prepare for the next letter.

The solution presented above for finding the Huffman code 
uses the {\tt insert-pair} subroutine to keep ordered the 
\verb'@pairs' array of pairs. This makes it easy to find the remaining 
least common letters or pseudo-letters. You might remember 
from Section~\ref{heap} that heaps are a good data structure 
when the aim is to access rapidly to the smallest items 
of a collection. As a further exercise, you may want to 
rewrite the solution using a binary heap. David Huffman's original 
solution actually used a tree (called the Huffman tree) very 
similar to a heap.
\index{heap} 
\index{huffman!code}
\index{Huffman!tree}


\section{Exercises of Chapter~\ref{regex_grammars}: Regexes and grammars}

\subsection{Exercise in Section~\ref{february_rule}: getting the February dates right (p.~\pageref{february_rule})}
\label{sol_february_rule}
\index{February, number of days}
\index{grammar!date}
\index{leap year}

We want to check if February dates are valid.

To begin with, let's exclude February dates that are 
larger than 29. This can be done by just expanding the 
code assertion shown in the code to recognize dates.

\begin{verbatim}
my $string = "Leap day : 2016-02-29.";                                         
my token year { \d ** 4 }                                        
my token month {   
    1 <[0..2]>                            # 10 to 12                     
    || 0 <[1..9]>                         # 01 to 09                     
};
my token day { (\d ** 2) <?{1 <= $0 <= 31 }> }  
my token sep { '/' || '-' }                                                 
my rule date { [   <year> (<sep>) <month> $0 <day> 
                 || <day> (<sep>) <month> $0 <year> 
                 || <month>\s<day>',' <year>
                ] <!{ ($<day> > 30 and $<month> ==  4|6|9|11) or 
                       $<day> > 29 and $<month> eq '02' }>
}                         

if $string ~~ /<date>/ {
    say ~$/;                              # 2016-02-29
    say "Day\t= "   , ~$/<date><day>;     # 29
    say "Month\t= " , ~$/<date><month>;   # 02
    say "Year\t= "  , ~$/<date><year>;    # 2016
}                    
\end{verbatim}

This is fine, February has 29 days since 2016 is a leap 
year. But this code would validate Feb. 29 for 2015 or 
2017, which is wrong since they are not leap years.

\subsubsection{Recognizing a leap year}
\index{leap year}

\index{Julian calendar}
In the old Julian calendar (named after Julius Caesar), 
leap years are years which are divisible by 4. It turned 
out that the Julian calendar had too many leap years 
to reflect the astronomical reality, so that the 
calendar drifted about 3 days for every period of four 
centuries.
\index{Caesar, Julius}

\index{Gregorian calendar}
The Gregorian calendar, introduced by Pope Gregory~XIII 
in 1582, corrected the Julian calendar with 
the following additional rule: years divisible by 100 should 
be leap only if they are divisible by 400. So that, by the 
Gregorian calendar, 1700, 1800, 1900 and 2100 are not leap, 
but 2000 and 2400 are leap.
\index{Gregory XIII, Pope}

Depending on what kind of dates your program is going to 
encounter, you might decide to simplify the rules. If 
you are writing a module supposed to be accurate for any 
date far in the past or in the future, you probably want 
to implement the exact Gregorian rule. But if you know 
that you're going to meet only dates of the current 
period, you might choose a much simpler rule.

\index{Y2K bug}
In particular, since 2000 is an exception to the exception 
and is leap, any year between 1901 and 2099 is leap if it 
is divisible by 4 and not leap otherwise. This rule is 
most probably sufficient for any business application 
written in 2017. There is probably no reason to make it 
more complicated than it needs to be (although it may be 
argued that it is the same type of reasoning which led to 
the great fear of the ``Y2K'' bug). 

With this simplification in mind, a subroutine to find out 
if a year is leap should simply return true if it is 
divisible by 4 and might thus look like this:

\begin{verbatim}
sub is-leap ($year) { # works for years between 1901 and 2099
    return True if $year %% 4; 
    return False;
}
\end{verbatim}

Or simpler:

\begin{verbatim}
sub is-leap ($year) { # works for years between 1901 and 2099
    return $year %% 4; 
}
\end{verbatim}

If you want to implement the full Gregorian rule, it might 
look like this:

\begin{verbatim}
sub is-leap ($year) { # Gregorian rule for any year
    return False if $year % 4;   # no if not divisible by 4
    return True if $year % 100;  # yes if divisible by 4 and not by 100
    return False if $year % 400; # no if divisible by 100 and not by 400
    True;                        # yes if divisible by 400
}
\end{verbatim}

or, if you like concision (or obfuscation):
\begin{verbatim}
sub is-leap ($y) { $y %% 400 or ($y %% 4 and not $y %% 100) }
\end{verbatim}

The code above is given as an example on how to compute 
if a year is leap, since it is an interesting and 
classical problem, but Perl actually provides the method for that 
in the {\tt Dateish} role. For example:
\index{is-leap-year function}

\begin{verbatim}
> say Dateish.is-leap-year(2016)
True
> say Dateish.is-leap-year(2015)
False
\end{verbatim}

\subsubsection{Back to the February date validation}

You \emph{can} add the rules for Feb.~29 in the code example 
above if you wish, but we would suggest this is getting 
slightly too complicated for a code assertion within 
the {\tt date} rule: adding a quick boolean condition 
in a code assertion within a rule is fine, but when 
the condition becomes more complicated, it will tend 
to make the rule more difficult to understand. Think 
about the person who will have to maintain the code in 
a year from now (and that person might be you).
 
We prefer to move this out of the {\tt date} rule into 
a dedicated subroutine checking all dates for February:

\begin{verbatim}
sub feb-date-not-valid ($year, $day) {
    return False if $day <= 28;
    return True if $day > 29;
    return False if Dateish.is-leap-year($year);
    True;
}
\end{verbatim}

The {\tt date} rule now looks like this:

\begin{verbatim}
my rule date { [   <year> (<sep>) <month> $0 <day> 
                 || <day> (<sep>) <month> $0 <year> 
                 || <month>\s<day>',' <year>
                ] <!{ ($<day> > 30 and $<month> ==  4|6|9|11) or 
                      $<month> eq '02' and feb-date-not-valid $<year>, $<day>}>
} 
\end{verbatim}

I had originally called the new subroutine {\tt check-feb-29} 
but I changed it to {\tt feb-date-not-valid} in order to 
better show that it returns a true value if the date 
is not valid. This may seem secondary, but choosing good 
names for your identifier is important because that 
self-documents your programs and clarify their semantics.

Once we've introduced this minimal subroutine, we 
might go one step further and move the rest of the 
code assertion into the subroutine, so that the final code 
assertion would contain only a call to the new version 
of the subroutine. This is left as a further exercise 
to the reader.

\subsection{Exercise~\ref{calculator} (p.~\pageref{calculator}): 
A grammar for an arithmetic calculator}
\label{sol_calculator}
\index{calculator}
\index{arithmetic calculator}
\index{grammar!arithmetic calculator}
\index{calculator!grammar}

\subsubsection{The grammar}

Here's one way to write the grammar:
\index{grammar}

\begin{verbatim}
my grammar Calculator {
    rule TOP            { <expr> }
    rule expr           { <term> + % <plus-minus-op> }
    token plus-minus-op { [< + - >] }
    rule term           { <atom> + % <mult-div-op> }
    token mult-div-op   { [< * / >] }
    rule atom {
        | <num> { make +$<num> }
        | <paren-expr> { make $<paren-expr>.made}
    }
    rule num            { <sign> ? [\d+ | \d+\.\d+ | \.\d+ ] }
    rule paren-expr     { '(' <expr> ')' }
    token sign          { [< + - >] }
}
\end{verbatim}

This is quite simple.

An expression (\verb'expr') is made of one or several terms 
separated by ``+'' or ``-'' operators. A term is made of one 
or several atoms separated ``*'' or ``/'' operators. An atom 
may be a bare number or a parenthesized expression.

\index{precedence}
This guarantees that precedence rules are satisfied. 
Multiplications and divisions will be 
evaluated before additions and subtractions, since, when 
parsing an expression, you need to evaluate the individual 
terms before you can complete the expression evaluation. 
Similarly, since a parenthesized expression is an atom, it 
will have to be evaluated before the evaluation of the term 
in which it appears can be fully evaluated. Note that, in 
the case of parenthesized expression, the \verb'expr' rule 
is called recursively.
\index{recursion}

\subsubsection{The actions}
\index{actions!class}

Notice that we have included two actions in the grammar (in 
the {\tt atom} rule). One reason was for convenience: since 
the {\tt atom} rule covers two very different named 
sub-rules, it is a bit easier to include the action just 
in the context of the sub-rules. If an action had been attached 
to the  {\tt atom} rule, it would have required to find out 
which sub-rule had been matched to know which action to 
perform. Nothing difficult, but it would have made the  
code slightly more complex. The other reason was for pedagogical 
purpose: although it often makes sense to create an actions 
class, it is useful to know that actions may be inserted in the 
grammar part. For a very simple grammar, it might be 
over-engineering to create an actions class with just one or 
two actions.
\index{grammar}

The actions class might look like this:

\begin{verbatim}
class CalcActions {
    method TOP ($/) {
        make $<expr>.made
    }
    method expr ($/) {
        $.calculate($/, $<term>, $<plus-minus-op>)
    }
    method term ($/) {
        $.calculate($/, $<atom>, $<mult-div-op>)
    }
    method paren-expr ($/) {
         make $<expr>.made;
    }
    method calculate ($/, $operands, $operators) {
        my $result = (shift $operands).made;
        while my $op = shift $operators {
            my $num = (shift $operands).made;
            given $op {
                when '+' { $result += $num; }
                when '-' { $result -= $num; }
                when '*' { $result *= $num; }
                when '/' { $result /= $num; }
                default  { die "unknown operator "}
            }
        }
        make $result;
    }
}
\end{verbatim}

The {\tt calculate} method computes expressions (terms 
separated by addition or subtraction operators) and terms 
(atoms separated by multiplication or division operators) 
from left to right, since these operators are left associative.

This grammar for a calculator and its associated actions class 
may be tested with the following code:
\index{grammar!arithmetic calculator}

\begin{verbatim}
for |< 3*4 5/6 3+5 74-32 5+7/3 5*3*2 (4*5) (3*2)+5 4+3-1/5 4+(3-1)/4 >,
    "12 + 6 * 5", " 7 + 12 + 23", " 2 + (10 * 4) ", "3 * (7 + 7)" { 
    my $result = Calculator.parse($_, :actions(CalcActions));
    # say $result;
    printf "%-15s %.3f\n", $/,  $result.made if $result;
}
\end{verbatim}

which will display the following results:

\begin{verbatim}
3*4             12.000
5/6             0.833
3+5             8.000
74-32           42.000
5+7/3           7.333
5*3*2           30.000
(4*5)           20.000
(3*2)+5         11.000
4+3-1/5         6.800
4+(3-1)/4       4.500
12 + 6 * 5      42.000
 7 + 12 + 23    42.000
 2 + (10 * 4)   42.000
3 * (7 + 7)     42.000
\end{verbatim}

You might wonder if this code works correctly with nested 
parenthesized expression. I originally thought, when I wrote this 
code, that it might dysfunction and that I might have to change 
something to get nested parenthesized expressions right and 
properly balanced. It turns out that it works fine out of the 
box. For example, with the following test code with relatively 
deeply nested parenthesized expressions:
\index{nested expressions}

\begin{verbatim}
for "(((2+3)*(5-2))-1)*3", "2 * ((4-1)*((3*7) - (5+2)))"  { 
    my $result = Calculator.parse($_, :actions(CalcActions));
    printf "%-30s %.3f\n", $/,  $result.made if $result;
}
\end{verbatim}

The result is correct:
\begin{verbatim}
(((2+3)*(5-2))-1)*3            42.000
2 * ((4-1)*((3*7) - (5+2)))    84.000
\end{verbatim}

\index{exponentiation}
\index{precedence}
As an additional exercise, you might want to add exponentiation 
to the list of allowed operators. Remember that exponentiation has 
higher precedence than multiplication and division (so you probably 
want to put it somewhere near the atom level). In the event that 
you want to handle nested exponentiation operators, also 
remember that they are usually right associative:

\begin{verbatim}
2**3**2 = 2**(3**2) = 2 ** 9 = 512;  # Not: (2**3)**2 or 64
\end{verbatim}

\section{Exercises of Chapter~\ref{functional programming}: Functional programming}

\subsection{Exercise~\ref{quicksort}: Making a functional implementation of quick sort)}
\label{sol_quicksort}
\index{quicksort}
\index{sort!quick sort}

Here's one way to implement the quick sort algorithm in 
functional programming style.

\begin{verbatim}
sub quicksort (@input) {
    return @input if @input.elems <= 1;
    my $pivot = @input[@input.elems div 2];
    return flat quicksort(grep {$_ < $pivot}, @input), 
        (grep {$_ == $pivot}, @input), 
        quicksort(grep {$_ > $pivot}, @input);
}
\end{verbatim}

This functional version of the program reflects directly 
the approach of the quick sort algorithm:
\begin{itemize}
\item If the  array has less than two items, it is already 
sorted, so return it immediately (this is the base case 
stopping the recursion).
\item Else, choose an item as a pivot (here, we pick the 
middle element or one immediately near the middle).
\item Partition the array into three sublists containing 
items respectively smaller than, greater than and equal to 
the pivot.
\item Sort the first two sublists by a recursive call to 
the \verb'quicksort' function (but don't call \verb'quicksort' 
on the sublist containing items equal to the pivot: not only is 
it already sorted, but it would fail to meet the base 
case and would enter into infinite recursion).
\item  Return the list obtained by concatenating the three 
sub-lists.
\end{itemize}
\index{sort!quick sort}

\index{pivot!quick sort algorithm}
\index{divide and conquer algorithm}
In principle, you could choose any item as the pivot, including 
the first or the last item of the array. But for 
some specific input (such as arrays already almost sorted, 
forward or backward), this can increase significantly the 
run time because the partitioning becomes totally unbalanced, 
thereby losing the advantage of the divide and conquer strategy. 
Picking an element in the middle strongly reduces the 
probability of such pathological behavior. Another possible option 
to prevent such risk is to select the pivot at random among 
the array elements.


