

\chapter{Arrays and lists}
\label{arrays}

This chapter presents one of Perl's most useful built-in types, 
arrays.


\section{Lists and arrays are sequences}
\label{sequence}

Like strings, {\bf list} and {\bf arrays} are sequences of 
values.  In a string, the values are characters; in a list or
in an array, they can be any type.  The values in a list or 
in an array are called {\bf elements} or sometimes {\bf items}.
\index{list}
\index{array}
\index{type!list}
\index{type!array}
\index{element}
\index{sequence}
\index{item}

There are several important differences between lists and arrays. The main ones 
are that lists are ordered and immutable collections of items: 
you can't change the number of items in a list and you can't 
change the individual items either. Arrays, by contrast, are 
variables and are generally mutable: you can add elements 
to an array, or remove elements from it. And you can access 
the individual elements of an array and modify them. For this 
to be possible, arrays usually have a name (as other variables), 
although some arrays are anonymous, which means that they 
have no name, but have some other ways or accessing them.

A list is also an ephemeral thing (unless it is assigned 
to a variable or some other thing): it ceases to exist as 
soon as it has been used, usually as soon as the program 
control flow goes to the next code line. An array, on the 
other hand, has some form of persistence: you may be able to 
use it somewhere else in the program if the variable 
containing it is still within scope.

There are several ways to create a new list; the simplest is 
to enumerate its values, separated by commas:
\index{comma operator}
\index{operator!comma}

\begin{verbatim}
> 3, 4, 5
(3 4 5)
> say (3, 4, 5).WHAT;
(List)
say $_ for 1, 2, 3;
1
2
3
\end{verbatim}
%

You don't need the parentheses to create a list, but they are 
often useful to delimit it, i.e. to stipulate where 
it starts and where it ends, and, in some cases, to override 
precedence.

We have been using lists earlier in this book. If we write:

\begin{verbatim}
> print "$_ " for 1, 3, 5, 9, "\n";
1 3 5 9
 >
> print "$_ " for 1..10;
1 2 3 4 5 6 7 8 9 10 >
\end{verbatim}

we are basically creating and using a list of integers (from 
the standpoint of the type hierarchy of Perl, this is 
not entirely accurate technically for the second example, 
\verb'1..10' has a \emph{Range} type, and it gets 
transformed into a \emph{Seq} type, but this approximation is 
good enough for our purpose here).
\index{range!type}
\index{range!operator}

Arrays are variables whose names start with the sigil \verb'@'.
Named arrays  need to be declared before they are used, just 
as any other variable we've seen so far. One of the easiest 
ways to create an array is to assign a list to it:
\index{sigil}

\begin{verbatim}
> my @odd_digits = 1, 3, 5, 7, 9;
[1 3 5 7 9]
> say @odd_digits.WHAT;
(Array)
> my @single_digit_numbers = 0..9;
[0 1 2 3 4 5 6 7 8 9]
\end{verbatim}

Under the Perl REPL, an array is displayed between square 
brackets (\verb"[" and \verb"]"), while lists were displayed 
between round parentheses.
\index{REPL}
\index{square bracket operator}
\index{operator!square bracket}

If the items don't contain any space characters, it is quite handy 
to construct a list (and assign it to an array if needed) using 
the \verb'<...>' quote-word operator:
\index{quote-word operator}

\begin{verbatim}
> my @weekdays = <mon tue wed thu fri>;
[mon tue wed thu fri]
> my @weekend = <sat sun>;
[sat sun]
\end{verbatim}

The advantage of this construction is that there is no need to 
separate the items with commas and no need to put them between 
quotes when the items are strings. Basically, 
the quote-word operator breaks up its content on whitespace 
and returns a list of words, which can then be 
used in a loop or assigned to an array as in the example 
above.

Most of the rest of this chapter will be devoted to arrays 
rather than lists, but keep in mind that many of the array 
functions and operators we will study here also work on lists 
(at least most of those that would not violate the 
immutability property of lists).
 
The items of an array (or a list) don't need to be of the 
same type:

\begin{verbatim}
> my @heterogeneous-array = 1, 2.3, pi, "str", (1, 2, 4);
[1 2.3 3.14159265358979 str (1 2 4)]
\end{verbatim}

Here, the array is composed of an integer, a rational, a 
float ({\tt Num} type), a string and a list of three integers. 
It may not be advisable for the sake of the developer's 
mental sanity to use an array with such wildly heterogeneous 
items, but Perl will not complain about that: it is up 
to you to make sense of your data.

The array above contains even a list of items. If you iterate 
over the elements of this array for example with a {\tt for} 
loop statement, this list will come as one distinct element, 
it will not get ``flattened'' as three elements 
of the array. Similarly, {\tt elems} is a method to 
count the number of items of an array (or of a list). 
Using it on the above array produces the following result:

\begin{verbatim}
> say @heterogeneous-array.elems;
5
\end{verbatim}
%

As it can be seen, the \verb'(1, 2, 4)' list ``counts'' 
as one single array element.

A list within another list is {\bf nested}.
\index{nested list}
\index{list!nested}

An array that contains no elements is
called an empty array; you can create one with empty
perentheses, \verb"()":
\begin{verbatim}
> my @empty = ();
[]
\end{verbatim}
\index{empty list}
\index{list!empty}

But this is rarely useful, since just declaring an array 
without defining it has the same effect:

\begin{verbatim}
> my @empty;
[]
\end{verbatim}

So using the empty brackets would be needed essentially for 
resetting an existing array to an empty array.


\section{Arrays are mutable}
\label{mutable}
\index{list!element}
\index{access}
\index{index}
\index{subscript}
\index{bracket operator}
\index{operator!bracket}

\index{index!starting at zero}
The syntax for accessing the elements of an array or a list 
uses the square brackets operator.  The expression inside the 
brackets specifies the index or subscript, which can be a 
literal integer (or some value that can be coerced into 
an integer), a variable containing a numerical value, a list 
or a range of numerical values, a piece of code returning a 
numerical value, etc.  Indices are offsets compared to the beginning of the array 
or the list (much in the same way as the values returned 
by the {\tt index} function on strings), so that they start 
at 0. Thus, the first item of an array has the index 0, the 
second item the index 1, and so on:
\index{zero, index starting at}


\begin{verbatim}
say <sat sun>[1];             # -> sun  (accessing a list item)
my @weekdays = <mon tue wed thu fri>;     # assigning an array
say "The third day is @weekdays[2]";      # -> The third day is wed
\end{verbatim}
%

You may also use ranges or lists of indices to access to
\emph{slices} of an array or a list:
\index{slice}
\index{slice!operator}
\index{operator!slice}
\index{index!slice}
\index{list!slice}
\index{slice!list}

\begin{verbatim}
> my @even-digits = 0, 2, 4, 6, 8;
[0 2 4 6 8]
> my @small-even_digits = @even-digits[0..2];
[0 2 4]
> my @min-max-even-digits = @even-digits[0, 4]
[0 8]
\end{verbatim}

If you need a slice in the opposite order, you can use the 
{\tt reverse} function to reverse the range:

\begin{verbatim}
> my @reverse-small-even_digits = @even-digits[reverse 0..2];
[4 2 0]
\end{verbatim}

or reverse the data returned by the slice expression:

\begin{verbatim}
> my @reverse-small-even_digits = reverse @even-digits[0..2];
[4 2 0]
\end{verbatim}

Unlike lists, arrays are mutable.  When the bracket operator 
appears after an array on the left side of an assignment, 
it identifies the element of the array that will be assigned.
\index{mutability}

\begin{verbatim}
> my @even-digits = 0, 2, 2, 6, 8;   # Oops, error on the second 2
[0 2 2 6 8]
> @even-digits[2] = 4; # fixing the faulty third digit
4
> say @even-digits
[0 2 4 6 8]
\end{verbatim}
%

The third element of {\tt \@even-digits}, which was  
(presumably by mistake) 2 is now 4. If the index corresponds 
to an item which does not exist yet in the array, the array 
will be expanded to include the new element:

\begin{verbatim}
> my @odds = 1, 3, 5;
[1 3 5]
> @odds[3] = 7;
7
> say @odds;
[1 3 5 7]
\end{verbatim}

\index{index!starting at zero}
\index{item assignment}
\index{assignment!item}
\index{reassignment}

\index{elems function or method}
\index{end}
The {\tt elems} function or method returns the number of 
elements of an array. The {\tt end} function or method 
returns the index of the last elements of an array:

\begin{verbatim}
my @nums = 1..5;      # -> [1 2 3 4 5]
say @nums.elems;      # -> 5
say elems @nums;      # -> 5
say @nums.end;        # -> 4
\end{verbatim}

The {\tt end} method returns the result of the {\tt elems} 
method minus one because, since indices start at 0, the 
index of the last element is one less than the number 
of elements.
\index{zero, index starting at}

The {\tt unique} function or method returns a sequence 
of unique elements of the input list or array (i.e. 
it returns the original list without any duplicate 
values):
\index{unique function}
\index{function!unique}

\begin{verbatim}
> say < a b d c a f d g>.unique;
(a b d c f g)
\end{verbatim}

If you know that the input is sorted (and that, therefore, 
duplicates are adjacent), use the {\tt squish} function 
instead of {\tt unique}, as this is likely to be more 
efficient.
\index{squish function}
\index{function!squish}

To know whether two arrays are identical (structurally the 
same, same type and same values), use the {\tt eqv} equivalence 
operator. To know if they just contain the same elements, use 
the \verb'~~' smart match operator. Between two arrays or lists, the 
\verb'==' numeric equality operator will return {\tt True} if the 
arrays have the same number of elements and {\tt False} otherwise, 
because \verb'==' coerces its arguments to Numeric type, so that 
it compares the number of elements.


\begin{verbatim}
> my @even1 = 0, 2, 4, 6, 8;
[0 2 4 6 8]
> my @even2 = reverse 8, 6, 4, 2, 0;
[0 2 4 6 8]
> say @even1 eqv @even2           # same items, structurally the same
True
> say <1 2 3 4 5> eqv 1..5;       # same items, structurally different
False
> say <1 2 3 4 5> ~~ 1..5;        # same items, True
True
> my @array = 1..5;               
[1 2 3 4 5]
>  say <1 2 3 4 5> ~~ @array;     # same elements, True
True
>  say <1 2 3 4 6> ~~ @array;     # not the same elements
False
> say <1 2 3 4 5> == <5 6 7 8 9>; # compares the numbers of items
True
\end{verbatim}

The \verb'<1 2 3 4 5> eqv 1..5' statement returns False because, 
although they have the same items, the arguments are structurally 
different entities (one is a list and the other one a range).

\section{Adding new elements to an array or removing some}

We've just seen that assigning an item to an index that does 
not exists yet will expand the array. There are other 
ways of expanding an array.

Perl has operators to add elements to, or remove one element 
from, an array:
\index{pop function}
\index{push function}
\index{shift function}
\index{unshift function}

\begin{itemize}
\item {\tt shift}: removes the first item of an array and returns it;
\item {\tt pop}: removes the last item of an array and returns it;
\item {\tt unshift}: adds an item or a list of items at the 
beginning of an array;
\item {\tt push}: adds an item or a list of items at the 
end of an array;
\end{itemize}

These are a few examples for each:

\begin{verbatim}
> my @numbers = <2 4 6 7>;
[2 4 6 7]
> push @numbers, 8, 9;
[2 4 6 7 8 9]
> unshift @numbers, 0, 1;
[0 1 2 4 6 7 8 9]
> my $num = shift @numbers
0
> $num = pop @numbers
9
> say @numbers
[1 2 4 6 7 8]
\end{verbatim}

As you might expect by now, these routines also come with a 
method invocation syntax. For example:

\begin{verbatim}
> my @numbers = <2 4 6 7>;
[2 4 6 7]
> @numbers.push(8, 9)
[2 4 6 7 8 9]
\end{verbatim}

Note, however, that if you {\tt push} or {\tt unshift} an 
array onto another array, you'll get something different 
than what you might expect:

\begin{verbatim}
> my @numbers = <2 4 6 7>;
[2 4 6 7]
> my @add-array = 8, 10;
[8 10]
> @numbers.push(@add-array);
[2 4 6 7 [8 10]]
\end{verbatim}

As you can see, \verb'@add-array' is added as an entity 
to the \verb'@numbers' array, \verb'@add-array' becomes 
the new last item of the original array. If you want to 
add the items of \verb'@add-array' to the original array, 
you may use the {\tt append} method instead of {\tt push}:

\begin{verbatim}
> my @numbers = <2 4 6 7>;
[2 4 6 7]
> @numbers.append(@add-array);
[2 4 6 7 8 10]
\end{verbatim}

Or you can use the ``|'' prefix operator to flatten the 
added array into a list of arguments:

\begin{verbatim}
> my @numbers = <2 4 6 7>;
[2 4 6 7]
> @numbers.push(|@add-array);
[2 4 6 7 8 10]
\end{verbatim}

There is also a {\tt prepend} method which can replace 
{\tt unshift} to add individual items of an array at the 
beginning of an existing array.


\section{Stacks and queues}
\label{stacks_queues}

\index{stack}
\index{queue}
Stack and queues are very commonly used data structures in 
computer science.

\index{LIFO}
\index{last in / first out}
A stack is a \emph{last in / first out (LIFO)} data structure. Think of 
piled-up plates. When you put a clean plate onto the stack, 
you usually put it on top; when you take out one, you also 
take it from the top. So the first plate that you take 
is the last one that you've put. A CS stack implements the same 
idea: you use it when the first piece of data you need from a 
data structure is the last one you've put there.

\index{FIFO}
\index{first in / first out}
A queue, by contrast, is a \emph{first in / first out (FIFO)} data 
structure. This is the idea of persons standing in a 
line waiting to pay at the supermarket. The first 
person that will be served is the first person who entered 
the queue.

A stack may be implemented with an array and the {\tt push} 
and {\tt pop} functions, which respectively add an item (or 
several) at the end of an array and take one from the end of 
the array. This is a somewhat simplistic implementation of 
a stack:
\index{pop function}
\index{push function}
\index{function!pop}
\index{function!push}
\label{stack_code}

\begin{verbatim}
sub put-in-stack (@stack, $new_item) {
    push @stack, $new_item;
}
sub take-from-stack (@stack) {
    my $item = pop @stack;
    return $item;
}
my @a-stack = 1, 2, 3, 4, 5;
put-in-stack @a-stack, 6;
say @a-stack;
say take-from-stack @a-stack for 1..3;
\end{verbatim}

This example will print this:

\begin{verbatim}
[1 2 3 4 5 6]
6
5
4
\end{verbatim}

\index{stack}
This stack is simplistic because, at the very least, a more 
robust implementation should do something sensible when you 
try to {\tt take-from-stack} from an empty stack. It would 
also be wise to add signatures to the subroutines. In 
addition, you might want to {\tt put-in-stack} more than 
one element in one go. Take a look at the solution to the 
exercise on queues below (\ref{sol_exercise_queue}) to 
figure out on how this stack may be improved.

You could obtain the same stack features using the  
{\tt unshift} and {\tt shift} functions instead of {\tt push} 
and {\tt pop}. The items will be added at the beginning of 
the array and taken from the beginning, but you will still 
have the LIFO behavior.
\index{shift function}
\index{unshift function}
\index{push function}
\index{LIFO}

\label{exercise_queue}
As an exercise, try to implement a FIFO queue on the same model. 
Hint: you probably want to use an array and the {\tt unshift} 
and {\tt pop} functions (or the {\tt push} and {\tt shift} functions). Solution: \ref{sol_exercise_queue}.
\index{queue}
\index{shift function}
\index{unshift function}
\index{push function}
\index{pop}
\index{FIFO}

\section{Other ways to modify an array}
\label{modify_array}

The {\tt shift} and {\tt pop} functions remove respectively 
the first and the last item of an array and return that 
item. It is possible to do almost the same operation on any item 
of an array, using the {\tt delete} \emph{adverb}:

\begin{verbatim}
my @fruit = <apple banana pear cherry pineapple orange>;
my $removed = @fruit[2]:delete; 
say $removed;  # -> pear
say @fruit;    # -> [apple banana (Any) cherry pineapple orange]
\end{verbatim}

Notice that the third element (``pear'') is removed and 
returned, but the array is not reorganized, the operation leaves 
a sort of ``empty slot'', an undefined item, in the 
middle of the array. The colon ``:'' syntax used here is the 
operator for an adverb (we have seen other adverbs in  
section~\ref{regex} about regexes in the chapter on strings); 
for the time being, you may think of it as a kind of special 
method operating on one element of an item collection.

We have seen how to use array slices to retrieve several 
items of an array or a list at a time. The same slice syntax 
can also be used on the left side of an assignment to modify 
some elements of an array:
\index{slice!assignment}

\begin{verbatim}
my @digits = <1 2 3 6 5 4 7 8 9>
@digits[2..4] = 4, 5, 6
say @digits;   # -> [1 2 4 5 6 4 7 8 9]
\end{verbatim}

Of course, you can't do that with lists, since, as you 
remember, they are immutable.

The {\tt splice} function may be regarded as the Swiss Army 
knife of arrays. It can add, remove and return one or 
several items to or from an array. The general syntax is 
as follows:

\begin{verbatim}
my @out_array = splice @array, $start, $num_elems, @replacement;
\end{verbatim}
%
The arguments for {\tt splice} are the input array, the index 
of the first element on which to make changes, the number of 
elements to be affected by the operation, and a list of 
replacements for the elements to be removed. For example, 
to perform the slice assignment shown just above, it is 
possible to do this:

\begin{verbatim}
my @digits = <1 2 3 6 5 4 7 8 9>
my @removed_digits = splice @digits, 3, 3, 4, 5, 6;
say @removed_digits;     # -> [6 5 4]
say @digits;             # -> [1 2 4 5 6 7 8 9]
\end{verbatim}
%
Here, the {\tt splice} statement removed 3 elements (6, 5, 4) 
and replaced them with the replacement arguments (4, 5, 6). 
And it returned the removed items into \verb'@removed_digits'. 
The number of replacements needs not be the same as the number 
of removed items, in which case the array size will grow or 
shrink. For example, if no replacement is provided, then 
{\tt splice} will just remove and return the required number 
of elements and the array size will be reduced by the same number:

\begin{verbatim}
my @digits = 1..9;
my @removed_digits = splice @digits, 3, 2;
say @removed_digits;     # -> [4 5]
say @digits;             # -> [1 2 3 6 7 8 9]
\end{verbatim}
%

Conversely, if the number of elements to be removed is zero, 
no element will be removed, an empty array will be returned, 
and the elements in the replacement list will be added in 
the right place:

\begin{verbatim}
my @digits = <1 2 3 6 4 7 8 9>;
my @removed_digits = splice @digits, 3, 0, 42;
say @removed_digits;     # -> []
say @digits;             # -> [1 2 3 42 6 4 7 8 9]
\end{verbatim}
%

Assuming the {\tt shift} function did not exist in Perl, 
you could write a {\tt my-shift} subroutine to simulate it:

\begin{verbatim}
sub my-shift (@array) {
    my @result = splice @array, 0, 1;
    return @result[0];
}
my @letters = 'a'..'j';
my $letter = my-shift @letters;
say $letter;             # -> a
say @letters;            # -> [b c d e f g h i j]
\end{verbatim}

We might raise an exception if the array passed to 
{\tt my-shift} is empty. This could be done by modifying 
the subroutine as follows:

\begin{verbatim}
sub my-shift (@array) {
    die "Cannot my-shift from an empty array" unless @array;
    my @result = splice @array, 0, 1;
    return @result[0];
}
\end{verbatim}
%

or by adding a non-empty constraint on the array in the 
subroutine signature:
\begin{verbatim}
sub my-shift (@array where @array > 0) {
    my @result = splice @array, 0, 1;
    return @result[0];
}    
\end{verbatim}
%

The \verb'@array > 0' expression evaluates to {\tt True} if 
the number of elements of the array is more than 0, i.e. if the 
array is not empty. It is equivalent to \verb'@array.elems > 0'.

\label{splice_exercise}
As an exercise, write subroutines to simulate the {\tt pop}, 
{\tt unshift}, {\tt push} and {\tt delete} built-ins. Solution: \ref{sol_splice_exercise}.


\section{Traversing a list}
\index{list!traversal}
\index{traversal!list}
\index{for loop}
\index{loop!for}
\index{statement!for}

The most common way to traverse the elements of a list or an 
array is with a {\tt for} loop.  The syntax for an array is 
the same as what we have already seen in earlier chapters 
for lists:

\begin{verbatim}
my @colors = <red orange yellow green blue indigo violet>;
for @colors -> $color {
    say $color;
}
\end{verbatim}
%
This works well if you only need to read the elements of the
list.  But if you want to write or update the elements of an array, you
need a doubly-pointed block. For example, you might use the 
{\tt tc} (title case) function to capitalize the first letter 
of each word of the array:
\index{tc function}

\begin{verbatim}
my @colors = <red orange yellow green blue indigo violet>;
for @colors <-> $color {$color = tc $color};
say @colors;   # -> [Red Orange Yellow Green Blue Indigo Violet]
\end{verbatim}
%
Here the \verb'$color' loop variable is a read-and-write alias 
on the array's items, so that changes made to this alias will 
be reflected in the array. This works well with arrays, but 
would not work with lists, which are immutable. You would 
get an error with a list:

\begin{verbatim}
> for <red orange yellow> <-> $color { $color = tc $color}
Parameter '$color' expected a writable container, but got Str value...
\end{verbatim}

You may also use the syntax of a {\tt for} loop with the 
\verb'$_' topical variable. For example, this uses the
{\tt uc} (upper case) function to capitalize each word of 
the previous array:
\index{topical variable}
\index{uc function or method}

\begin{verbatim}
for @colors { 
    $_ = $_.uc 
}
say @colors; # -> [RED ORANGE YELLOW GREEN BLUE INDIGO VIOLET]
\end{verbatim}
%

Sometimes, you want to traverse an array and need to know the 
index of the elements you are visiting. A common way to do 
that is to use the \verb'..' range operator to iterate on 
the indices. For instance, to print the index and the value of each element of an array:

\begin{verbatim}
for 0..@colors.end -> $idx { 
    say "$idx  @colors[$idx]"; 
}
\end{verbatim}

This is useful, for example, for traversing two (or more) arrays 
in parallel:

\begin{verbatim}
my @letters = 'a'..'e';
my @numbers = 1..5;
for 0..@letters.end -> $idx { 
    say "@letters[$idx] -> @numbers[$idx]"; 
}
\end{verbatim}
%

which will print:
\begin{verbatim}
a -> 1
b -> 2
c -> 3
d -> 4
e -> 5
\end{verbatim}

You don't really need to specify the index range yourself, as 
the {\tt keys} function will return a list of indices for 
the array or the list:
\index{keys function or method}

\begin{verbatim}
for keys @colors -> $idx { 
    say "$idx  @colors[$idx]"; 
}
\end{verbatim}

Another way to iterate over the indices and values of an 
array is the {\tt kv} (``keys values'') function or method, 
which returns the index and value of each array item:
\index{kv function or method}

\begin{verbatim}
for @letters.kv -> $idx, $val { 
    say "$idx $val";
}
\end{verbatim}

In list context, the \verb'@letters.kv' simply returns an 
interleaved sequence of indexes and values:

\begin{verbatim}
my @letters = 'a'..'e';
say @letters.kv;    # -> (0 a 1 b 2 c 3 d 4 e)
\end{verbatim}

It is the pointy block with two iteration variables 
which makes it possible to process both an index and a value at each step of the loop. You can of course have more than two iteration variables if needed.


\section{New looping constructs}
\index{loop}

Since the subject of this chapter is arrays and lists, 
it is probably the right time to briefly study two 
looping constructs that we had left aside so far.

The first one uses the same {\tt for} keyword as above, 
but with a different syntax for the iteration variable(s).
\index{for loop}

\begin{verbatim}
my @letters = 'a'..'e';
for @letters { 
    say $^a-letter; 
}
\end{verbatim}

\index{placeholder}
\index{placeholder parameter}
\index{twigil}
\index{self-declared parameter}
The \verb'^' in the \verb'$^a-letter' variable is called a \emph{twigil}, i.e. sort of a second sigil. This twigil 
states that the \verb'$^a-letter' variable is a  \emph{placeholder parameter} or a \emph{self-declared 
positional parameter}. This is a positional parameter 
of the current block that needs not be declared in 
the signature.

If the block uses more than one placeholder, they are 
associated to the input according to their lexicographic 
(alphabetic) order:

\begin{verbatim}
my @letters = 'a'..'e';
for @letters.kv { 
    say $^a ~ " -> " ~ $^b; 
}
\end{verbatim}
%
which will print:
\begin{verbatim}
0 -> a
1 -> b
2 -> c
3 -> d
4 -> e
\end{verbatim}
%

As seen just above, the {\tt kv} function returns an 
interleaved sequence of indexes and values. Since \verb'$^a' comes before \verb'$^b' in the alphabetic 
order, \verb'$^a' will be bound to the index and \verb'$^b' with the value for each pair of the input.

Placeholders can also be used for subroutines:

\begin{verbatim}
> sub divide { $^first / $^second }
sub divide ($first, $second) { #`(Sub|230787048) ... }
> divide 6, 4
1.5
\end{verbatim}
%

These placeholders may not be used very often for 
simply traversing arrays, but we will see later how  
they are very useful in cases where is would be quite 
unpractical to have to declare the parameters.
\index{placeholder}

\index{loop!keyword}
\index{loop!statement}
\index{C-style loop}
\label{C-style loop}
The second new looping construct we want to introduce 
here uses the {\tt loop} keyword and is similar to 
the C-style {\tt for} loop (i.e. the loop of the 
C programming language). In this type of loop, you 
declare between a pair of parentheses three expressions 
separated by semi-colons: the iteration variable's 
initial value, the condition on which the loop should 
terminate and the change made to the iteration variable on 
each iteration:

\begin{verbatim}
loop (my $i = 0; $i < 5; $i++) {
    say $i, " -> " ~ @letters[$i];
}
\end{verbatim}
%
For most common loops, the {\tt for} loops seen earlier 
are easier to write and usually more efficient. This 
special {\tt loop} construct should probably be used 
only when the exit condition or the change made to 
the iteration variable are quite unusual and would 
be difficult to express in a regular {\tt for} loop. As 
an exception, the {\tt loop} construct with no 
three-part specification is quite common and even 
idiomatic for making an infinite loop:
\index{infinite loop}

\begin{verbatim}
loop {
    # do something 
    # last if ...
}
\end{verbatim}
%


\section{Map, filter and reduce}
\label{map_filter}

\subsection{Reducing a list to a value}

To add up all the numbers in a list, you can use a loop like this:


\begin{verbatim}
sub add_all (@numbers) {
    my $total = 0;
    for @numbers -> $x {
        $total += $x;
    }
    return $total;
\end{verbatim}
%
{\tt \$total} is initialized to 0.  Each time through the loop,
{\tt \$x} gets one element from the list and is added to 
{\tt \$total}. As the loop runs, {\tt total} accumulates the 
sum of the elements; a variable used this way is sometimes 
called an {\bf accumulator}.
\index{accumulator!sum}

An operation like this that combines a sequence of elements into
a single value is often called {\bf reduce} (or sometimes 
``fold'' in some other programming languages). These ideas 
are derived from some functional programming languages such 
as LISP.
\index{reduce pattern}
\index{pattern!reduce}
\index{traversal}

Perl~6 has a {\tt reduce} function, which generates a single 
"combined" value from a list of values, by repeatedly 
applying a function which knows how to combine two values. 
Using the {\tt reduce} function to compute the sum of the 
first ten numbers might look like this:

\begin{verbatim}
> my $sum = reduce { $^a + $^b }, 1..10;
55
\end{verbatim}

There is also a ``*'' so-called \emph{whatever} term, 
which makes it possible to write this:

\begin{verbatim}
my $sum = (1..10).reduce: * + *;   # -> 55
\end{verbatim}

but we'll leave it aside for the time being.

Remember the \emph{factorial} function of section~\ref{for_loops}? 
It used a {\tt for} loop to compute the product of the \emph{n} first 
integers up to a limit. It could be rewritten as follows using the 
{\tt reduce} function:
\index{factorial}
\index{factorial!using the reduce function}

\begin{verbatim}
sub factorial (Int $num) { 
    return reduce { $^a * $^b }, 1..$num;
}
say factorial 10;   # -> 3628800
\end{verbatim}
%
In fact, the code to compute the factorial is so short with 
the {\tt reduce} function that it may be argued that it has 
become unnecessary to write a subroutine for that. You could 
just ``inline'' the code:

\begin{verbatim}
my $fact10 = reduce { $^a * $^b }, 1..10;     # 3628800
\end{verbatim}
%

We can do many more powerful things with that, but we'll come 
back to that later, as it requires a few syntactic features that 
we haven't seen yet. 

Perl~6 also has a reduction operator, 
or rather a reduction \emph{metaoperator}. An operator 
usually works on variables or values; a metaoperator acts 
on other operators. Given a list and an operator, the 
\verb'[...]' meta-operator applies repeatedly the operator 
to all the values of the list to produce a single value.
\index{reduction operator}
\index{metaoperator}

For example, the following also prints the sum of all the 
elements of a list:

\begin{verbatim}
say [+] 1, 2, 3, 4;           # ->  10
\end{verbatim}

This basically takes the first two values, adds them up, and adds the result with the next value, and so on. Actually, 
there is a form of this operator, with a backslash before 
the operator, which also returns the intermediate results:

\begin{verbatim}
say [\+] 1, 2, 3, 4;          # -> (1 3 6 10)
\end{verbatim}

This meta-operator can be used to transform basically any 
associative infix operator into a list operator returning 
a single value.

The factorial function can now be rewritten as:
\index{factorial}
\index{factorial!using the reduction meta-operator}

\begin{verbatim}
sub fact(Int $x){
    [*] 1..$x; 
}
my $factorial = fact(10);     # -> 3628800
\end{verbatim}

The reduction metaoperator can also be used with relational operators 
to check if the elements of an array or a list are in the 
correct numerical or alphabetical order:

\begin{verbatim}
say [<] 3, 5, 7;              # -> True
say [<] 3, 5, 7, 6;           # -> False
say [lt] <a c d f r t y>;     # -> True
\end{verbatim}

\subsection{Mapping a list to another list}

Sometimes you want to traverse one list while building
another.  For example, the following function takes a list 
of strings and returns a new list that contains capitalized 
strings:

\begin{verbatim}
sub capitalize_all(@words):
    my @result;
    push @result, $_.uc for @words;
    return @result;
}
my @lc_words = <one two three>;
my @all_caps = capitalize_all(@lc_words); # -> [ONE TWO THREE]
\end{verbatim}
%
\verb'@result' is declared as an empty array; each time through
the loop, we add the next element.  So \verb'@result' is 
another kind of accumulator.
\index{accumulator!list}

An operation like \verb"capitalize_all" is sometimes called a {\bf map} because it ``maps'' a function (in this case the {\tt uc} method) onto each of the elements in a sequence.
\index{map pattern}
\index{pattern!map}
\index{filter pattern}
\index{pattern!filter}

Perl has a {\tt map} function which makes it possible to 
do that in just one statement:

\begin{verbatim}
my @lc_words = <one two three>;
my @all_caps = map { .uc }, @lc_words;      # -> [ONE TWO THREE]
\end{verbatim}
%

Here, the {\tt map} function applies the {\tt uc} method to 
each item of the \verb'@lc_words' array and returns them 
into the \verb'@all_caps' array. More precisely, the {\tt map} 
function assigns repeatedly each item of the \verb'@lc_words' 
array to the \verb'$_' topical variable, applies the 
code block following the {\tt map} keyword to \verb'$_' in 
order to create new values and returns a list of these new 
values.
\index{topical variable}

To generate a list of even numbers between 1 and 10, we might use the range operator to generate numbers between 1 and 5 
and use map to multiply them by two:

\begin{verbatim}
my @evens = map { $_ * 2 }, 1..5;  # -> [2 4 6 8 10]
\end{verbatim}
%

Instead of using the \verb'$_' topical variable, we might 
also use a pointy block syntax with an explicit iteration 
variable:

\begin{verbatim}
my @evens = map -> $num { $num * 2 }, 1..5;  # -> [2 4 6 8 10]
\end{verbatim}
%

or an anonymous block with a placeholder variable:

\begin{verbatim}
my @evens = map { $^num * 2 }, 1..5;  # -> [2 4 6 8 10]
\end{verbatim}
%

Instead of a code block, the first argument to {\tt map} can 
be a code reference (a subroutine reference):

\begin{verbatim}
sub double-sq-root-plus-one (Numeric $x) { 
    1 + 2 * sqrt $x;
}
my @results = map &double-sq-root-plus-one, 4, 16, 42;
say @results;     # -> [5 9 13.9614813968157]
\end{verbatim}
%

The subroutine name needs to be prefixed with the ampersand 
sigil to make clear that it is a parameter to {\tt map} and 
not a direct call of the subroutine.

If the name of the array on the left side and on the right 
side of the assignment is the same, then the modification 
seems to be made ``in place'', i.e. it appears as if the original array is modified in the process.


This is an immensely 
powerful and expressive function, we will come back to it 
later.

\subsection{Filtering the elements of a list}

Another common list operation is to select some elements from
a list and return a sublist.  For example, the following
function takes a list of strings and returns a list that 
contains only the strings containing a vowel:

\begin{verbatim}
sub contains-vowel(Str $string) {
    return True if $string ~~ /<[aeiouy]>/;
}
sub filter_words_with_vowels (@strings) {
    my @kept-string;
    for @string -> $st { 
        push @kept-string, $st if contains-vowel $st;
    }
    return @kept-string;
}  
\end{verbatim}
%

{\tt contains-vowel} is a subroutine that returns 
{\tt True} if the string contains at least one vowel 
(we consider ``y'' to be a vowel for our purpose).

The \verb"filter_words_with_vowels" subroutine will return 
a list of strings containing at least one vowel.

An operation like \verb"filter_words_with_vowels" is called 
a {\bf filter} because it selects some of the elements and 
filters out the others.
\index{filter}

Perl has a function called {\tt grep} to do that in just 
one statement:
\index{grep}

\begin{verbatim}
my @filtered = grep { /<[aeiouy]>/ }, @input;
\end{verbatim}
%

The name of the {\tt grep} built-in function used to 
filter some input comes from the Unix world, where it 
is a utility to filter from an input file the lines that 
match a given pattern.

In the code example above, all of \verb'@input' strings 
will be tested against the 
{\tt grep} block, and those matching the regex will go into the 
{\tt \@filtered} array. Just like {\tt map}, the {\tt grep} 
function assigns repeatedly each item of the {\tt \@input} 
array to the \verb'$_' topical variable, applies the 
code block following the {\tt grep} keyword to \verb'$_'  
returns a list of the values for which the code block 
evaluates to true. Here, the code block is a simple regex 
applied to the \verb'$_' variable.

Just as for {\tt map}, we could have used a function 
reference as the first argument to  {\tt grep}:

\begin{verbatim}
my @filtered = grep &contains-vowel, @input;
\end{verbatim}
%

To generate a list of even numbers between 1 and 10, we might 
use the range operator to generate numbers between 1 and 10 
and use {\tt grep} to filter out odd numbers:

\begin{verbatim}
my @evens = grep { $_ %% 2 }, 1..10;  # -> [2 4 6 8 10]
\end{verbatim}
%

\label{exercise_squares}
As an exercise, write a program using {\tt map} to 
produce an array containing the square of the numbers 
of the input list and a program using {\tt grep} to keep 
only the numbers of an input list that are perfect squares. Solution: \ref{sol_exercise_squares}.

Most common list operations can be expressed as a combination
of map, grep and reduce.

\subsection{Higher order functions and functional programming}
\label{array_functional_programming}

Besides their immediate usefulness, the {\tt reduce}, {\tt map} 
and {\tt grep} functions we have been using here do something 
qualitatively new. The arguments to these functions are not 
just data: their first argument is a code block or a function. 
We are not only passing to them the data that they will have 
to use or transform, we are also passing to them the code 
that will process the data.

The {\tt reduce}, {\tt map} and {\tt grep} functions are 
what is often called higher order functions, functions that 
manipulate not only data, but also other functions. These 
functions can be thought of as generic abstract functions: 
they perform a purely technical operation: process the  
elements of a list and apply to each of them a behavior 
defined in the code block or the function of the first 
parameter.

These ideas are to a large extent rooted in functional 
programming, a programming paradigm which is very different 
from what we had seen so far and which has been implemented 
historically in languages such as Lisp, Caml, Ocaml, Scheme, Erlang or Haskell. Perl~6 is not a functional 
programming language in the same sense as these languages, 
because it can also use other programming paradigms, 
but it has incorporated most of their useful features, so 
that you can use the expressive power and inherent safety 
of this programming model, without being forced to do so 
if and when you would prefer a different model, and
without having to learn a new and sometimes somewhat abstruse or clunky syntax.

This is immensely useful and can give you an incredible 
expressive power for solving certain types of problems. But 
other types of problems might be better solved with the more ``traditional'' procedural or imperative programming model, 
while others may benefit of an object-oriented approach. Perl~6 
lets you choose the programming model you want to use, and even 
makes it possible to combine seamlessly several of them in the 
same program.

Functional programming is so important that a full chapter 
of this book will be devoted to the functional programming 
features of Perl (see chapter~\ref{functional programming}). 
Before that, make sure to read the subsection~\ref{functional_queue} 
in the array and list section of the exercise solution chapter.

\section{Fixed-size, typed and shaped arrays}

By default, arrays can contain items of any type, including 
items of different types, and can auto-extend as you need. 
Perl will take care of the underlying gory details for you, 
so that you don't have to worry about them. This is very 
practical but this also comes with a cost: some array 
operations might be unexpectedly slow, because Perl may 
have to perform quite a bit of house-cleaning behind the 
scene, such as memory allocation or reallocation, copying 
a full array within memory, etc.

In some cases, however, it is possible to know beforehand 
the size of an array and the data type of its items. If 
Perl knows about these, it might be able to be faster and 
to use much less memory. It might also helps you 
to prevent subtle bugs.

To declare the type of the elements of an array, just 
specify it when declaring the array. For example, to 
declare an array of integers:

\begin{verbatim}
> my Int @numbers = 1..20;
[1 2 3 4 5 6 7 8 9 10 11 12 13 14 15 16 17 18 19 20]
> @numbers[7] = 3.5;     # ERROR
Type check failed in assignment to @numbers; expected Int but got Rat
  in block <unit> at <unknown file> line 1
\end{verbatim}
%

Similarly, you can declare the size of an array. There are 
twelve months in a year, so you might tell Perl that your 
\verb'@months' array will never contain more that twelve 
items:
\begin{verbatim}
> my @months[12] = 1..7;
[1 2 3 4 5 6 7 (Any) (Any) (Any) (Any) (Any)]
> say @months.elems
12
> say @months[3];
4
> say @months[12];
Index 12 for dimension 1 out of range (must be 0..11)
\end{verbatim}
%

Here, Perl has allocated 12 ``slots'' to the array, even though 
the last five are currently undefined. Perl may not need 
to reallocate memory when you define the tenth item of 
the array. And Perl tells you about your mistake if you 
accidentally try to access an out-of-range item.

Such arrays are often calles \emph{shaped} arrays.
\index{shaped array}

Defining both the type of the elements and the maximal size 
of the array may lead to a noticeable performance gain 
in terms of execution speed (at least for some operations) 
and reduce significantly the memory usage of the program, 
especially when handling large arrays.

\section{Multidimensional arrays}
\label{multidimensional array}
\index{multidimensional array}
\index{array!multidimensional}
\label{multidimensional_array}

Arrays we have seen so far are one-dimensional. In some 
languages, they are called vectors. Arrays can be 
multidimensional (you may then call them matrices).

For example, you might use a two-dimensional array to 
store a list of employees with their respective salary:

\begin{verbatim}
> my @employees;
[]
> @employees[0;0] = "Liz";
Liz
> @employees[0;1] = 3000;
3000
> @employees[1] = ["Bob"; 2500];
[Bob 2500]
> @employees[2] = ["Jack"; 2000];
[Jack 2000]
> @employees[3] = ["Betty"; 1800];
[Betty 1800]
> say @employees[1;1];
2500
> say @employees[2];
[Jack 2000]
> say @employees;
[[Liz 3000] [Bob 2500] [Jack 2000] [Betty 1800]]
\end{verbatim}

It is possible to have more than two dimensions. For example, 
we may have a \verb'@pixels' array in which the first two 
dimensions are the ``x'' and ``y'' coordinates of a dot in
a picture and the third dimension the RGB colors:

\begin{verbatim}
> my @pixels;
[]
> @pixels[0; 0] = [ <0 78 255> ];
[0 78 255]
> @pixels[1; 0] = [ <7 74 246> ];
[7 74 246]
> @pixels[0; 1] = [ 9, 55, 232 ];
[9 55 232]
> say @pixels[0;1];
[9 55 232]
> say @pixels[0;1;1]
55
> say @pixels;
[[[0 78 255] [9 55 232]] [[7 74 246]]]
\end{verbatim} 

Multidimensional arrays can also have a fixed size. For 
example, this may be a declaration for two-dimensional array 
where the first dimension is the month in the year and 
the second the day in the month:

\begin{verbatim}
my @date[12, 31];
\end{verbatim}


\section{Sorting arrays or lists}
\label{sorting}
\index{sorting!data}

Sorting data is a very common operation in computer 
science. Perl has a {\tt sort} function to sort an array 
or a list and return the sorted result:
\index{sort}
\index{sort!function or method}
\index{function!sort}

\begin{verbatim}
say sort <4 6 2 9 1 5 11>;  # -> (1 2 4 5 6 9 11)
\end{verbatim}

There are several types of sorts. The most common are numeric 
sort and lexicographic (or alphabetic) sort. They differ 
in the way they compare individual items to be sorted. 
\index{numeric sort}
\index{alphabetic sort}
\index{lexicographic sort}
\index{sort!numeric}
\index{sort!alphabetic}
\index{sort!lexicographic}

In alphabetic sort, you first compare the first letter of 
the words to be compared; a word starting with an ``a'' 
will always come before a word starting with a ``b'' 
(or any other letter) in an ascending sort, irrespective 
of the value or number of the other characters. You need to 
compare the second character of two words only if 
the first character of these words is the same. 

Numeric sorting is very different: it is the overall 
value of the number which is of interest. For example, 
if we are sorting integers, 11 is larger than 9 because 
it has more digits. But alphabetic sorting of 9 and 11 
would consider 11 to be smaller than 9, because the 
first digit is smaller.

So an alphabetic or lexicographic sort of the list of 
integers above would return:

\begin{verbatim}
(1 11 2 4 5 6 9)
\end{verbatim}

The consequence is that, with many programming languages, 
when you want to sort data, you need to specify which 
type of sort you want. With consistent data (every item 
of the same type), Perl~6 is usually clever enough to 
find out which type of sort is best suited to your 
needs. So that this will also perform the type of 
sort that you probably expect:

\begin{verbatim}
say sort <a ab abc ac bc cb ca>; # ->(a ab abc ac bc ca cb)
\end{verbatim}

Even with mixed data types, {\tt sort} can do a pretty good 
job at providing a result that may very well be what 
you are looking for:

\begin{verbatim}
say sort <1 11 5 4 12 a ab abc ac bc ca cb>;
         # -> (1 4 5 11 12 a ab abc ac bc ca cb)
\end{verbatim}

There are cases, however, where this simple use of the 
{\tt sort} function will fail to return what you 
probably want:

\begin{verbatim}
say sort <a ab abc A bc BAC AC>; # -> (A AC BAC a ab abc bc)
\end{verbatim}
%

Here, {\tt sort} puts all strings starting with an 
upper-case letter before any string starting with a 
lower case letter, probably not what you want. It 
looks even worse if the strings use extended 
ASCII characters:

\begin{verbatim}
say sort <a ab àb abc Ñ A bc BAC AC>;
        # -> (A AC BAC a ab abc bc Ñ àb)
\end{verbatim}
%

\index{sort!ASCIIbetical}
The reason is that, when sorting strings, {\tt sort} 
uses the internal numeric encoding of letters. This 
was sometimes called "ASCIIbetical" order (by contrast 
with alphabetical order), but the term is now too limited 
and somewhat obsolete, because in Perl~6 is using Unicode 
rather than ASCII. 

Clearly, these are cases where more advanced sorting 
techniques are needed. 

\section{More advanced sorting techniques}
\label{advanced_sort}
\index{sorting!advanced}

\index{sort}
\index{sort!code object}
\index{cmp operator}
\index{operator!cmp}
The {\tt sort} routine typically takes two arguments, 
a code object and a list of items to be sorted, and returns 
a new sorted list. If no code object is specified, as in the 
examples we have seen above, the {\tt cmp} built-in comparison 
operator is used to compare the elements. If 
a code object is provided (and if it accepts two 
arguments), then it is used to perform the comparison, 
which tells {\tt sort} which of the two elements should 
come first in the final order.

There are three built-in comparison operator that can be used 
for sorting. They are sometimes called three-way comparators 
because the compare their operands and return a value meaning 
that the first operand should be considered less than, equal to 
or more than the second operand for the purpose of determining 
in which order these operands should be sorted. The {\tt leg} 
operator coerces its arguments to strings and performs a 
lexicographic comparison. The \verb'<=>' operator coerces 
its arguments to numbers (Real) and does a numeric comparison. 
The afore-mentioned {\tt cmp} operator is the ``smart'' 
three-way comparator, which compares strings with string 
semantics and numbers with number semantics.

Most of our simple examples above worked well with 
strings and numbers because they implicitly used the 
default {\tt cmp} operator, which ``guesses'' quite 
well which type of comparison to perform.

\index{three-way comparator}
\index{leg operator}
\index{$<=>$ operator}
\index{cmp operator}
\index{operator!cmp}
\index{operator!leg}
\index{operator!$<=>$ (numeric comparison)}


In other words, this:

\begin{verbatim}
say sort <4 6 2 9 1 5 11>;  # -> (1 2 4 5 6 9 11)
\end{verbatim}

is equivalent to this:

\begin{verbatim}
say sort { $^a cmp $^b }, <4 6 2 9 1 5 11>;
     # -> (1 2 4 5 6 9 11)
\end{verbatim}

\index{placeholder}
The code block used here as the first argument to the 
{\tt sort} routine uses again the placeholder 
parameters (or self-declared parameters) seen earlier 
in this chapter. The {\tt cmp} routine receives two 
arguments which are bound to \verb'$^a' and \verb'$^b' 
and returns to the sort function information about which 
of the two items should come first in the resulting order.
\index{cmp operator}
\index{operator!cmp}

\index{sort!reverse order}
If you wanted to sort in reverse order, you could just 
swap the order of the two placeholder parameters:

\begin{verbatim}
say sort { $^b cmp $^a }, <4 6 2 9 1 5 11>;
     # -> (11 9 6 5 4 2 1)
\end{verbatim}

Note that this example is given only for the purpose of 
explaining some features of the placeholder parameters, 
it might be easier to obtain the same result with the 
following code:

\begin{verbatim}
say reverse sort <4 6 2 9 1 5 11>;  # -> (11 9 6 5 4 2 1)
\end{verbatim}
\index{reverse}

The reason {\tt sort} does a good job even with mixed 
strings and integers is because the default comparison 
function, {\tt cmp}, is pretty clever and guesses by 
looking at its arguments whether it should perform 
a lexicographic order or numeric order comparison.

If it gets too complicated for {\tt cmp}, or, more 
generally, when a specific or custom order is required, 
then you have to write your own ad-hoc comparison 
subroutine.
\index{cmp operator}
\index{operator!cmp}

For example, if we take again the example of strings 
with mixed case letters, we may achieve a case-insensitive 
alphabetical order this way:
\index{sort!case insensitive}

\begin{verbatim}
say sort { $^a.lc cmp $^b.lc}, <a ab abc A bc BAC AC>;
     # -> (a A ab abc AC BAC bc)
\end{verbatim}

or this way:
\begin{verbatim}
say sort { $^a.lc leg $^b.lc}, <a ab abc A bc BAC AC>;
     # -> (a A ab abc AC BAC bc)
\end{verbatim}

\index{lc function or method}
\index{function!lc}
\index{method!lc}
Here, when the comparison code block receives its two 
arguments, the {\tt lc} method casts them to lower-case 
before performing the comparison. Notice that this has 
no impact on the case of the output, since the lower-case 
transformation is local to the comparison code block and 
has no impact on the data handled by {\tt sort}. We will 
see shortly that there is a simpler and more efficient 
way of doing such a transformation before comparing the 
arguments.

If the comparison specification is more complicated, we 
may need to write it in a separated subroutine and let 
{\tt sort} call that subroutine. Suppose we have a list 
of strings which are all formed of leading digits 
followed by a group of letters and followed other 
irrelevant characters, and that we want to sort the 
strings according to the group of letters that follows 
the digits.

Let's start by writing the comparison subroutine:
\index{comparison subroutine}
\index{sort!comparison subroutine}


\begin{verbatim}
sub my_comp ($str1, $str2) {
    my $cmp1 = $0 if $str1 ~~ /\d+(\w+)/; 
    my $cmp2 = $0 if $str2 ~~ /\d+(\w+)/; 
    return $cmp1 cmp $cmp2;
}
\end{verbatim}

Nothing complicated: it takes two arguments, uses a regex 
for extracting the group of letters in each of the 
arguments, and returns the result of the {\tt cmp} 
function on the extracted strings. In the real world, something 
might need to be done if either of the extractions fails, but 
we will assume for our purpose here that this will not happen.
\index{regex}
\index{cmp operator}

The sorting is now quite straight forward, we just need to pass the above subroutine to the {\tt sort} function:

\begin{verbatim}
say sort &my_comp, < 22ac 34bd 56aa3 12c; 4abc( 1ca 45bc >;
     # -> (56aa3 4abc( 22ac 45bc 34bd 12c; 1ca)
\end{verbatim}

We only need to prefix the comparison subroutine with the 
``\&'' ampersand sigil and it works fine: the strings are sorted 
in accordance to the letter groups that follow the leading digits.
\index{sigil}
\index{ampersand sigil}

In all the examples above, the comparison subroutine 
accepted two parameters, the two items to be compared. 
The {\tt sort} function may also work with a code object 
taking only one parameter. In that case, the code object 
is not a comparison code block or subroutine, but is a 
code object implementing the transformation to be 
applied to the items before using the default {\tt cmp} 
comparison routine. 
\index{sort!transformation subroutine}
\index{cmp operator}
\index{operator!cmp}

For example, if we take once more the example of strings 
with mixed case letters, we may achieve a case-insensitive 
alphabetical order yet in a new way:
\index{sort!case insensitive}

\begin{verbatim}
say sort { $_.lc }, <a ab abc A bc BAC AC>;
     # -> (a A ab abc AC BAC bc)
\end{verbatim}

This could also be written with a placeholder parameter:
\index{placeholder}

\begin{verbatim}
say sort { $^a.lc }, <a ab abc A bc BAC AC>;
     # -> (a A ab abc AC BAC bc)
\end{verbatim}

Here, since the comparison code block takes only one 
argument, it is meant to transform each of the items 
before performing the standard {\tt cmp} routine on 
the arguments. This does not only make things simpler, 
but this is also probably more efficient, especially if 
the number of items to be sorted is large and if the 
transformation subroutine is relatively costly: the 
transformed values are actually \emph{cached} (i.e. 
stored in memory for repeated use), so that the 
transformation is done only once for each item, despite 
the fact that the comparison routine is called many times 
for each item in a sort.
\index{cache}

Similarly, we could sort numbers according to their 
absolute values:
\index{abs function or method}
\index{function!abs}
\index{method!abs}

\begin{verbatim}
say sort {$_.abs}, <4 -2 5 3 -12 42 8 -7>; # -> (-2 3 4 5 -7 8 -12 42)
\end{verbatim}

If you think about it, the ``more complicated'' example 
with digits and letters requiring a separate subroutine 
is also applying the same transformation to both its 
arguments. As an exercise, write a (simpler) sorting 
program using a transformation subroutine and the default 
{\tt cmp} operator on transformed items. Solution: \ref{sol_sort_exercise}.
\label{sort_exercise}
\index{cmp operator}
\index{sort}

Needless to say, the (so-called) advanced uses of the 
{\tt sort} function presented in this section are yet 
other examples of the functional programming style. The 
comparison subroutines and the transformation subroutines 
are passed around as arguments to the {\tt sort} 
function, and, more broadly, all of the functions, 
subroutines and code blocks used here are higher-order 
functions or first-class functions.
\index{functional programming}

\section{Debugging}
\index{debugging}

Careless use of arrays (and other mutable objects)
can lead to long hours of debugging.  Here are some common
pitfalls and ways to avoid them:

\begin{enumerate}

\item Some array built-in functions and methods modify 
their argument(s) and others don't.

It may be tempting to write code like this:

\begin{verbatim}
@array = splice @array, 1, 2, $new_item;   # WRONG!
\end{verbatim}
\index{splice function}

The {\tt splice} function returns the elements it has 
deleted from the array, not the array, which is modified 
``in place''.

Before using array methods and operators, you should 
read the documentation carefully and perhaps test them 
in interactive mode.

When traversing an array, for example with {\tt for} 
or {\tt map}, the \verb'$_' topical variable is an 
\emph{alias} for the successive items of the array, and 
not a copy of them. This means that if you change 
\verb'$_', the change will be reflected in the array. 
There may be some cases where this is what you want, 
and others where you don't care (if you no longer need the 
original array), but this is error-prone and should perhaps  
be avoided (or at least done only with great care).

\begin{verbatim}
my @numbers = <1 2 3>;
push @doubles, $_*=2 for @numbers;  # WRONG (probably)
say @numbers; # -> [2 4 6]
\end{verbatim}

The error here is that the \verb'$_*=2' statement is 
modifying \verb'$_', so that the \verb'@numbers' array is 
also modified, whereas the intent was certainly to 
populate the new numbers into \verb'@doubles', not 
to modify \verb'@numbers'.

The same code applied to a literal list instead of an 
array leads to a run time error because a list is 
immutable:

\begin{verbatim}
> push @doubles, $_*=2 for <1 2 3>; # WRONG (definitely)
Cannot assign to an immutable value
\end{verbatim}

The fix is quite easy in this case and consists in 
using an expression which does not modify \verb'$_' 
but returns the new desired value:

\begin{verbatim}
push @doubles, $_ * 2 for @numbers; # OK
\end{verbatim}

The same goes with {\tt map}:

\begin{verbatim}
my @numbers = <1 2 3>;
say map { ++$_}, @numbers;          # WRONG (probably)
say @numbers; # -> [2 3 4]
\end{verbatim}
%

Here again, using an expression not modifying \verb'$_' 
but returning the new desired value will fix the problem:

\begin{verbatim}
my @numbers = <1 2 3>;
say map { $_ + 1}, @numbers;        # -> (2 3 4)
say @numbers; # -> [1 2 3]
\end{verbatim}
%

\end{enumerate}



\section{Glossary}

\begin{description}

\item[list:] An immutable sequence of values.
\index{list}

\item[array:] A variable containing a mutable sequence 
of values.
\index{list}

\item[element:] One of the values in a list or an 
array (or some other sequence), also called items.
\index{element}
\index{item}

\item[nested array:] An array that is an element of another array.
\index{nested list}

\item[accumulator:] A variable used in a loop to add up or
accumulate a result.
\index{accumulator}

\item[augmented assignment:] A statement that updates the value
of a variable using an operator like \verb"+=".
\index{assignment!augmented}
\index{augmented assignment}
\index{traversal}

\item[reduce:] A processing pattern that traverses a sequence 
and accumulates the elements into a single result.
\index{reduce pattern}
\index{pattern!reduce}

\item[map:] A processing pattern that traverses a 
sequence and performs an operation on each element. 
Also the name of a Perl built-in function performing 
such a processing pattern.
\index{map pattern}
\index{pattern!map}

\item[filter:] A processing pattern that traverses a 
list and selects the elements that satisfy some criterion. 
{\tt grep} is a Perl implementation of a filter.
\index{filter pattern}
\index{pattern!filter}

\item[alias:] A circumstance where an identifier refers 
directly some variable or value, so that 
a change to it would lead to a change of 
the variable or value. It essentially means having 
two names for the same value, container or object.
\index{alias}

\end{description}


\section{Exercises}
\label{array_exercises}

\begin{exercise}

Write a subroutine called \verb"nested-sum" that takes an 
array of arrays of integers and adds up the elements from all of the nested arrays. For example:
\label{nested_sum}

\begin{verbatim}[fontshape=up]
my @AoA = [[1, 2], [3], [4, 5, 6]];
say nested-sum(@AoA);        # -> 21
\end{verbatim}

Solution: \ref{sol_nested_sum}.

\end{exercise}

\begin{exercise}
\label{cumulative}
\index{cumulative sum}

Write a subroutine called {\tt cumul-sum} that takes a list of numbers 
and returns the cumulative sum; that is, a new list where the 
$i$th element is the sum of the first $i+1$ elements from 
the original list. For example:
\label{cumsum}

\begin{verbatim}[fontshape=up]
my @nums = [1, 2, 3, 4];
say cumul-sum(@nums);           # -> [1, 3, 6, 10]
\end{verbatim}

Solution: \ref{sol_cumsum}.

\end{exercise}

\begin{exercise}

Write a subroutine called \verb"middle" that takes a list and
returns a new list that contains all but the first and last
elements.  For example:
\label{middle}

\begin{verbatim}[fontshape=up]
say middle(1, 2, 3, 4);      # -> (2, 3)
\end{verbatim}

Solution: \ref{sol_middle}.

\end{exercise}

\begin{exercise}

Write a subroutine called \verb"chop-it" that takes 
an array, modifies it by removing the first and last 
elements, and returns nothing useful. For example:
\label{chop}

\begin{verbatim}[fontshape=up]
my @nums = 1, 2, 3, 4;
chop-it(@nums);
say @nums;                   # -> [2, 3]
\end{verbatim}

Solution: \ref{sol_chop}.

\end{exercise}


\begin{exercise}
Write a subroutine called \verb"is-sorted" that takes 
a list (or array) of numbers as a parameter and returns 
{\tt True} if the list is sorted in ascending order and 
{\tt False} otherwise.  For example:
\label{is_sorted}
\index{is-sorted}

\begin{verbatim}[fontshape=up]
> is-sorted (1, 2, 2);
True
> is-sorted (1, 2, 1);
False
\end{verbatim}

Solution: \ref{sol_is_sorted}.

\end{exercise}


\begin{exercise}

\label{is_anagram}
\index{anagram}
\index{is-anagram}
Two words are anagrams if you can rearrange the letters 
from one to spell the other.  Write a subroutine called 
\verb"is-anagram" that takes two strings and returns 
{\tt True} if they are anagrams.

Solution: \ref{sol_is_anagram}.

\end{exercise}



\begin{exercise}
\label{has_duplicates}
\index{duplicate}
\index{has-duplicates}
\index{uniqueness}

Write a subroutine called \verb"has-duplicates" that takes
a list or an array and returns {\tt True} if there is 
any element that appears more than once.  It should 
not modify the original input.

Solution: \ref{sol_has_duplicates}.

\end{exercise}


\begin{exercise}

This exercise pertains to the so-called Birthday Paradox, 
which you can read about at 
\url{http://en.wikipedia.org/wiki/Birthday_paradox}.
\index{birthday paradox}
\label{birthdays}

If there are 23 students in your class, what are the 
chances that two of you have the same birthday?  You 
can estimate this probability by generating random 
samples of 23 birthdays and checking for duplicates. 
Hint: you can generate random birthdays
with the {\tt rand} and the {\tt int} functions.
\index{rand function}
\index{function!rand}

Solution: \ref{sol_birthdays}.

\end{exercise}



\begin{exercise}

\label{push_unshift}
Write a subroutine that reads the file {\tt words.txt} and builds
a list with one element per word.  Write two versions of
this function, one using the {\tt push} method and the
other using the idiom {\tt unshift}.  Which one takes
longer to run?  Why?
\index{push function}
\index{unshift function}

Solution: \ref{sol_push_unshift}.


\end{exercise}


\begin{exercise}
\label{bisection}
\index{membership!bisection search}
\index{bisection search}
\index{search!bisection}
\index{membership!binary search}
\index{binary search}
\index{search, binary}
\index{half-interval search}

To check whether a word is in the word list, you could check each element in turn, but it would be slow because it searches through the words in order.

If the words are in alphabetical order, we can speed 
things up considerably with a bisection search (also 
known as binary search), which is similar to what you 
do when you look a word up in the dictionary.  You
start somewhere in the middle and check to see whether 
the word you are looking for comes before the word in 
the middle of the list.  If so, you search the first 
half of the list the same way.  Otherwise, you search
the second half.

Either way, you cut the remaining search space in half. 
If the word list has 113,809 words, it will take at 
most about 17 steps to find the word or conclude that 
it's not there.

Write a function called \verb"bisect" that takes a 
sorted list and a target value and returns information 
about whether the target value is in the list or not.
\index{bisect}

Solution: \ref{sol_bisection}

\end{exercise}

\begin{exercise}
\index{reverse word pair}
\label{reverse_pair}

Two words are a ``reverse pair'' if each is the reverse of the
other.  Write a program that finds all the reverse pairs in the
word list.

Solution: \ref{sol_reverse_pair}.

\end{exercise}

\begin{exercise}
\index{interlocking words}
\label{interlock}

Two words ``interlock'' if taking alternating letters from 
each forms a new word.  For example, ``shoe'' and ``cold''
interlock to form ``schooled''.

Write a program that finds in our word list all pairs of 
words that interlock. Hint: don't enumerate all pairs!

Solution: \ref{sol_interlock}

Credit: This exercise is inspired by an example at \url{http://puzzlers.org}.

\end{exercise}


