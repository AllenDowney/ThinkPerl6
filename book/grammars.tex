


\chapter{Regexes and grammars}
\label{regex_grammars}
\index{regex}
\index{regular expression}
\index{grammar}

Regular expressions or regexes have been introduced in 
sections~\ref{regex} to \ref{substitutions} (p. \pageref{regex} 
to \pageref{regex_adverbs}) of Chapter~\ref{strings} 
about strings. You might want to review those sections 
of this book before reading this chapter if you don't 
remember much about regexes. You don't need to remember the 
details of everything we have covered earlier and we will explain 
again briefly specific parts of the functionality that we 
will be using, but you are expected to understand generally 
how regexes work.

\section{A brief reminder}

\index{pattern}
As you hopefully remember, regexes, as we have studied 
them so far, are about string exploration using patterns. 
A pattern is a sequence of (often special) characters that 
is supposed to describe a string or part of a string. A 
pattern matches a string if a correspondence can be found 
between the pattern and the string. 

For example, the 
following code snippet looks into the string for the letter 
``a'', followed by any number (but at least one) of letters 
``b'' or ``c'', followed by zero or more digits followed by
a ``B'' or a ``C'':

\begin{verbatim}
my $str = "foo13abbccbcbcbb42Cbar";
say ~$/ if $str ~~ /a <[bc]>+ (\d*) [B|C]/;  # -> abbccbcbcbb42C
say ~$0;                                     # -> 42
\end{verbatim}

\index{smart match operator}
\index{operator!smart match}
This code uses the \verb'~~' smart match operator to 
check whether the \verb'$str' string matches the 
\verb'/a <[bc]>+ (\d*) [B|C]/' pattern. Remember that 
spaces are usually not significant in a regex pattern 
(unless specified otherwise).

The pattern is made of the following components:
\begin{itemize}

\item \verb'a': a literal match of letter ``a'';

\index{character class}
\item \verb'<[bc]>+': the \verb'<[bc]>' is a character class 
meaning letter ``b'' or ``c''; the \verb'+' quantifier 
says characters matching the character class ``b'' or ``c'' 
can be repeated one or more times;

\index{quantifier}
\index{capture, regex}
\item \verb'(\d*)': the \verb'\d' atom is a digit character 
class, the \verb'*' quantifier means 0 or more occurrences 
of the previous atom, and the enclosing parentheses request 
a capture of these digits (if any) into the \verb'$0' 
variable (a special variable that is really a shortcut 
for \verb'$/[0]');

\index{alternation}
\item \verb'[B|C]': \verb'B|C' is an alternation (either 
a ``B'' or a ``C''), and the square brackets regroup this 
alternation into one subpattern (and also enable proper 
precedence).

\end{itemize}

\index{match object}
If the match is successful (as is the case in this example), 
the result is stored into the \emph{match object}, \verb'$/'.
Printing \verb'~$/' displays a stringified version of the 
match object. And printing \verb'$0' (or \verb'$/[0]') 
displays the capture (part of the match that is between 
parentheses, in this case the number ``42'').

\index{lexing}
\index{parsing}
This is what might be called low-level matching: pattern 
recognition is done mostly at the individual character level.
Perl~6 offers ways to group and name regex patterns so that 
these individual patterns can then be used as building 
blocks for higher level matching: recognizing words and 
sequences of words (rather than just characters), for the 
purpose of performing what is called lexical analysis 
(or lexing) and grammatical analysis or parsing on a piece 
of text. 

\index{Perl 6 grammar}
This chapter is mostly devoted to this higher type 
of matching, leading to the creation of full-fledged grammars 
that can analyze structured text such as XML or HTML texts, 
JSON or YAML documents or even computer programs: Perl~6 
programs are actually parsed using a Perl~6 grammar written 
in Perl~6.

Grammars are a very important topic in computer science, 
but, obviously, most programmers don't commonly write 
full-fledged grammars for parsing programming languages. 
However, writing a simple grammar and a simple parser 
might be, or perhaps should be, a much more common task. 

Quite often, people spend a lot of effort at deciphering a 
simple configuration file with low level techniques, whereas 
writing simple parser might be a lot easier and much more 
efficient. Perl~6 offers all the tools to do that 
very easily.

\index{domain-specific language}
\index{DSL}
Sometimes, you also need to develop a domain-specific 
language (DSL), i.e. a usually relatively small sub-language 
adapted to a specific field of knowledge (scientific, 
engineering, business, art, or other) having its own conventions, 
symbols, operators and so on. With a grammar and Perl's 
ability to create its own operators, you can often express 
specialized knowledge within the terminology framework of 
subject-matter expert.

\section{Declarative programming, another programming paradigm}

\index{declarative programming}
\index{programming!declarative}
Both regexes and grammars are examples of yet another programming 
paradigm that we haven't really explored so far: declarative 
programming. This is a programming model in which, contrary 
to ordinary imperative or procedural programming, you don't 
state how to do something and don't choose your control flow. 
Rather, you specify a set of definitions, rules, properties 
and possibly some constraints and actions, and let the program 
apply those to derive some new information about the input 
data. 

\index{programming!declarative}
\index{programming!logic}
\index{programming!functional}
\index{artificial intelligence}
\index{database query language}
\index{compilation}
\index{Yacc}
\index{Bison}
\index{makefile}

This form of programming is widely used in logic programming 
(e.g. Prolog), artificial intelligence, expert systems, 
data analysis, database query languages (e.g. SQL), text and 
code source recognition (e.g. Lex and Flex), program 
compilation (e.g. Yacc or Bison), configuration management, 
makefiles, and also in some ways functional programming.



\section{Captures}

\index{capturing}
As we have seen in the regex examples at the beginning 
of this chapter, round parentheses 
not only group things together, but also capture data: they 
make the string matched by the subpattern within 
the parentheses available as a special variable:

\begin{verbatim}
my $str =  'number 42';
say "Number is $0" if $str ~~ /number \s+ (\d+) /;  # -> Number is 42
\end{verbatim}
%

\index{numbered capture}
\index{capture!numbered}
Here, the pattern matched the \verb'$str' string and the 
part of the pattern within parentheses was captured into 
the \verb'$0' special variable. Where there are several 
parenthesized groups, they are captured into variables 
named \verb'$0', \verb'$1',  \verb'$2', etc. (from 
left to right):

\begin{verbatim}
say "$0 $1 $2" if "abcde" ~~ /(a) b (c) d (e)/;       # -> a c e
\end{verbatim}
%

This is fine for simple captures, but the numbering of 
captures can become tedious if there are many captures and 
somewhat complicated when there are nested parentheses in 
the pattern:

\begin{verbatim}
if 'abc' ~~ / ( a (.) (.) ) / {
    say "Outside: $0";                   # Outside: abc
    say "Inside: $0[0] and $0[1]";       # Inside: b and c
}
\end{verbatim}

When it gets complicated, it is often better to use another 
feature called \emph{named captures}. The standard way to 
name a capture is as follows:
\index{named!capture}
\index{capture!named}

\begin{verbatim}
if 'abc;%' ~~ / $<capture_name> = \w+ / {
    say ~$<capture_name>;                # abc
}
\end{verbatim} 

The use of the named capture, \verb'$<capture_name>', 
is a shorthand for accessing the \verb'$/' match object as 
a hash, in other words: \verb"$/{ 'capture_name' }" or 
\verb'$/<capture_name>'.

Named captures can be nested using regular capture group syntax:

\begin{verbatim}
if 'abc' ~~ / $<overall>=( a $<part1>=(.) $<part2>=(.) ) / {
    say "Overall: $<overall>";           # Overall: abc
    say "Part 1: $<overall><part1>";     # Part 1: b
    say "Part 2: $<overall><part2>";     # Part 2: c
}
\end{verbatim} 

\index{match object}
Assigning the match object to a hash gives you easy programmatic 
access to all named captures:

\begin{verbatim}
if 'abc' ~~ / $<overall>=( a $<part1>=(.) $<part2>=(.) ) / {
    my %capture = $/.hash;    
    say ~%capture<overall>;              # -> abc
    for kv %capture<overall> -> $key, $val {
        say $key, " ", ~$val;            # -> part2 c \n part1 b
    }
}
\end{verbatim} 

But you might as well do the same thing directly on the 
match object without having to perform an extra hash 
assignment:

\begin{verbatim}
if 'abc' ~~ / $<overall>=( a $<part1>=(.) $<part2>=(.) ) / {
     say "Overall: $<overall>";          # -> Overall: abc
    for kv %<overall> -> $key, $val {
        say $key, " ", ~$val;            # -> part2 c \n part1 b
    }
}
\end{verbatim}

Remember that, in the above code, \verb'$<overall>' is 
really a shortcut for  \verb'$/<overall>', i.e. for a 
hash type of access to the \verb'$/' match object.

There is, however, a more convenient way to get named 
captures which is discussed in the next section on 
named rules.

\section{Subrules or named rules}
\label{subrules}
\index{subrule}
\index{named!rule}
\index{named!regex}
\index{named!token}

It is possible to store pieces of regexes into \emph{named rules}. The following example uses a named regex, which 
is one of the kinds of regex rules, to match a text line:

\begin{verbatim}
my regex line { \N* \n }  # any number of characters other 
                          # than new lines, followed by 1 new line
if "abc\ndef" ~~ /<line> def/ {
    say "First line: ", $<line>.chomp;      # First line: abc
}
\end{verbatim} 

Notice that the syntax with a block of code is akin 
to a subroutine or method definition. This is not a 
coincidence, we will see that named rules are very 
similar to methods. Especially, rules can call 
each other (or even in some cases call themselves 
recursively) just like methods and subroutines, and we will see 
that this is a very powerful and expressive feature.

\index{named!regex}
A named regex can be declared with 
\verb'my regex name { regex body }', and called with 
{\tt <name>}. 

As it can be seen in the example above, calling a named 
regex creates a named capture with the same name. If you 
need a different name for the capture, you can do this 
with the syntax {\tt <capturename=regexname>}. For example,
in this example, we call twice the same named regex and, 
for convenience, use a different name to distinguish 
the two captures:

\begin{verbatim}
my regex line { \N* \n }
if "abc\ndef\n" ~~ / <first=line> <second=line> / {
    say "First line: ", $<first>.chomp;   # -> First line: abc
    say "Second line: ", $<second>.chomp; # -> Second line: def
    print $_.chomp for $<line>.list;      # -> abc  def
}
\end{verbatim}

Here, we have used {\tt chomp} method calls to remove 
the new line characters from the captures. There is in 
fact a way to match on the new line character but exclude 
it from the capture:
\begin{verbatim}
my regex line { \N* )> \n }
if "abc\ndef\n" ~~ / <first=line> <second=line> / {
    say "First line: ", ~$<first>;       # -> First line: abc
    say "Second line: ", ~$<second>;     # -> Second line: def
    print $<line>.list;                  # -> abc  def
}
\end{verbatim}

This relatively little-known token, "\verb')>'" marks 
the endpoint of the match's overall capture. Anything 
after it will participate to the match but will not 
be captured by the named regex. Similarly, the "\verb'<)'" 
token indicates the start of the capture.

Named regexes are only one form (and probably not the 
most common) of the named rules, which 
come in three main flavors:
\begin{itemize}
\item named regex, in which the regex behaves like ordinary 
regexes;
\index{ratchet}
\index{adverb!:ratchet}
\index{backtracking}
\index{token}
\index{named!regex}
\item named tokens, in which the regex has an implicit 
{\tt :ratchet} adverb, which means that there is no 
backtracking;
\index{named!token}
\index{ratchet}
\index{adverb!:ratchet}
\index{backtracking}
\index{sigspace}
\index{adverb!:sigspace}
\index{named!rule}
\item named rules, in which the regex has an implicit 
{\tt :ratchet} adverb, just as named tokens, and also 
an implicit {\tt :sigspace} adverb, which means that 
whitespace within the pattern is not ignored.
\end{itemize}

In the two examples above, we probably did not need the 
regexes to backtrack. We could have used a named token 
instead of a named regex:

\begin{verbatim}
my token line { \N* \n }
if "abc\ndef" ~~ /<line> def/ {
    say "First line: ", $<line>.chomp;    # First line: abc
}
\end{verbatim} 

But we would have to remove the space from \emph{within the pattern} for a rule to match:

\begin{verbatim}
my rule line { \N*\n }
if "abc\ndef" ~~ /<line> def/ {
    say "First line: ", $<line>.chomp;    # First line: abc
}
\end{verbatim} 

Collectively, they are usually referred to as rules, 
independently of the keyword used for their definition.

Remember the various regexes we experimented for 
extracting dates from a string in 
subsection~\ref{extracting_dates} 
(p. \pageref{extracting_dates})? The last example was 
using subpatterns as building blocks for constructing 
the full pattern. We could now rewrite it, with the 
added feature of recognizing a multi-format date, as 
follows:

\index{matching a date}
\begin{verbatim}
my $string = "Christmas : 2016-12-25.";                                         
my token year { \d ** 4 }                                        
my token month {   
    1 <[0..2]>                            # 10 to 12                     
    || 0 <[1..9]>                         # 01 to 09                     
};
my token day { (\d ** 2) <?{1 <= $0 <= 31 }> }  
my token sep { '/' || '-' } 
my rule date {  <year> (<sep>) <month> $0 <day> 
                || <day> (<sep>) <month> $0 <year> 
                || <month>\s<day>',' <year>
}                         

if $string ~~ /<date>/ {
    say ~$/;                              # -> 2016-12-25
    say "Day\t= "   , ~$/<date><day>;     # -> 25
    say "Month\t= " , ~$/<date><month>;   # -> 12
    say "Year\t= "  , ~$/<date><year>;    # -> 2016
}          
\end{verbatim} 

The first four named regexes define the basic building 
blocks for matching the year, the month, the day and 
possible separators. Then, the {\tt date}  
named rule uses these building blocks to define an 
alternation between three possible date formats.

\index{date validation}
This checks that the day in the month is between 0 and 31 
and that the month is between 01 and 12, and this is 
probably sufficient to recognize dates in a 
text, but this would recognize ``2016-11-31'' as a date, 
although November only has 30~days. We may want to be a 
little bit more strict about valid dates and prevent that
by adding a code assertion to the {\tt date} named rule:
\index{code assertion}
\index{assertion!code}

\begin{verbatim}
my rule date { [    <year> (<sep>) <month> $0 <day> 
                 || <day> (<sep>) <month> $0 <year> 
                 || <month>\s<day>',' <year>
               ] <!{ $<day> > 30 and $<month> ==  2|4|6|9|11}>
}                         
\end{verbatim}

\begin{exercise}
\label{february_rule}
This is better, but we can still match an invalid date 
such as ``2016-02-30''. As an exercise, change the code 
assertion to reject a ``Feb, 30'' date. If you feel 
courageous, you might even want to check the number of 
days in February depending on whether the date occurs in 
a leap year. You may also want to try to define and test 
other date formats. Solution: \ref{sol_february_rule}
\end{exercise}

Rules can (and usually should) be grouped in 
grammars, that's in fact what they have been designed 
for.

\section{Grammars}
\index{grammar}

Grammars are a powerful tool used to analyze textual data 
and often to return data structures that have been 
created by interpreting that text.

For example, any Perl~6 program is parsed and executed 
using a Perl~6 grammar written in Perl~6, and you could write a grammar 
for parsing (almost) any other programming language. 
To tell the truth, most programmers don't commonly 
write grammars for parsing programming languages. But 
grammars are very useful for many more common tasks.

\index{parsing!HTML}
\index{parsing!XML}
\index{HTML parsing}
\index{XML parsing}

If you ever tried to use regexes for analyzing a piece 
of HTML (or XML) text\footnote{Don't try to do it.}, you 
probably found out that this is quickly becoming next 
to impossible, except perhaps for the most simple 
HTML data. For analyzing any piece of such data, you 
need an actual parser which, in turn, will usually be 
based on an underlying grammar.

If you didn't like grammar in school, don't let that 
scare you off grammars. Grammars are nothing complicated, 
they just allow you to group named rules, just as classes 
allow you to group methods of regular code.

\index{namespace}
\index{parse method}
\index{fileparse method}
\index{grammar!methods}
\index{actions!class}
A grammar creates a namespace and is introduced with the 
keyword {\tt grammar}. It usually groups a number of 
named rules, just in the same way as a class groups 
a number of methods. A grammar is actually a class that 
inherits from the {\tt Grammar} superclass, which provides 
methods such as {\tt parse} to analyze a string and 
{\tt .parsefile} to analyze a file. Moreover, although 
we will not do it in the scope of this book, you can 
actually write some methods in a grammar, and even import 
some roles. And, as we shall see, grammars are often 
associated with some actions classes or objects.

\index{TOP rule}
Unless told otherwise, the parsing methods will look for 
a default rule named ``TOP'' (which may be a named regex, 
token or rule) to start the parsing. The date parsing rules 
used above might be assembled into a grammar as follows:

\index{grammar!date}
\label{dategrammar}
\begin{verbatim}
grammar My-date {
    rule TOP { \s*? 
               [    <year> (<sep>) <month> $0 <day>
                 || <day> (<sep>) <month> $0 <year> 
                 || <month>\s<day>',' <year>                     
               ] \s* 
               <!{ ($<day> > 30 and $<month> ==  2|4|6|9|11)}>  
             }
    token year  { \d ** 4 }                                        
    token month {  1 <[0..2]> || 0 <[1..9]> }                
    token day   { (\d ** 2) <?{1 <= $0 <= 31 }> }  
    token sep   { '/' || '-' } 
}                         

for " 2016/12/25 ", " 2016-02-25 ", " 31/04/2016 " -> $string {
	my $matched = My-date.parse($string);
	say ~$matched if defined $matched;
}
\end{verbatim}

This will print out:
\begin{verbatim}
 2016/12/25
 2016-02-25
\end{verbatim}

The code assertion within the ``TOP'' rule prevents invalid 
dates such as ``31/04/2016'' from being matched, you would 
need to add some code for handling the end of February dates,
as we did in the solution to the previous exercise (see 
\ref{sol_february_rule}) if this is important. You may 
want to do it as an exercise.

Besides that, this code is not very different from our 
earlier code, but there are a few changes which are 
significant.

\index{namespace}
\index{lexical scope}
\index{my!declarator}

We renamed the {\tt date} rule as {\tt TOP} because this 
is the default name searched by {\tt parse} for the top level 
rule. A grammar is creating its own namespace and 
lexical scope, and we no longer need to declare our rules 
with the {\tt my} declarator (which is required for
rules declared outside of a grammar). 

Within a grammar, the order in which the rules are 
defined is generally not relevant, so that we could define 
the {\tt TOP} rule first, even though it uses tokens 
that are defined afterwards (which again would have not 
been possible with rules used outside a grammar). This is 
important because, within a grammar, you can have many rules 
that call each other (or rules that call themselves 
recursively), which would be unpractical if the order of 
the rule definitions mattered.

\index{parse method}
\index{subparse method}
If you're parsing the input string with the {\tt .parse} method,
the {\tt TOP} rule is automatically anchored to the start and end 
of the string, which means that the grammar has to match 
the whole string to be successful. This is why we had to 
add patterns for spaces at the beginning and at the end of 
our {\tt TOP} rule to match our strings which have some 
spaces before and after the date itself. There is another 
method, {\tt .subparse}, which does not have to reach the 
end of the string to be successful, but we would still need to 
have the space pattern at the beginning of the rule.

\section{Grammar inheritance}
\index{inheritance!grammar}
\index{grammar!inheritance}
\index{grammar!Message}

A grammar can inherit from another grammar, just as a 
class can inherit from another class.

Consider this very simple (almost simplistic) grammar 
for parsing a mail message:

\begin{verbatim}
grammar Message {
    rule  TOP    { <greet> $<body>=<line>+? <end> }
    rule greet    { [Hi||Hello||Hey] $<to>=\S+? ',' }
    rule end      { Later dude ',' $<from>=.+ }
    token line    { \N* \n}
}
\end{verbatim}

We can test it with the following code:

\begin{verbatim}
my $msg = "Hello Tom,
I hope you're well and that your car is now repaired.
Later dude, Liz";

my $matched = Message.parse($msg);
if defined $matched { 
    say "Greeting \t= ", ~$matched<greet>.chomp;
    say "Addressee\t= $matched<greet><to>";
    say "Author   \t= $matched<end><from>";
    say "Content  \t= $matched<body>";
}
\end{verbatim}

This will print out the following:

\begin{verbatim}
Greeting        = Hello Tom,
Addressee       = Tom
Author          = Liz
Content         = I hope you're well and that your car is now repaired.
\end{verbatim}

Suppose now that we want a similar grammar for parsing 
a more formal message and we figure out that we could 
reuse part of the {\tt Message} grammar. We can have our 
new child grammar inherit from the existing parent:

\index{grammar!Message}
\index{grammar!FormalMessage}
\begin{verbatim}
grammar FormalMessage is Message {
    rule greet { [Dear] $<to>=\S+? ',' }
    rule end { [Yours sincerely|Best regards] ',' $<from>=.+ }
}
\end{verbatim}

The {\tt is Message} trait in the header tells Perl that 
{\tt FormalMessage} should inherit from the {\tt Message} grammar. 
Only two rules {\tt greet} and {\tt end} need to be 
redefined; the others (the {\tt TOP} rule 
and the {\tt line} token) will be inherited from the 
{\tt Message} grammar.

Let's try some code to run it:

\begin{verbatim}
my $formal_msg = "Dear Thomas,
enclosed is our invoice for June 2016.
Best regards, Elizabeth.";
my $matched2 = FormalMessage.parse($formal_msg);
if defined $matched2 { 
    say "Greeting \t= ", ~$matched2<greet>.chomp;
    say "Addressee\t= $matched2<greet><to>";
    say "Author   \t= $matched2<end><from>";
    say "Content  \t= $matched2<body>";
}
\end{verbatim}

which will print:

\begin{verbatim}
Greeting        = Dear Thomas,
Addressee       = Thomas
Author          = Elizabeth.
Content         = enclosed is our invoice for June 2016.
\end{verbatim}

\section{Actions objects}

\index{actions!object}
\index{actions!class}
\index{action method}
\index{reduction method}
\index{parse tree}
\index{match object}
\label{actions_object}

A successful grammar match gives you a parse tree of 
match objects (objects of Match type). This tree 
recapitulates all the individual 
``sub-matches'' that contributed to the overall match, so 
that it can quickly become very large and complicated. 
The deeper that match tree gets, and the more branches in 
the grammar there are, the harder it becomes to navigate the 
match tree to get the information you are actually 
interested in.

\index{named!rule}
\index{abstract syntax tree (AST)}
\index{AST, abstract syntax tree}
To avoid the need for diving deep into a match tree, you 
can supply an actions object. After each successful match 
of a named rule in your grammar, it tries to call a method 
having the same name as the grammar rule, giving it the 
newly created Match object as a positional argument. 
If no such method exists, it is skipped. Action methods 
are sometimes also called reduction methods. If it exists, 
the action method is often used to construct an abstract 
syntax tree (AST), i.e. a data structure presumably 
simpler to explore and to use than the match object tree, 
or it can do any other thing deemed to be useful.

In this somewhat simplistic example of a basic arithmetic  
calculator, the actions don't try to build an AST but simply 
do the bulk of the calculation work between the various tokens
matched by the grammar:
\index{arithmetic calculator}
\index{grammar!arithmetic calculator}

\begin{verbatim}
grammar ArithmGrammar {
    token TOP { \s* <num> \s* <operation> \s* <num> \s*}
    token operation { <[^*+/-]> }
    token num { \d+ | \d+\.\d+ | \.\d+ }
}
class ArithmActions {
    method TOP($/) {
        given $<operation> {
            when '*' { $/.make([*] $/<num>)}
            when '+' { $/.make([+] $<num>)}
            when '/' { $/.make($<num>[0] / $<num>[1]) }
            when '-' { $/.make([-] $<num>) }
            when '^' { $/.make($<num>[0] ** $<num>[1]) }
        }
    }
}
for '   6*7  ', '46.2 -4.2', '28+ 14.0 ',
    '70 * .6 ', '126   /3', '6.4807407 ^ 2' -> $op {
        my $match = ArithmGrammar.parse($op, :actions(ArithmActions));
        say "$match\t= ", $match.made;
}
\end{verbatim}

This prints the following output:

\begin{verbatim}
   6*7          = 42
46.2 -4.2       = 42
28+ 14.0        = 42
70 * .6         = 42
126   /3        = 42
6.4807407 ^ 2   = 42.00000002063649
\end{verbatim}

The aim of this example is not to describe how to implement 
a basic calculator (there are better ways to do that, we'll 
come back to that), but only to show how actions may be 
used in conjunction with a grammar.

The grammar is quite simple and is looking for two decimal numbers 
separated by an infix arithmetic operator. If there is a 
match, \verb'$/<num>' (or \verb'$<num>' for short) will refer to an 
array containing the two numbers (and \verb'$/<operation>' 
will contain the arithmetic operator).

\index{parse method}
\index{actions!object}
\index{actions!class}
The {\tt parse} 
method is called with an {\tt actions:} named argument, 
the {\tt ArithmActions} class, which tells Perl which 
action object to use with the grammar. In this example, we 
don't really pass an action object, but simply the name 
of the actions class (actually a type object), 
because there is no need to instantiate an object. In 
other cases, for example if there was a need to initialize 
or somehow use some object attributes, we would need to pass 
an actual object that would have to be constructed beforehand.

\index{TOP rule}
\index{make method}
\index{AST}
\index{made method}
Whenever the {\tt TOP} rule succeeds, the {\tt TOP} method 
of class {\tt ArithmActions} is invoked with the match 
object for the current rule as the argument.  This method 
calls the {\tt make} method on the match object and 
returns the result of the actual arithmetic operation 
between the two numbers. Then, the {\tt made} method 
in the caller code (within the {\tt for} loop) returns 
that result.

\section{A grammar for parsing JSON}
\index{JSON grammar}
\index{grammar!JSON}
\index{parsing!JSON}

JSON (or \emph{Java-Script Object Notation}) is an 
open-standard format for text data derived from 
the object notation in the JavaScript programming language. 
It has become one of the commonly-used standards for 
serializing data structures, which makes it possible, for 
example, to exchange them between different platforms and 
different programming languages, to send them over a network, 
or to store them permanently in files on disks.

\subsection{The JSON format}
\index{JSON!format}
\index{JSON!object}
\index{JSON!array}

The JSON format is quite simple and is composed of two types 
of structural entities:
\begin{itemize}
\item objects or unordered lists of name-value pairs 
(basically corresponding to hashes in Perl);
\item arrays, or ordered lists of values.
\end{itemize}

\index{JSON!base types}
\index{JSON!string}
\index{JSON!Boolean}
\index{JSON!number}
\index{JSON!value}
Values can be either (recursively) objects or arrays 
as defined just above, or basic data types, which are: 
strings, numbers, Boolean (true or false) and \emph{null} 
(empty value or undefined value). A string is a sequence 
of Unicode characters between quote marks, and numbers 
are signed decimal numbers that may contain a fractional 
part and may use exponential ``E'' notation.

\subsection{Our JSON sample}
\index{JSON!sample}

To illustrate the format description above and for the 
purpose of our tests, we will use an example borrowed from 
the Wikipedia article on JSON 
(\url{https://en.wikipedia.org/wiki/JSON}), which is a 
possible JSON description of a person:

\begin{verbatim}
{
  "firstName": "John",
  "lastName": "Smith",
  "isAlive": true,
  "age": 25,
  "address": {
    "streetAddress": "21 2nd Street",
    "city": "New York",
    "state": "NY",
    "postalCode": "10021-3100"
  },
  "phoneNumbers": [
    {
      "type": "home",
      "number": "212 555-1234"
    },
    {
      "type": "office",
      "number": "646 555-4567"
    },
    {
      "type": "mobile",
      "number": "123 456-7890"
    }
  ],
  "children": [],
  "spouse": null,  
  "Bank-account": {
    "credit": 2342.25
}
\end{verbatim}

\index{JSON!number}
We've just added a {\tt Bank-account} object to provide the 
possibility of testing JSON non-integer numbers.
 
\subsection{Writing the JSON grammar step by step}


\subsubsection{Numbers}
\index{JSON!number}

The example JSON document only has integers and decimal numbers, 
but we need to be able to recognize numbers such as ``17'', 
``-138.27'', ``1.2e-3'', ``.35'', etc. We can use the following rule:

\begin{verbatim}
token number {
    [\+|\-]?              # optional sign
    [ \d+ [ \. \d+ ]? ]   # integer part and optional fractional part
      | [ \. \d+ ]        # or only a fractional part
    [ <[eE]> [\+|\-]? \d+ ]?    # optional exponent
}
\end{verbatim}

\subsubsection{JSON strings}
\index{JSON!string}

There are many possible patterns to define a string. For 
the sample JSON document used as an example, the following 
rule will be sufficient:

\begin{verbatim}
token string {
    \" <[ \w \s \- ' ]>+ \" 
}
\end{verbatim}

This will match a double-quoted sequence of alphanumeric 
characters, spaces, dashes and apostrophes.

For a real JSON parser, a rule using a negative character 
class excluding anything that cannot belong to a string 
might be better, for example:

\begin{verbatim}
token string {
    \" <-[\n " \t]>* \"
}
\end{verbatim}

i.e. a double quoted sequence of any characters other than 
double quotes, newlines and tabulations.

You might want to study the JSON standard to figure out 
exactly what is accepted or forbidden in a JSON string. 
For our purpose, the first rule above will be sufficient.

\subsubsection{JSON objects}
\index{JSON!object}
\index{key-value pair}
\index{JSON!value}

JSON objects are lists of key-value pairs. Lists are 
delimited by curly brackets and pairs separated by 
commas. A key-value pair is a string followed by a colon, 
followed by a value (to be defined later). This can be 
defined as follows:

\begin{verbatim}
rule object     { '{'  <pairlist> '}' }
rule pairlist   { [<pair> [',' <pair>]*] }
rule pair       { <string> ':' <value>  }
\end{verbatim}

\index{modified quantifier}
We can use a regex feature that we haven't seen yet, the 
quantifier modifier, to simplify the {\tt pairlist} 
rule. To more easily match things like comma separated 
values, you can tack on a \verb'%' modifier to any of the regular quantifiers to specify a separator that must occur between each of the matches. So, for example \verb"/a+ % ','/" will match ``a'' or 
``a,a'' or ``a,a,a'', etc.

Thus, the {\tt pairlist} rule can be rewritten as follows:

\begin{verbatim}
rule pairlist   {<pair> + % \,}
\end{verbatim}

or:

\begin{verbatim}
rule pairlist   {<pair> * % \,}
\end{verbatim}

if we accept that a {\tt pairlist} may also be empty.

\subsubsection{JSON arrays}
\index{JSON!array}

Arrays are comma-separated lists of values between square 
brackets:

\begin{verbatim}
rule array       { '[' <valueList> ']'}
rule valueList {  <value> * % \, }
\end{verbatim}

Here, we used again the modified quantifier shown 
just above.

\subsubsection{JSON values}
\index{JSON!value}

Values are objects, arrays, string, numbers, Booleans 
(true or false) or \emph{null}.

\begin{verbatim}
token value { | <object> | <array> | <string> | <number> 
               | true    | false     | null 
    }
\end{verbatim}

\subsection{The JSON grammar}
\index{JSON grammar}

We have defined all the elements of the JSON grammar, we 
only need to declare a grammar and to add a {\tt TOP} rule 
to complete it:

\begin{verbatim}
grammar JSON-Grammar {
    token TOP      { \s* [ <object> | <array> ] \s* }
    rule object    { '{' \s* <pairlist> '}' \s* }
    rule pairlist  {  <pair> * % \, }
    rule pair      {  <string>':' <value> }
    rule array     { '[' <valueList> ']'}
    rule valueList {  <value> * % \, }
    token string   {  \" <[ \w \s \- ' ]>+ \"  }
    token number   { 
      [\+|\-]?  
      [ \d+ [ \. \d+ ]? ] | [ \. \d+ ]  
      [ <[eE]> [\+|\-]? \d+ ]?
    }
    token value    { <object> | <array> | <string> | <number> 
                     | true    | false     | null 
    }
}
\end{verbatim}

We can now test the grammar with our sample JSON string 
and try to print the match object:

\begin{verbatim}
my $match = JSON-Grammar.parse($JSON-string);
say ~$match if $match;
\end{verbatim}

This produces the following output:

\begin{verbatim}
{
  "firstName": "John",
  "lastName": "Smith",
  "isAlive": true,
  "age": 25,
  "address": {
    "streetAddress": "21 2nd Street
    "city": "New York",
    "state": "NY",
    "postalCode": "10021-3100"
  },
  "phoneNumbers": [
    {
      "type": "home",
      "number": "212 555-1234"
    },
    {
      "type": "office",
      "number": "646 555-4567"
    },
    {
      "type": "mobile",
      "number": "123 456-7890"
    }
  ],
  "children": [],
  "spouse": null,
  "Bank-account": {
        "credit": 2342.25
  }
}
\end{verbatim}

The sample JSON document has been fully matched. This 
JSON grammar works perfectly on it, and takes less than 
20~lines of code. If you think about it, this is 
really powerful. Test it for yourself. Try to change 
the grammar in various places to see if it still works. 
You could also try to introduce errors into the JSON 
document (for example to remove a comma between two values 
of a list) and the match should no longer occur.

It may be objected that this grammar covers only a subset 
of JSON. This is sort of true, but not really: it is almost complete. 
True, it would not be recommended to use this grammar 
in a production environment for parsing JSON documents, 
because it has been built only for pedagogical purposes 
and may not comply with every single fine detail of the JSON 
standard.

Take a look at the grammar of the Perl~6 {\tt JSON::Tiny} 
module (\url{https://github.com/moritz/json}), which can 
parse any valid JSON document. It is not much more 
complicated (except for the use of proto regexes, a topic 
that we haven't covered here), and it is not much longer, 
as it holds in about 35~code lines.

\subsection{Adding actions}
\index{actions!object}
\index{actions!class}

The JSON grammar works fine, but printing out the tree of 
parse objects just for our relatively small JSON document 
will display about 300 lines of text, as 
it provides all the details of everything that has been 
matched, rule by rule and subpattern by subpattern. This 
can be very useful to understand what the grammar does 
(especially when it does not work as expected), but 
exploring that tree to extract the data can be quite tedious.

\index{abstract syntax tree (AST)}
\index{AST, abstract syntax tree}
Let us add an actions class to build an abstract syntax tree 
(AST).

\begin{verbatim}
class JSON-actions {
    method TOP($/) {
        make $/.values.[0].made;
    };
    method object($/) {
        make $<pairlist>.made.hash.item;
    }
    method pairlist($/) {
        make $<pair>>>.made.flat;
    }
    method pair($/) {
        make $<string>.made => $<value>.made;
    }
    method array($/) {
        make $<valueList>.made.item;
    }
    method valueList($/) {
        make [$<value>.map(*.made)];
    }
    method string($/) { make ~$0 }
    method number($/) { make +$/.Str; }
    method value($/) { 
        given ~$/ {
            when "true"  {make Bool::True;}
            when "false" {make Bool::False;}
            when "null"  {make Any;}
            default { make $<val>.made;}
        }  
   }
}
\end{verbatim}

\index{named!capture}
For this actions class to work, we need to make a small change 
to the grammar. The {\tt value} method uses a {\tt val} named 
capture to access its content; we need to add it to the 
{\tt value} token:

\begin{verbatim}
token value { <val=object> | <val=array> | <val=string> 
              | <val=number> | true | false | null
}
\end{verbatim}

We can now call it with the following syntax:

\begin{verbatim}
my $j-actions = JSON-actions.new();
my $match = JSON-Grammar.parse($JSON-string, :actions($j-actions));
say $match.made;
\end{verbatim}

Notice that, here, we've used an actions object rather than the
actions class, but this is just for the purpose of showing 
how to do it, we could have used the class directly as before.

\index{abstract syntax tree (AST)}
\index{AST, abstract syntax tree}
The last statement in the above code prints out the abstract syntax 
tree (AST). We have reformatted the output to better show the 
structure of the AST :

\begin{verbatim}
{
    Bank-account => {
        credit => 2342.25
    }, 
    address => {
        city => New York, 
        postalCode => 10021-3100, 
        state => NY, 
        streetAddress => 21 2nd Street
    }, 
    age => 25, 
    children => [], 
    firstName => John, 
    isAlive => True, 
    lastName => Smith, 
    phoneNumbers => [
        {number => 212 555-1234, type => home} 
        {number => 646 555-4567, type => office} 
        {number => 123 456-7890, type => mobile}
    ], 
    spouse => (Any)
}
\end{verbatim}

In this case, the top structure is a hash (it could also have been 
an array with a different JSON input string). We can now explore 
this hash to find the data of interest for us. For example:

\begin{verbatim}
say "Keys are: \n", $match.made.keys;
say "\nSome values:";
say $match.made{$_} for <firstName lastName isAlive>;
say $match.made<address><city>;
say "\nPhone numbers:";
say $match.made<phoneNumbers>[$_]<type number> 
    for 0..$match.made<phoneNumbers>.end;
\end{verbatim}

which will display the following output:

\begin{verbatim}
Keys are:
(lastName Bank-account phoneNumbers children address age firstName spouse isAlive)

Some values:
John
Smith
True
New York

Phone numbers:
(home 212 555-1234)
(office 646 555-4567)
(mobile 123 456-7890)
\end{verbatim}

\section{Inheritance and mutable grammars}

\index{subclassing}
The capacity for a grammar to inherit from another one opens 
the door to very rich possibilities in terms of 
extending the Perl~6 language itself. It is possible, for 
example in the context of a module or a framework, to ``subclass'' 
the standard Perl grammar, i.e. to write a new child grammar that 
inherits from the standard Perl grammar, but adds a new 
feature, overloads an operator or modifies some other syntax 
element, and to run this program with the same Perl~6 compiler,
but with this locally modified grammar.

This means that it is actually possible to dynamically extend 
the language for new needs, perhaps without even changing the compiler 
or the virtual machine. These are however advanced topics which 
are more geared towards language gurus than to beginners. So 
we only mention these exciting possibilities with the hope of 
whetting your appetite and push you to study these further, 
but will not dwell further onto them in this book.

\section{Debugging}

Writing grammars is a lot of fun, but it can also be difficult 
or even tedious when you start.

When you started to practice programming with this book, you probably 
made a lot of small stupid mistakes that initially prevented your 
programs from compiling and running, or from doing what 
you expected. With practice, however, you hopefully gradually 
made fewer errors and spent less time chasing bugs.

When you begin to learn grammars (and to a lesser extent regexes), 
you may feel like starting again at square one. Even 
very good programmers often make silly mistakes when they start 
writing grammars. It is a different programming paradigm, and 
it requires a new learning phase.

\index{testing}
In this case, small is beautiful. Start with small regexes and 
small rules, and with small test input. Test individual regexes 
or rules under the REPL, and add them to your code only when 
you're confident that they do what you want.

Write test cases at the same time as your code (or actually even 
before you write the code), and make sure that you pass all the 
relevant tests before moving on. And add new tests when you add 
new rules.

One standard debugging technique is to add print statements 
to the code in order to figure out various information about 
the state of the program (such as the value of variables, the 
flow of execution of the program, etc.). You can also do that 
with regexes and grammars.

Let's take the example of the very simple grammar for 
matching dates of section \ref{dategrammar} and let's suppose 
that you have written that:

\begin{verbatim}
grammar My-Date {
    token TOP { \s* <year> '-' <month> '-' <day> }
    token year  { \d ** 4 }                                        
    token month {  1 <[0..2]> || 0 <[1..9]> }                
    token day   { (\d ** 2) <?{1 <= $0 <= 31 }> }  
}                         
my $string = " 2016-07-31 ";
say so My-Date.parse($string);                 # -> False
\end{verbatim}

This test fails.

At this point, it has already become a bit difficult to figure 
out why the grammar fails (unless we have thoroughly tested 
each of the three tokens before building the grammar, but 
let's assume for the sake of this discussion that we haven't).
Let's not try to randomly change things here or there and see 
if it works better, we would be likely to spend hours doing 
that and probably not get anywhere. Let's be more methodical.

Let's first test the building-block tokens, {\tt year}, 
{\tt month}, and {\tt day}. We've seen before that 
the {\tt parse} method looks \emph{by default} for the {\tt TOP} 
rule in the grammar, but you can specify another rule, and 
that's what we need here. We can test them individually:

\begin{verbatim}
say so My-Date.parse("2016", :rule<year>);    # -> True
say so My-Date.parse("07",   :rule<month>);   # -> True
say so My-Date.parse("31",   :rule<day>);     # -> True
\end{verbatim}

These three tokens seem to work fine. At this point, you 
might be guessing where the problem is, but let's assume 
you don't.

We need to debug the ``TOP'' token. We can just use the common 
debugging method of printing where we are at various stages 
of the program. You can insert a print statement block 
in a named rule. Let's try to change the TOP token to this:

\begin{verbatim}
    token TOP { \s* <year>  { say "matched year"; }
                '-' <month> { say "matched month";}
                '-' <day>   { say "matched day";  }
              }
\end{verbatim}
 
This displays the following output:

\begin{verbatim}
matched year
matched month
matched day
\end{verbatim}

So, even the ``TOP'' token seems to work almost to the end. At 
this point, we should be able to figure out that we lack 
final spacing in the ``TOP'' token. 

So either we should add an optional spacing at the end of 
the token:

\begin{verbatim}
token TOP { \s* <year> '-' <month> '-' <day> \s*}
\end{verbatim}

or change it to a rule:

\begin{verbatim}
rule TOP { \s* <year> '-' <month> '-' <day> }
\end{verbatim}

or it was possibly the test string that was wrong (because 
it wasn't supposed to have spaces) and needed to be fixed.

If you have an actions class, you can also add print statements 
to the actions methods.

\index{debugger}
\index{debugger!stepping through a regex}
Remember also that the Perl debugger (see 
section~\ref{perl-debugger}) can be very helpful. We have 
briefly shown in subsection~\ref{regex-debugging} 
(p.~\pageref{regex-debugging}) how to go step by step 
through a regex match. Most of what has been described 
there also applies to debugging grammars.

\index{debugger!debugging grammars}
Finally, there is a very good module, \verb'Regex::Tracer', for 
debugging regexes and grammars (\url{https://github.com/jnthn/grammar-debugger/}), working with Rakudo. If you add:

\begin{verbatim}
use Regex::Tracer;
\end{verbatim}

to your program, then any grammar within the lexical scope will 
print out debugging information about the rules which tried to 
match, those which succeeded and those which failed.

\section{Glossary}

\begin{description}

\index{grammar}
\item[grammar:] A high level tool for performing lexical and 
grammatical analysis of structured text. In Perl~6, 
more specifically a namespace containing a collection of 
named rules aimed at this type of analysis.

\index{lexing}
\item[lexing:] Performing a lexical analysis of a source text, and especially dividing it into ``words'' or tokens.

\index{parsing}
\item[parsing:] Performing a grammatical analysis of a source text, 
and especially assembling words or tokens into sentences or 
expressions and statements that make some semantic sense. 

\index{declarative programming}
\index{programming!declarative}
\item[declarative programming:] a programming model where you 
specify definitions, rules, properties, and constraints, rather 
than statements and instructions, and let the program derive 
new knowledge from these definitions and rules. Regexes and 
grammars are examples of declarative programming.

\index{match object}
\item[match object:] in Perl~6, an object (of type {\tt Match}), 
usually noted \verb'$/', which contains (sometimes very) 
detailed information about what was successfully matched 
by a regex or a grammar. The \verb'$/' match object will be 
set to {\tt Nil} if the match failed.

\index{capture}
\item[capture:] the fact that parts of the target string that 
are matched by a regex (or a grammar) can be retrieved through 
the use of a number of dedicated special variables.

\index{rule}
\item[rule:] in broad terms, named rules are regexes that use a 
method syntax and are usually stored in a grammar. More specifically, 
one category of these named rules (along with named regexes and 
tokens).

\index{AST}
\index{abstract syntax tree (AST)}
\index{AST, abstract syntax tree}
\item[Abstract Syntax Tree (AST):] a data structure often 
summarizing the match object and used for further exploitation 
of the useful data. The match object is populated automatically 
by Perl, whereas the AST contains information deemed useful by 
the programmer.

\index{actions!class}
\item[Actions class:] a class used in conjunction with a grammar 
to perform certain actions when a grammar rule matches 
something in the input data. If a method with the same name 
as a rule in the grammar exists in the actions class, it will 
be called whenever the rule matches.

\end{description}

\section{Exercise: A grammar for an arithmetic calculator}
\label{calculator}
\index{calculator}
\index{calculator!grammar}

The arithmetic calculator presented in section~\ref{actions_object} 
above is very simplistic. In particular, it can parse only 
simple arithmetic expressions composed of two operands separated 
by one infix operator.

We would like to be able to parse more complicated arithmetic 
expressions. Especially the calculator should be able to handle:
\begin{itemize}
\item Expressions with several different operators (among the four 
basic arithmetic operators) and multiple operands;
\item Standard precedence rules between operators (for example,
multiplications should be performed prior to additions);
\index{precedence}
\item Parentheses to override usual precedence rules.
\index{parenthese}
\end{itemize}

These are a few examples of expressions the calculator should 
parse and compute correctly:
\begin{verbatim}
3 + 4 + 5;
3 + 4 * 5;   # result should be 23
(3 + 4) * 5; # result should be 35 
\end{verbatim}

\begin{exercise}
Your mission, \verb'[Dan|Jim]', should you choose to accept it, 
is to write such a grammar. As usual, should you fail, the 
Government shall deny any knowledge of your actions class.

There are many possible ways to do that, the solution presented 
in section~\ref{sol_calculator} is only one of them.
\end{exercise}

